% Formatting tricks so we can fit two inference rules side by side
\renewcommand{\predicatebegin}{\hspace{-.5em}$}
\renewcommand{\predicateend}{\hspace{-.5em}$}

% \mathlig{-x}{\mathbin{\multimap}}
\mathlig{-x}{\mathbin{-\mspace{-8mu}\times}}
\newcommand\opfail[1]{\overset{{#1}}{-x}}
\mathlig{++}{\mathbin{{+}\mspace{-10mu}{+}}}
\mathlig{->>}{\twoheadrightarrow}
\mathlig{<-}{\leftarrow}

\reservestyle{\functions}{\texttt}
\functions{isValidElement,isValidTransaction,getElementFromTransaction,
getElementsFromBatch,Hash,isValidBatch, AddElement,isReady,revertHash,
shouldConsolidateHash,reset,newEpoch,Sign,isValidSignature}
\reservestyle{\setchainOp}{\texttt}
\setchainOp{add,get,theset[the\_set],epoch,history}
\reservestyle{\tendermint}{\texttt}
\tendermint{broadcastTx[Tendermint.Broadcast],abciquery[Tendermint.Query]}
\reservestyle{\abci}{\texttt}
\abci{CheckTx,DeliverTx,EndBlock,BeginBlock,Query,Commit}
\reservestyle{\variablesProg}{\texttt}
\variablesProg{epoch,history,element,transaction,batch}

\reservestyle{\txexec}{\texttt}
\txexec{nogas[no-gas],txsuc[tx-suc],txfail[tx-fail]}

% Operations
\reservestyle{\operations}{\mathsf}
\operations{dest,param,money,src}

% New opcode
\newcommand{\textbfup}[1]{\textup{\texttt{#1}}}
\reservestyle{\mech}{\textbfup}
%\mech{first,count,fail,setFail,isSelfOnly,txmem[transaction memory]}
\mech{first,count,fail,setFail,isSelfOnly,txmem[trmem],bstore,ustore,queue}

%Sets/Types
\reservestyle{\types}{\textit}
\types{Addr,SmartContract}

%Monitors
\newcommand{\Mon}{\textit{mon}}
\newcommand{\InitMon}{\ensuremath{\Mec{init}^{\Mec{mon}}}\xspace}
\newcommand{\BeginMon}{\ensuremath{\textit{begin}^{\textit{mon}}}}
\newcommand{\EndMon}{\ensuremath{\Mec{end}^{\Mec{mon}}}\xspace}
\newcommand{\TermMon}{\ensuremath{\Mec{term}^{\Mec{mon}}}\xspace}
\newcommand{\SchMon}{\ensuremath{\textit{Sch}^{\textit{mon}}}}
\newcommand{\StMon}{\ensuremath{\textit{St}^{\textit{mon}}}}
\newcommand{\TStorageMon}{\ensuremath{\TStorage^{\textit{mon}}}}
\newcommand{\GMon}{\TStorageMon}

\newcommand{\MON}{\textit{MON}}
\newcommand{\TMon}{\ensuremath{\textit{T}^{\textit{mon}}}}
\newcommand{\RunMon}{\ensuremath{\textit{Run}^{\textit{mon}}}}

%\reservestyle{\monitors}{\mathsf}
\reservestyle{\monitors}{\mathit}
\monitors{init,term,begin,end,hookup,fst}

\reservestyle{\infrule}{\textsc}
\infrule{einit,emon,eterm,efail,failinit,failterm,failop,failtx,
   opok[op-ok],opfail[op-fail],bfail[op-begin],efail[op-end]}


%ABCI
\newcommand{\BeginBlock}{\<BeginBlock>\xspace}
\newcommand{\EndBlock}{\<EndBlock>\xspace}
\newcommand{\CheckTx}{\<CheckTx>\xspace}
\newcommand{\Commit}{\<Commit>\xspace}
\newcommand{\DeliverTx}{\<DeliverTx>\xspace}
\newcommand{\Query}{\<Query>\xspace}

%Operation memory
\newcommand{\OpMem}{\Mec{Operation Memory}}
\newcommand{\opmem}{\<txmem>\xspace}
\newcommand{\InitOm}{\ensuremath{\textit{init}^{\textit{OM}}}}
\newcommand{\StOm}{\ensuremath{\textit{St}^{\textit{OM}}}}
\newcommand{\TOm}{\ensuremath{\textit{T}^{\textit{OM}}}}
\newcommand{\om}{\textit{om}}
\newcommand{\RunOm}{\ensuremath{\textit{Run}^{\textit{OM}}}}
\newcommand{\ScOm}{\ensuremath{\textit{Sc}^{\textit{OM}}}}

%First
\newcommand{\StFirst}{\textit{St}^{\textit{First}}}
\newcommand{\TFirst}{\textit{T}^{\textit{First}}}
\newcommand{\first}{\<first>\xspace}
\newcommand{\First}{\Mec{First}}
\newcommand{\RunFirst}{\textit{Run}^{\textit{First}}}
\newcommand{\ScFirst}{\textit{Sc}^{\textit{First}}}


%Count
\newcommand{\StCount}{\textit{St}^{\textit{Count}}}
\newcommand{\TCount}{\textit{T}^{\textit{Count}}}
\newcommand{\countmech}{\<count>\xspace}
\newcommand{\Count}{\Mec{Count}}
\newcommand{\RunCount}{\textit{Run}^{\textit{Count}}}
\newcommand{\ScCount}{\textit{Sc}^{\textit{Count}}}

%Storage Hookup
\newcommand{\THookup}{\textit{T}^{\textit{Hookup}}}
\newcommand{\Hookup}{\textit{Hookup}\xspace}
\newcommand{\RunHookup}{\textit{Run}^{\textit{Hookup}}}
\newcommand{\ScHookup}{\textit{Sc}^{\textit{Hookup}}}
\newcommand{\SchHookup}{\textit{Sch}^{\textit{Hookup}}}
\newcommand{\UnboundedStorageHookup}{\Mec{Unbounded Storage Hookup}}
\newcommand{\BoundedStorageHookup}{\Mec{Bounded Storage Hookup}}

\newcommand{\unboundedstoragehookup}{unbounded storage hookup\xspace}
\newcommand{\boundedstoragehookup}{bounded storage hookup\xspace}


%Fail/No-Fail Hookup
\newcommand{\FNFHookup}{\Mec{Fail-NoFail Hookup}}
\newcommand{\Verdict}{\textit{verdict}}
\newcommand{\StFNF}{\textit{St}^{\textit{F/NF}}}
\newcommand{\TFNF}{\textit{T}^{\textit{F/NF}}}
\newcommand{\RunFNF}{\textit{Run}^{\textit{F/NF}}}
\newcommand{\ScFNF}{\textit{Sc}^{\textit{F/NF}}}

%The values of F/NF
\newcommand{\fail}{\textbf{F}\xspace}
\newcommand{\nofail}{\textbf{NF}\xspace}

\newcommand{\fnfhookup}{fail-no fail hookup\xspace}


%First + Fail/NoFail Hookup
\newcommand{\FNF}{\Mec{F/NF}}
\newcommand{\FirstFNF}{\textit{First+F/NF}}
\newcommand{\StFirstFNF}{\textit{St}^{\textit{First+F/NF}}}
\newcommand{\TFirstFNF}{\textit{T}^{\textit{First+F/NF}}}
\newcommand{\RunFirstFNF}{\textit{Run}^{\textit{First+F/NF}}}
\newcommand{\ScFirstFNF}{\textit{Sc}^{\textit{First+F/NF}}}

%Queue Info
\newcommand{\QueueInfo}{queue info\xspace}


%%%
\makeatletter
\let\orgdescriptionlabel\descriptionlabel
\renewcommand*{\descriptionlabel}[1]{%
  \let\orglabel\label
  \let\label\@gobble
  \phantomsection
  \edef\@currentlabel{#1}%
  %\edef\@currentlabelname{#1}%
  \let\label\orglabel
  \orgdescriptionlabel{#1}%
}
\makeatother

\newcommand{\PR}[1]{\ensuremath{\textit{#1}}}
\newcommand{\GS}{\PR{GS}\xspace}
\newcommand{\EPOCH}{\<epoch>}
\newcommand{\CURRENT}{\<current>}
\newcommand{\HISTORY}{\<history>}
\newcommand{\THESET}{\<theset>}
\newcommand{\epochs}{Setchain\xspace}
%\newcommand{\Get}{\PR{get}\xspace}
%\newcommand{\Add}{\PR{add}\xspace}
\newcommand{\BRBdeliver}{\<BRB>.\<Deliver>}
\newcommand{\BRBbroadcast}{\<BRB>.\<Broadcast>}
\newcommand{\BABdeliver}{\<BAB>.\<Deliver>}
\newcommand{\BABbroadcast}{\<BAB>.\<Broadcast>}
\newcommand{\setchain}{Setchain\xspace}
\newcommand{\vanilla}{Vanilla\xspace}
\newcommand{\hashchain}{Hashchain\xspace}
\newcommand{\compresschain}{Compresschain\xspace}
\newcommand{\collector}{\emph{collector}\xspace}
\newcommand{\hcollector}{\emph{hash collector}\xspace}
\newcommand{\setchains}{Setchains\xspace}
\newcommand{\SPH}{\texttt{SIGNATURES\_PER\_HASH}\xspace}

\newcommand{\Hspeed}{H1}
\newcommand{\Hbetteralg}{H2}
\newcommand{\Hagr}{H3}
\newcommand{\Hbyzantine}{H4}
\newcommand{\Hdegrade}{H5}

\newcommand{\Dom}{\PR{Dom}}
\newcommand{\Propset}{\PR{propset}}
\newcommand{\Servers}{N}

\newcommand{\byzantineset}{\mathbb{B}}
\newcommand{\correctset}{\mathbb{C}}
\newcommand{\Proposal}{\PR{proposal}}


\newcommand{\AlgTwo}{Alg.~\ref{DPO-alg-basb}\xspace}
\newcommand{\AlgThree}{Alg.~\ref{DPO-alg-basb3}\xspace}
% \newcommand{\AlgTwoSet}{Alg.~\ref{DPO-alg-basb}+set\xspace}
\newcommand{\AlgTwoSet}{Alg.~\ref{DPO-alg-basb}+set\xspace}
%\newcommand{\AlgThreeSet}{Alg.~\ref{DPO-alg-basb3}+set\xspace}
\newcommand{\AlgThreeSet}{Alg.~\ref{DPO-alg3-aggregated}\xspace}

\newcommand{\TXONE}[1]{\ensuremath{\textit{#1}}\xspace}
\newcommand{\TXTWO}[2]{\ensuremath{\textit{#1}\_\textit{#2}}\xspace}
\newcommand{\TXTHREE}[3]{\ensuremath{\textit{#1}\_\textit{#2}\_\textit{#3}}\xspace}
\newcommand{\HaveSubset}{\TXTWO{have}{subset}}
\newcommand{\GenerateInvalidElems}{\TXTHREE{generate}{invalid}{elems}}
\newcommand{\HavocNumber}{\TXTWO{havoc}{number}}
\newcommand{\HavocSubset}{\TXTWO{havoc}{subset}}
\newcommand{\HavocPartition}{\TXTWO{havoc}{partition}}
\newcommand{\HavocElement}{\TXTWO{havoc}{element}}
\newcommand{\Knowledge}{\TXONE{knowledge}}
\newcommand{\ByzantineKnowledge}{\TXTWO{byzantine}{knowledge}}
\newcommand{\ValidElements}{\TXTWO{valid}{elements}}
\newcommand{\EV}{\TXONE{ev}}

\newcommand{\SendBrb}{\TXONE{SendBRB}}
\newcommand{\Tau}{\mathcal{T}}
