\section{Objetivo del capítulo}\label{sec:prelim}

%% Agregar algo de firmas criptográficas.

En este capítulo se presenta una breve descripción de los conceptos principales
necesarios para la comprensión y discusión sobre las
contribuciones de este trabajo.

\section{Sistemas distribuidos: donde todo comenzó}

La computación distribuida nació a fines de la década de 1970 cuando investigadores y profesionales
comenzaron a tener en cuenta las características intrínsecas de los sistemas distribuidos físicamente.
El campo luego emergió como un área de investigación especializada distinta al de redes, al de sistemas
operativos, y al de computación paralela.~\cite{raynal.dist.systems}

La computación distribuida surge cuando se tiene que resolver un problema en términos de entidades
distribuidas (usualmente llamadas procesos, nodos, actores, agentes, pares, etc) en donde cada
entidad tiene conocimiento parcial sobre los diversos parámetros involucrados en el problema que
se tiene que resolver.
Es usual, en este contexto, hacer hincapié en dos tipos distintos fundamentales de procesos:
los \textit{clientes} (o procesos clientes) y los \textit{servidores} (o procesos servidores).
Así, los servidores son aquellos que proveen determinado recurso o servicio, mientras que los
clientes son quienes los demandan.

Es clave notar que el hecho de que las entidades (y la información que consume cada una de ellas)
estén distribuidas no es una propiedad que se encuentre bajo el control de los programadores, sino 
que es una imposición dada por las características propias del problema. En la 
Figura ~\ref{fig:distributed-systems} se expresa esto mediante un punto de vista
de la arquitectura del sistema, en donde cada par $(p_i, e_i)$ denota una entidad informática $p_i$ y su
entrada asociada $e_i$.

\begin{figure}
  \centering
  \includegraphics[scale=0.25]{figures/distributed-systems.png}
  \caption{Estructura básica de la computación distribuida}
  \label{fig:distributed-systems}
\end{figure}


Como se ilustra en la Figura ~\ref{fig:distributed-systems}, un sistema distribuido está hecho de una
colección de unidades de computación (llamadas también unidades informáticas) distribuidas, cada una
abstraída a través de la noción de un proceso, interconectada a través de un medio de comunicación.
Se asume que los procesos cooperan por un objetivo común, lo cual significa que intercambian
información de una manera u otra. Es interesante notar que el hecho de que las entidades
cooperen no significa la asunción de que todas las entidades se comportarán de la forma
acordada, ya sea por un error (por ejemplo, en la programación o en una red) o por un comportamiento
malintencionado.
Se profundizará en este punto en la siguiente sección. % ~\ref{subsubsec:consensus}.

% Two main process failures models are considered in this book:
% • Crash failures. A process commits a crash failure when it prematurely stops its execution. Until
% it crashes (if it ever crashes), a process correctly executes its local algorithm.
% • Byzantine failures. A process commits a Byzantine failure when it does not follow the behav-
% ior assigned to it by its local algorithm. This kind of failure is also called arbitrary failure
% (sometimes known as malicious when the failure is intentional). Let us notice that crash failures
% (which are an unexpected definitive halt) are a proper subset of Byzantine failures.
% A simple example of a Byzantine failure is the the following: while it is assumed to send the
% same value to all processes, a process sends different values to different subsets of processes,
% and no value at all to other processes. This is a typical Byzantine behavior. Moreover, Byzantine
% processes can collude to foil the processes that are not Byzantine.
% From a terminology point of view, let us consider an execution E (an execution is also called a
% run). The processes that commit failures are said to be faulty in E. The other processes are said to
% be correct or non-faulty in E. It is not known in advance if a given process is correct or faulty, this is
% specific to each execution.
% Given a process failure model, the model parameter t is used to denote the maximal number of
% processes that can be faulty in an execution.

% Fuente: Fault-Tolerant Message-Passing Distributed Systems An Algorithmic Approach

\section{Algoritmos de consenso}\label{subsubsec:consensus}
El problema de consenso es uno de los más celebrados en los sistemas de computación distribuida tolerante a fallas.
%
Abstrae diversos problemas en donde, de una manera u otra, los procesos deben ponerse de acuerdo.
%

Los problemas de consenso requieren acuerdo entre un número de procesos para determinar un valor particular.
%
Algunos de los procesos pueden fallar o ser no confiables de diversas maneras, por lo que los protocolos de
consenso deben ser resilientes a fallas.
%
Los procesos deben de alguna forma establecer sus valores candidatos, comunicándose entre ellos, y
poniéndose de acuerdo en un valor específico.
%
Los protocolos que resuelven el problema de consenso son diseñados para lidiar con un número limitado de procesos
que fallan.
%
Estos protocolos deben satisfacer ciertos requerimientos para ser utilizados.
%
A lo largo de la historia se han desarrollado distintos algoritmos de consenso, que funcionan bajo diferentes
escenarios (cantidad de nodos, cantidad de nodos que pueden fallar, formas en las que pueden fallar, etc.)

%

Si bien a lo largo de este trabajo no se entrará en detalles sobre ningún algoritmo de consenso en particular,
es pertienente discutir conceptos y nomenclatura básica.

\subsection{El problema de los generales bizantinos}

Los sistemas de computación confiables deben lidiar con componentes que, por mal funcionamiento,
dan información conflictiva a diferentes partes del sistema.
La situación puede ser expresada abstractamente
en términos de un grupo de generales de la armada bizantina que acampan con sus tropas alrededor de una
ciudad enemiga.
Comunicándose solo a través de mensajeros, los generales deben ponerse de acuerdo en una plan
de batalla común.
Sin embargo, uno o más de ellos pueden ser traidores, quienes tratarán de confundir a los
otros.
El problema es encontrar un algoritmo que asegure que los generales leales lleguen a un acuerdo.
Está demostrado que, usando solo mensajes orales, este problema es resoluble si y solo si más de dos tercios
de los generales son leales~\cite{byzantineproblem}.

%

Con esta analogía de los generales bizantinos se instauraron nuevos conceptos que hoy son ampliamente utilizados.
La habilidad de tolerar máquinas que fallan de formas arbritarias, incluyendo aquellas maliciosas,
es conocida como \emph{tolerancia a fallas bizantinas}.
A su vez, se inició el estudio de los algoritmos de consenso tolerantes a fallas
bizantinas (\textit{Bizantyne fault-tolerant algoritms}), también conocidos como algoritmos de consenso BFT (por sus siglas en inglés).
Si bien la teoría de
BFT es de décadas pasadas, las implementaciones de software se han vuelto populares recientemente,
debido al amplio éxito de tecnologías blockchains como Bitcoin~\cite{nakamoto06bitcoin} y Ethereum~\cite{ethereum}.

\section{Firmas digitales}\label{subsec:signatures}
Usualmente, los protocolos BFT hacen un uso muy fuerte de la criptografía, fundamentalmente para autenticar
mensajes y decisiones que, a su vez, pueden ser utilizadas por otros para validar criptográficamente los resultados
obtenidos gracias al protocolo de consenso.

%
Las firmas digitales son una de las primitivas más poderosas de la criptografía.
Son similares a las firmas de puño y letra, en el sentido de que apuntan a cumplir las mismas propiedades.
Sin embargo, al ser firmas criptográficas, presentan garantías:
\begin{itemize}
  \item La firma solo puede ser utilizada por su dueño para firmar mensajes arbritarios.
  \item Cualquiera puede verificar la firma de un mensaje.
\end{itemize}
 
Los \textit{esquemas de firmas digitales} pertenecen al dominio de lo que se conoce como \textit{criptografía asimétrica}
o \textit{criptografía de clave pública}.
%
Esta categoría de la criptografía hace uso de distintas claves para distintas funciones (en oposición a una única
clave usada en la \textit{criptografía simétrica}), o provee diferentes puntos de vistas a diferentes participantes.
%
En líneas generales esto se puede expresar mediante la existencia de dos claves: una clave pública (conocida por cualquiera)
y una clave privada (que debe mantenerse secreta). Estas claves se encuentras conectadas entre sí y tienen roles complementarios.


Un esquema de firmas típicamente consiste de algoritmos diferentes:
\begin{itemize}
  \item Un algoritmo de generación de pares de claves que un firmante usa para crear una nueva clave privada y una pública.
  \item Un algoritmo para firmar que toma una clave privada y un mensaje, y produce una firma del mismo.
  \item Un algoritmo de verificación de firmas que toma una clave pública, un mensaje, y una firma, y determina si la firma
  fue creada para dicho mensaje con la clave privada asociada a la clave pública. 
\end{itemize}

La clave privada es también conocida como la \textit{clave de firmado} y la clave pública como la \textit{clave de verificación}.
En la Figura ~\ref{fig:digital-signatures} se recapitulan estos tres algoritmos.

\begin{figure}
  \centering
  \subfloat[Generación de claves]{\includegraphics[width = 1.6in]{figures/signature-0000.png}}
  \hspace{1.4em}
  \subfloat[Firma de mensajes]{\includegraphics[width = 1.7in]{figures/signature-1.png}}
  \hspace{1.4em}
  \subfloat[Verificación de firma]{\includegraphics[width = 1.7in]{figures/signature-2.png}}
  \caption{Interfaz de firma digital.}
  \label{fig:digital-signatures}
\end{figure}


Las firmas son utilizadas para autenticar el \textit{origen} de un mensaje (el mensaje fue generado por el dueño de la clave privada
asociada a la firma), así como también la \textit{integridad} del mismo (ya que si el mensaje fuese cambiado, la firma sería invalidada).

%

Existen distintos estándares para los esquemas de firmas digitales, basados en aritmética de módulo sobre grandes números (por ejemplo,
RSA-PSS~\cite{pss.rsa}) o en curvas elípticas (como ECDSA~\cite{ecdsa}).
Sin embargo, no se entrará en detalles sobre ellos.

\section{Los orígenes de la blockchain: Bitcoin}
% Fuente: Mastering Bitcoin. Programming the open blockchain.
Las blockchains tomaron popularidad con la invención de Bitcoin en 2008, luego de la publicación
de un artículo llamado \textit{Bitcoin: A Peer-to-Peer
Electronic Cash System}\footnote{Bitcoin: un sistema de dinero electrónico de par a par.}, escrito
bajo el pseudónimo de Satoshi Nakamoto.
Nakamoto combinó diversas invenciones previas como
\textit{b-money}~\cite{b.money} y \textit{HashCash}~\cite{hashcash} para crear un sistema de dinero electrónico completamente
descentralizado que, justamente, no necesita confiar en ninguna autoridad central para la emisión de monedas
o la ejecución de transacciones. 
La innovación clave en Bitcoin fue el uso de un sistema de cómputo distribuido que lleva
a cabo una elección global de las nuevas transacciones, permitiendo
a la red descentralizada llegar a un consenso sobre el estado de las mismas. Estas transacciones
se guardan empaquetadas como bloques en una bitácora distribuida pública conocida como blockchain.~\cite{mastering.bitcoin}
%
Se dice que estas transacciones se encuentran totalmente ordenadas porque es posible determinar para
dos transacciones cualesquieras en la blockchain cuál de ellas ocurrió antes.

%Esto resolvió de forma elegante el problema del doble gasto,
%en el cual una misma unidad de moneda se utiliza dos veces. Previamente, el problema del doble gasto fue
%una debilidad de los sistemas de moneda digital que fue abordado mediante servidores de información centrales.

%Bitcoin es una colección de conceptos y tecnologías que forman las bases de un ecosistema de dinero
%digital.
%
Si bien la noción de Bitcoin como criptomoneda no es imprescindible para la comprensión de las
contribuciones de este trabajo, es el primer caso de uso que da particular importancia a la blockchain
como tal y, por lo tanto, a su rendimiento. Por este motivo se considera pertinente presentar un breve resumen sobre
su funcionamiento, poniendo foco en el aspecto técnico del mismo (no monetario).

\subsection{Cuentas y transacciones}

Un usuario de Bitcoin está necesariamente lidiando con primitivas criptográficas.
%
Para comenzar, los usuarios de Bitcoin no tienen un nombre de usuario y una contraseña
con la cual inician sesión; en lugar de eso, tienen un par de claves ECDSA (\textit{Eliptic
Curve Digital Signature Algorithm}) que generan ellos mismos mediante la interfaz de firmas
digitales vista en la sección anterior.
%
El balance de un usuario no es más que una cantidad de BTCs\footnote{BTC es el código asociado a la moneda Bitcoin.} asociada a una clave pública,
y por lo tanto, para recibir BTCs, simplemente se debe compartir la clave pública con otros.

%
Para utilizar los BTCs que tiene disponible, el usuario firma una transacción con su clave privada.
%
De forma simplificada, una transacción se podría pensar tan solo como un mensaje (firmado) del tipo
\textquotedblleft Envío X BTCs a la clave pública Y\textquotedblright.

Una característica fundamental de Bitcoin es que no existe una base de datos real que contenga los
balances de cada cuenta.
%
En lugar de eso, los usuarios tienen BTCs disponibles para gastar, llamados
\textit{Unspent Transaction Outputs} (UTXOs).
%
Cada usuario puede gastar únicamente sus propios UTXOs
(puesto que necesita la clave privada asociada para hacerlo).
%
Cuando una transacción gasta algunos de estos UTXOs, desaparecen y aparecen nuevos,
disponibles para gastar para el usuario destino de la transacción.
%

De este modo, para saber cuántos BTCs tiene una cuenta, se deben contar todos los UTXOs que fueron
asignados a dicha clave pública.
%
En otras palabras, se debe contar todo el dinero que fue enviado a dicha clave pública y aún no
fue gastado.

Ahora surgen algunas preguntas. ¿Quién está a cargo de elegir y ordenar las transacciones?
¿Quién mantiene el registro de todas ellas y cómo lo hace?
% Las unidades de moneda llamadas bitcoins son usadas para almacenar y transmitir valor a través
% de los participantes de la red de bitcoin. Los usuarios se comunican entre ellos usando un protocolo 
% específico principalmente mediante internet. La pila del
% protocolo bitcoin, disponible como software de código abierto, se puede correr en una amplia gama de
% dispositivos, incluyendo computadoras portátiles y teléfonos inteligentes, haciéndolo una tecnología
% de fácil acceso.

% Los usuarios pueden transferir bitcoins en la red y hacer básicamente cualquier operación que puede ser
% realizada con monedas tradicionales: comprar y vender bienes, enviar dinero a personas u
% organizaciones, o extender créditos. Los bitcoins también pueden ser vendidos, comprados, o intercambiados por
% otras monedas en casas de cambio especializadas.

% A diferencia de las monedas tradicionales, los bitcoins son completamente virtuales. No hay monedas
% físicas e incluso no existen monedas en sí. Las monedas se representan en transacciones que transfieren
% valor desde el remitente hacia el destinatario. Los usuarios de bitcoin poseen claves criptográficas
% que permiten probar la propiedad de los bitcoins en la red. Con estas claves se pueden firmar las
% transacciones para desbloquear el valor y utilizarlo transfiriéndolo a un nuevo dueño. 
%Las claves
%son a menudo guardadas en billeteras digitales en las computadoras o celulares de los usuarios. La
%posesión de una clave que puede firmar una transacción es el único prerequisito para utilizar bitcoin,
%poniendo el control enteramente en manos de cada usuario.

% Bitcoin es un sistema distribuido de par a par. Como tal no tiene un servidor central o un punto de control.
% Los bitcoins son creados a través de un proceso llamado \textit{minería}, el cual involucra una competencia
% para encontrar solución a un problema matemático mientras se procesan las transacciones de bitcoin.
% Cualquier participante de la red (es decir, cualquiera quien use un dispositivo que corre la pila
% completa de protocolo de bitcoin) puede operar como minero, usando el poder de procesamiento de su
% computadora para verificar y guardar transacciones. Cada 10 minutos, en promedio, alguien logra validar
% las transacciones de los últimos 10 minutos y es compensando con bitcoin nuevo. Esencialmente, la minería
% de bitcoins descentraliza la emisión de moneda y las funciones de compensación de un banco central, evitando
% la necesidad del mismo.

% El protocolo de bitcoin incluye algoritmos nativos que regulan las funciones de minería a lo largo de la
% red. La dificultad de la tarea de procesamiento que los mineros deben realizar se ajusta dinámicamente
% de modo que, en promedio, alguien tiene éxito cada 10 minutos sin importar la cantidad de mineros (y
% de poder de cómputo) existente en un momento dado. El protocolo además disminuye a la mitad el ratio de creación
% de bitcoins cada 4 años, y limita el número total de bitcoins que serán creados a un número fijo de
% aproximadamente 21 millones de bitcoins.

%El resultado es que el número de bitcoins en circulación
%sigue una curva predecible fácilmente que llega a 21 millones para el 2140. Debido a la disminución de
%emisión de bitcoins, a largo plazo, el bitcoin es deflacionario. Incluso más, bitcoin no puede sufrir
%inflación por "imprimir" más dinero más allá de ratio esperado. 

%Detrás de escena, bitcoin es el nombre de un protocolo, una red de pares, y una innovación de
%computación distribuida.
% Bitcoin representa la culminación de décadas de investigación en criptografía
% y sistemas distribuidos, e incluye 4 innovaciones claves funcionando juntas:
% \begin{itemize}
%   \item Una red de par a par descentralizada (el protocolo de bitcoin).
%   \item Una bitácora distribuida pública de transacciones organizadas en bloques (la blockchain).
%   \item Un conjunto de reglas para la validación independiente de transacciones y la emisión de monedas (reglas de consenso).
%   \item Un mecanismo para lograr un consenso global descentralizado en la blockchain (algoritmo de prueba de trabajo).
% \end{itemize}

 \subsection{Prueba de trabajo}

 Para acordar el ordenamiento de las transacciones, Bitcoin permite a cualquiera proponer una lista
 de transacciones a incluirse en la siguiente página de la bitácora.
 %
 Esta propuesta contiene una lista de transacciones llamada \textit{bloque}.
 %
 Sin embargo, dado que la participación en Bitcoin es abierta a cualquiera, la propuesta de nuevos
 bloques debe ser restringida de alguna forma, de modo que solo una persona haga una propuesta
 para el siguiente bloque.
 %

Para lograr esto, el protocolo de Bitcoin establece que para proponer un bloque debe resolverse
un problema lo suficientemente difícil.
%
Este mecanismo se conoce como prueba de trabajo (\textit{POW}, por sus siglas en inglés).
%
En el caso de Bitcoin, el rompecabezas a resolver consisten en que el hash del bloque a proponer
tenga una representación binaria que comience con un número dado de ceros.

%
Las personas que se dedican a resolver el rompecabezas y proponer los nuevos bloques de transacciones
son denominadas \textit{mineros}.

\subsection{La forma de la blockchain}

La blockchain es la bitácora pública de Bitcoin, un registro ordenado, y con fecha y hora de las transacciones.
%
Cada nodo completo en la red de Bitcoin guarda independientemente una blockchain que contiene solo
bloques validados por ese nodo.
%
Cuando cierta cantidad de nodos tienen los mismos bloques en su propia blockchain
se dice que llegaron a un consenso. 
%The validation rules these nodes follow to maintain consensus are
%called consensus rules. This section describes many of the consensus rules used by Bitcoin Core.

La Figura \ref{fig:blockchain} muestra una versión simplificada de una blockchain.
Cada bloque tiene una sección de datos en donde
se agrupan y alojan una o más transacciones nuevas. Copias de cada transacción se hashean, los hashes luego
se emparejan, se hashean, se emparejan de nuevo, y se hashean hasta que queda un único hash: la raíz de un
\textit{merkle tree}~\cite{merkle.tree}. 
En la Figura \ref{fig:merkle-tree} se muestra un ejemplo de este tipo de árboles.
La raíz del merkle tree se aloja en el encabezado del bloque.

\begin{figure}[h]
  \centering
  \includegraphics[scale=0.4]{figures/blockchain.png}
  \caption{Versión simplificada de la blockchain}
  \label{fig:blockchain}
\end{figure}

%
Cada bloque además contiene el hash del encabezado del bloque anterior, encadenándolos.
La blockchain es realmente una sucesión de bloques, donde cada bloque refiere al previo,
hasta llegar al primero, conocido como \textit{génesis}.
Esto asegura que una transacción no pueda modificarse sin modificar el bloque
que la contiene y todos los bloques siguientes~\cite{bitcoin}.

\begin{figure}[h]
  \centering
  \includegraphics[scale=0.4]{figures/merkle-tree.png}
  \caption{Merkle tree}
  \label{fig:merkle-tree}
\end{figure}

%Las transacciones también están encadenadas. Es decir, cada transacción gasta los \textit{satoshis}\footnote{
%Satoshi es la unidad mínima de Bitcoin, llamada así por el pseudónimo de su creador, Satoshi Nakamoto. Un bitcoin
%es equivalente a 100 millones de Satoshis.} previamente recibidos en una o más transacciones previas, por lo
%que la entrada de una transacción es la salida de una transacción previa.

% Fuente: https://developer.bitcoin.org/devguide/block_chain.html

%\subsection{Blockchain actuales}

\section{Tendermint}\label{sec:tendermint}
Tendermint~\cite{Buchman.2018.Tendermint} es un motor de replicación de máquinas de estado que tolera fallas bizantinas. 
%
Fue uno de los primeros sistemas en adaptar protocolos de consenso clásicos tolerantes a fallas bizantinas
al paradigma de blockchain.
%
Es decir, donde el consenso se ejecuta sobre lotes de transacciones vinculados mediante
hashes criptográficos (usualmente llamados bloques) en una red pública y abierta.

Tendermint funciona como una capa intermedia de blockchain que soporta la replicación de aplicaciones arbitrarias,
escritas en cualquier lenguaje de programación~\cite{tendermint.design}.
%
En la Figura~\ref{fig:replication} se muestra un esquema general de la arquitectura de la replicación de máquinas de
estado.
%
Las transacciones se reciben desde el cliente, quien se comunica, mediante una API, con una (o más) de las máquinas
en la red.
%
Estas transacciones pasan a través del protocolo de consenso, quien es el encargado de ordernar dichas
transacciones en bloques, mediante una serie de pasos que involucran comunicación entre todas las máquinas
(señalizada en la figura con líneas punteadas).
%
Una vez ordenadas, las transacciones son ejecutadas en cada una de las máquinas, modificando el estado
de cada una de ellas.
%
De esta forma, se replican lotes de transacciones enviadas por los distintos clientes, que son ejecutadas en el mismo
orden en cada una de las máquinas en la red, resultando en un mismo estado a lo largo de múltiples máquinas.
%
Las figuras azules representan máquinas.
%
Las líneas punteadas representan la comunicación entre máquinas para llevar a cabo el protocolo de consenso para ordernar
transacciones.

Como motor de replicación de máquinas de estado, Tendermint es un sistema distribuido compuesto por un conjunto
dinámico de procesos o nodos.
%
Estos nodos varian con el tiempo, a medida que se unen o abandonan la red subyacente del sistema.
%
Un subconjunto de estos nodos son quienes ejecutan el algoritmo de consenso de Tendermint.
%
A estos nodos se los denomina \textit{validadores}.
%
Los nodos que no ejecutan el algoritmo de consenso de Tendermint pueden, sin embargo, participar
activamente en otros módulos del sistema.


\begin{figure}
  \centering
  \includegraphics[scale=0.3]{figures/state-machine-replication.png}
  \caption{Esquema simplificado de la replicación de máquinas de estado.}
  \label{fig:replication}
\end{figure}

\subsection{Arquitectura de capas de Tendermint}\label{subsec:layers}
% %
Tendermint consiste de dos componentes técnicos principales: un motor de consenso de blockchain, y una interfaz
de aplicación genérica.
%
El motor de consenso, llamado \textit{Tendermint Core}, asegura que las mismas transacciones sean registradas en cada
máquina en el mismo orden.
%
La interfaz de aplicación, llamada \textit{interfaz de aplicación de blockchain} (ABCI)\footnote{Proveniente de \textit{Application
BlockChain Interface} (ABCI) en inglés.}, habilita que las transacciones
sean procesadas en cualquier lenguaje de programación.

\begin{figure}
  \centering
  \includegraphics[scale=0.3]{figures/tendermint-layers.png}
  \caption{Arquitectura de capas de Tendermint.}
  \label{fig:tendermint-architecture}
\end{figure}

La Figura~\ref{fig:tendermint-architecture} presenta la estructura básica de Tendermint como un sistema de capas.
%
En este sistema de capas se incluyen los componentes fundamentales mencionados en el párrafo anterior.
%
La capa inferior se encarga de la comunicación entre pares, y provee comunicación para los principales
módulos de la blockchain.
%
Éstos son los módulos que se presentan en la capa inmediata superior, el
Tendermint Core: mempool, consenso, evidencia, sincronización rápida, y sincronización de estado.
%
Para el presente trabajo, de la capa Tendermint Core, solo serán pertinentes los módulos de mempool y de consenso
%
En la capa inmediata superior se encuentra la ya mencionada interfaz de aplicación, que es quien se encarga de comunicarse
con la aplicación en sí.
%
La aplicación reprsenta la capa superior de este sistema.
%
Como se puede observar en la figura, el cliente se comunica con la capa de Tendermint Core, por medio de los
módulos de mempool o consenso. 

Los lápices presentes tanto en el cliente como en la capa de aplicación indican que esos son los módulos creados
por la programadora.
%
Tendermint funciona como una caja negra que resuelve consenso para la creación de aplicaciones arbritarias.
%

\subsubsection{Tendermint Core}

La capa Tendermint Core tiene diversos módulos, de los cuales se presentarán brevemente dos: mempool y consenso.
%
El módulo de mempool es un punto de entrada de Tendermint.
%
Recibe, valida, almacena y difunde transacciones enviadas por los clientes. 
%
%Estas transacciones, en caso de ser consideradas válidas por la aplicación, son eventualmente incluidas en bloques
%por el protocolo de consenso.
%
Los nodos exponen una interfaz para recibir dichas transacciones de los clientes, mediante llamadas
RPC (\textit{Remote Procedure Call})\footnote{Las llamadas a procedimiento remoto son una forma de
interacción entre clientes y servidores.}.
%
%Remote procedure call is the synchronous language-level transfer of control between programs in
%disjoint address spaces whose primary communication medium is a narrow channel. The thesis of
%this dissertation is that remote procedure call (RPC) is a satisfactory and efficient programming
%language primitive for constructing distributed systems
%
%A client interacts with the system via RPC by
%submitting transactions that are added to the mempool module,
%and receiving a response generated by the consensus module.
%The consensus module is in charge of ordering and intermediating the execution of transactions, by means of the application
%interface. 
%
Por su parte, el módulo de consenso se encarga de ordenar e intermediar la ejecución de las transacciones, por medio de la
interfaz de aplicación. Es responsable de decidir el siguiente bloque de transacciones a añadir a la blockchain.

En la Figura~\ref{fig:tendermint-architecture} se ilustra mediante flechas la comunicación
entre el cliente y el Tendermint Core.
Simplificadamente, el cliente interactúa con el sistema mediante RPC al enviar transacciones
que (en caso de ser válidas) son añadidas al módulo de mempool,
y en general recibe respuestas generadas por el módulo de consenso.
Las transacciones enviadas por clientes que llegan al módulo de mempool se difunden a todos los nodos, y eventualmente son recibidas
por el nodo validador responsable de proponer el siguiente bloque como parte del algoritmo de consenso.
El módulo de consenso de este nodo es el responsable de recuperar la lista de transacciones pendientes
para construir el bloque que será propuesto como siguiente. 


\subsubsection{Application BlockChain Interface}\label{subsec:abci}
La interfaz de aplicación de blockchain es la interfaz entre el Tendermint Core y la aplicación replicada.
%
Un nodo de Tendermint mantiene tres conexiones ABCI con la aplicación replicada.
%

La \textit{conexión de consenso} se utiliza solo cuando se hace \textit{commit} sobre un nuevo bloque,
y comunica toda la información del bloque mediante una serie de peticiones: 
\BeginBlock, [\DeliverTx, ...], \EndBlock, \Commit.
%
Esto significa que, cuando un bloque es acordado mediante consenso, Tendermint envía una petición \BeginBlock
seguida por una lista de peticiones \DeliverTx (una por cada transacción en el bloque), que finalizan con las
peticiones \EndBlock y \Commit, en ese orden.

%

La \textit{conexión de mempool} es usada por el protocolo de reserva de transacciones \footnote{\textit{Transaction pool protocol}
en inglés.} para validar las transacciones enviadas por los clientes en relación al estado de la aplicación.
%
Esta conexión solo admite peticiones \CheckTx. Las transacciones se chequean (mediante \CheckTx) en el mismo orden
en que fueron recibidas por el nodo validador.
%\footnote{Los nodos validadores son responsables de hacer commit sobre nuevos
%bloques en la blockchain. Estos validadores participan en el protocolo de consenso mediante la difusión de votos, los
%cuales contienen firmas criptográficas generadas por las claves privadas de cada validador.}.
%
Si \CheckTx retorna \texttt{OK}, la transacción se mantiene en memoria y se retransmite a otros pares en el mismo orden
en que fue recibida.
%
En caso contrario, se descarta.
%
Es parte de la lógica de la aplicación definir cuándo una transacción es válida o no, e incluso la validación es opcional.
%
De esta manera, queda claro que Tendermint es agnóstico de la semántica de las transacciones.

%

La \textit{conexión de consulta} permite recuperar información de la instancia local de la aplicación, usada
por distintos módulos de Tendermint (tales como el módulo de filtrado de pares).
%
Se utiliza para consultar la aplicación sin forzar consenso.
%
Se expone mediante el servidor RPC del Tendermint Core, de modo que los clientes pueden
consultar el estado de la aplicación sin exponer un servidor de la aplicación en sí mismo. 

La Figura~\ref{fig:abci_flow} muestra el flujo de mensajes mediante las conexiones de consenso y de mempool.
%
Las flechas indican los sentidos de la comunicación.
%
La interacción entre la mempool y la aplicación se da mediante la conexión de mempool de la ABCI.
%
Cuando una transacción llega a la mempool, debe chequearse su validez contra el estado de la aplicación.
%
Esto se realiza mediante la petición \CheckTx de la ABCI.
%
La respuesta a dicha petición es generada por la lógica de la aplicación, quien se encarga de determinar la
validez de una transacción.
%
Por otro lado, la comunicación entre el módulo de consenso y la aplicación se da mediante la conexión de consenso.
%
Cuando a un nodo validador le toca el turno de ser proponente del próximo bloque a añadir en la blockchain,
consulta a la mempool para recuperar la lista de transacciones que se utilizarán para la construcción dicho bloque.
%
Una vez que la lógica de consenso resuelve cuál es el nuevo bloque, se comunica con la aplicación para
dar a conocerlo.
%
Esto se hace mediante una secuencia de peticiones: \BeginBlock, [\DeliverTx, ...], \EndBlock y \Commit.
%
La lógica de la aplicación es la encargada de actualizar su estado adecuadamente en función de las transacciones del
nuevo bloque y de generar las respuestas a dichas peticiones.


\begin{figure}
  \centering
  \includegraphics[scale=0.35]{figures/abci_msg_flow.png}
  \caption{Flujo de mensajes mediante la ABCI.}
  \label{fig:abci_flow}
\end{figure}

\section{Modelo de computación}\label{sec:model_system}
Habiendo presentado los conceptos preliminares básicos para este trabajo, se describe
el modelo de computación que utilizaremos de aquí en adelante.

El modelo involucra un sistema distribuido consistente de procesos (clientes y servidores) con una
red de comunicación subyacente en donde cada proceso puede comunicarse con cualquiera de los otros.
Los clientes representan a aquellos procesos cuyo objetivo es inyectar u obtener elementos en la \setchain,
mientras que los servidores representan a los procesos que se encargan de implementar la \setchain
como un tipo de datos distribuido con ciertas propiedades.

La comunicación entre procesos se realiza mediante el paso de mensajes.
Cada proceso computa independientemente y a su propia velocidad, y los aspectos internos de cada proceso
permanecen ocultos para el resto. Los retrasos en la transferencia de mensajes son aribitrarios pero finitos
y también permanecen desconocidos para los procesos.

Los procesos pueden fallar arbritariamente, pero el número de servidores que fallan (bizantinos) está
limitado por \textit{f}, y el número de servidores total, \textit{n}, es al menos $3f + 1$.
Se asumen \textit{canales confiables} entre procesos no bizantinos (correctos), por lo tanto ningún
mensaje se pierde, se duplica o se modifica.
Cada proceso (cliente o servidor) tiene un par de claves públicas y privadas.
Las claves públicas fueron disribuidas confiablemente entre todos los procesos que pueden interactuar
con otros.
Por lo tanto, se descarta la posibilidad de procesos espurios.
Se asume que los mensajes son autenticados, de modo que mensajes corruptos o fabricados por
procesos bizantinos son detectados y descartados por los procesos correctos.
Como resultado, la comunicación entre procesos corrrectos es confiable pero asíncrona.

%%% Local Variables:
%%% TeX-master: "article.tex"
%%% TeX-PDF-mode: t
%%% End:
