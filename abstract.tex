  \chapter*{Resumen}
  %
  %
  % What is the problem. Justifiy that the problem is a problem.
  %
  Un aspecto clave en la adopción de tecnologías \textit{blockchain} es el rendimiento,
  usualmente medido en términos de número de bloques por segundo.
  %
  En consecuencia, diversas técnicas que buscan mejorar la escalabilidad de las blockchains
  están siendo estudiadas y desarrolladas.
  %
  Las blockchains actuales requieren algoritmos de consenso que garanticen que las
  transacciones, empaquetadas en bloques, estén totalmente ordenadas.
  %
  Dado que esta imposición de un orden total puede ser innecesaria para algunas aplicaciones,
  un enfoque prometedor para mejorar la escalabilidad se basa en relajar esta exigencia.
  %

  Bajo esta idea subyacente nace \textit{Setchain}, un tipo de
  datos concurrente que implementa conjuntos de solo crecimiento distribuídos 
  (\textit{distributed grown-only sets}). A su vez,
  es tolerante a comportamientos bizantinos, y provee barreras o puntos de sincronización
  (llamados épocas), que dotan a la estructura de un orden parcial.
  %
  \setchain relaja el requerimiento de orden total y, de esta forma, busca lograr mayor
  rendimiento y escalabilidad.
  %
  Las \setchains se pueden usar para aquellas aplicaciones, como los registros digitales,
  en donde los elementos en la blockchain no necesitan estar ordenados, excepto a través
  de barreras ocasionales.
  
  %
  
  Distintos algoritmos bizantinos distribuídos que implementan \setchain fueron
  propuestos, pero no existía al momento ninguna implementacion eficiente compatible con
  los requerimientos de una aplicación del mundo real.
  
  %
  % What is the solution.
  %
  En el presente trabajo se propone una familia de implementaciones de \setchain 
  \textit{de mundo real} construídas sobre \textit{Tendermint}, una plataforma de
  aplicación de blockchain.  
  % 
  La mayor contribución es \textit{Haschain}, la cual explota el poder de compresión de las
  funciones hash para reducir la comunicación necesaria durante difusión y consenso,
  comunicando un hash de tamaño fijo en lugar de cientos o miles de elementos.
  %
  El precio a pagar es una algoritmo distribuído adicional para obtener el conjunto de
  elementos desde un hash. Esto se realiza garantizando tolerancia a servidores bizantinos.
  %
  En este trabajo se discute una implementación práctica de \hashchain construída enteramente
  sobre Tendermint, una plataforma madura usada en distintos proyectos de blockchain, como
  Cosmos o Tezos.
  % Ver qué se dice sobre las pruebas mas o menos formales que existen

%%% Local Variables:
%%% TeX-master: "article.tex"
%%% TeX-PDF-mode: t
%%% End:
