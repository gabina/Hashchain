  %\chapter*{Resumen}
  %
  %
  % What is the problem. Justifiy that the problem is a problem.
  %
  Un aspecto clave en la adopción de tecnologías \textit{blockchain} es el rendimiento,
  usualmente medido en términos de número de transacciones por segundo.
  %
  En consecuencia, diversas técnicas que buscan mejorar el rendimiento de las blockchains
  están siendo estudiadas y desarrolladas.
  %
  Las blockchains actuales requieren algoritmos de consenso que garanticen que las
  transacciones, empaquetadas en bloques, estén totalmente ordenadas.
  %
  Dado que esta imposición de un orden total puede ser innecesaria para algunas aplicaciones,
  un enfoque prometedor para mejorar el rendimiento se basa en relajar esta exigencia.
  %

  Bajo esta idea subyacente nace \textit{Setchain}~\cite{Capretto.2022.Setchain}, una estructura de
  datos concurrente que implementa conjuntos distribuidos que solo crecen
  (\textit{distributed grown-only sets}).
  %
  \setchain provee barreras o puntos de sincronización,
  llamados épocas (\textit{epochs} en inglés), que dotan a la estructura de un orden parcial.
  Esto implica que las épocas en sí presentan un orden entre ellas,
  pero que entre los elementos pertenecientes a una misma época no
  se puede establecer un orden.
  %
  De esta forma, relaja el requerimiento de orden total buscando lograr mayor
  rendimiento y escalabilidad.
  
  % Otro aspecto importante de las \setchains es que la validez de los elementos de cada
  % época se puede establecer independientemente de los otros elementos.

  %
  Las \setchains se pueden usar para aquellas aplicaciones, como los registros digitales,
  en donde los elementos en la blockchain no necesitan estar ordenados, excepto a través
  de barreras ocasionales.
  
  % Setchain en sí no es tolerante a comportamientos bizantinos. Sus implementaciones lo son.
  %A su vez, \setchain es tolerante a comportamientos bizantinos. Esto significa que
  %tiene la capacidad de llegar a un consenso sobre los elementos y las épocas,
  %aún si un cierto número de entidades participantes del sistema
  %distribuido se comporten de manera arbitraria (distinta a la establecida por el protocolo
  %de \setchain).
  
  Distintos algoritmos distribuidos tolerantes a comportamientos bizantinos que
  implementan \setchain fueron propuestos, pero no existía al momento ninguna
  implementación eficiente compatible con
  los requerimientos de una \textit{aplicación del mundo real}.
  
  %
  % What is the solution.
  %
  En el presente trabajo se propone una familia de implementaciones de \setchain 
  de mundo real y tolerantes a fallas bizantinas construidas sobre \textit{Tendermint}~\cite{Buchman.2018.Tendermint}.
  Tendermint es una plataforma madura de aplicación de blockchain usada en distintos
  proyectos de blockchain, como Cosmos~\cite{cosmos} o Tezos~\cite{goodman14tezos}. 
  % 
  
  La familia de implementaciones que se presenta en este trabajo sigue un enfoque
  incremental, proveyendo diversas aproximaciones a la solución final y más compleja;
  consta de tres variantes.
  Se comienza por la solución más básica de \setchain, llamada \vanilla.
  \compresschain es una variante intermedia utilizando un algoritmo de compresión.
  La mayor contribución es \hashchain, la cual explota el poder de compresión de las
  funciones hash para reducir la comunicación necesaria durante difusión y consenso,
  comunicando un hash de tamaño fijo en lugar de cientos o miles de elementos.
  %
  El precio a pagar es un algoritmo distribuido adicional para obtener el conjunto de
  elementos desde un hash.
  %

  % En este trabajo se discute una implementación práctica de \hashchain construida enteramente
  % sobre Tendermint, una plataforma madura
  % Ver qué se dice sobre las pruebas mas o menos formales que existen

  \section{Marco del trabajo}
  Este documento fue elaborado a partir del trabajo realizado en la pasantía de investigación
  en el Instituto Madrileño de Estudios Avanzados de Software (IMDEA Software) durante el período
  enero a julio de 2023.
  La tesina se desarrolló dentro del grupo de investigación de
  \emph{síntesis reactiva y verificación en tiempo de ejecución} que allí se desempeña.

  \section{Organización del trabajo}
  La tesina se organiza en capítulos, cada uno de ellos contando con una pequeña sección inicial
  describiendo los objetivos del mismo, excepto aquellos en los cuales no se consideró necesario por
  ser de extensión muy breve.
  %
  Se estructura según se describe a continuación.
  %
  En el capítulo \ref{chapter:introduction} se
  contextualiza el problema sobre el cual se trabaja, detallándolo y presentando brevemente
  el estado del arte.
  %
  En el capítulo \ref{chapter:prelimnaries} se presentan los conceptos de las ciencias de la computación que se
  consideran pertinentes para el análisis y comprensión de las contribuciones de este trabajo.
  %
  El capítulo \ref{chapter:setchain} ofrece una descripción detallada de la estructura de datos \setchain,
  comentando su interfaz y las propiedades que satisface.
  %
  El capítulo \ref{chapter:sol} aborda las tres implementaciones de \setchain, contribuciones originales
  de este trabajo.
  %
  En el capítulo \ref{chapter:proofs} se demuestra (mediante pruebas escritas en lenguaje natural) que
  las tres implementaciones propuestas cumplen las propiedades de \setchain.
  %
  El capítulo \ref{chapter:evaluation} detalla cómo fue llevada a cabo la evaluación empírica de las
  nuevas implementaciones y se presentan los resultados obtenidos.
  %
  En el capítulo \ref{chapter:conclusions} se discuten las conclusiones de este trabajo y 
  se comentan posibles pasos futuros que se derivan de esta tesina.
  %
  Finalmente, el capítulo \ref{chapter:glossary} consta de un breve glosario con algunos conceptos
  utilizados en inglés a lo largo del trabajo, ya que no cuentan con traducciones adecuadas al español.
  Estos términos son escritos en su mayor parte \textit{en cursiva}, de modo que se sugiere dirigirse
  al glosario cuando un término escrito con dicho estilo tipográfico genere dudas.


%%% Local Variables:
%%% TeX-master: "article.tex"
%%% TeX-PDF-mode: t
%%% End:
