\section{Objetivo del capítulo}
En este capítulo se exponen las pruebas de correctitud de los algoritmos presentados para
\vanilla, \compresschain, y \hashchain en el capítulo~\ref{chapter:sol}.
Es decir, se prueba que dichos algoritmos satisfacen las propiedades deseadas para una
implementación de \setchain introducidas en~\ref{sec:setchain-properties}.

\section{Propiedades de Tendermint}\label{sec:properties}
Dado que todas las soluciones presentadas en este trabajo son construidas sobre Tendermint,
para probar la correctitud de ellas, será necesario recapitular las propiedades que la
plataforma Tendermint garantiza.

\subsection{Modelo de computación}
El modelo de Tendermint considera un sistema de procesos que se comunican mediante el intercambio de
mensajes.
Los procesos pueden ser correctos o defectuosos, donde un proceso defectuoso se comporta de maneras
arbitrarias. Es decir, se consideran procesos bizantinos.

Se asume que cada proceso tiene una cantidad determinada de \textit{poder de voto} (el poder de
voto de un proceso puede ser 0).
%
En el contexto de Tendermint, denotamos con \textit{n} el poder de voto total de los validadores
en el sistema, y asumimos una cota superior \textit{f} en el poder de voto total proveniente de
los validadores bizantinos.
%
El algoritmo de consenso de Tendermint asume que $n > 3f$, es decir, requiere que el poder de voto
en manos de validadores bizantinos sea menor a 1/3 del poder de voto total.
% We consider a system of processes that communicate by exchanging messages. Processes can be correct or
% faulty, where a faulty process can behave in an arbitrary way, i.e., we consider Byzantine faults. We assume
% that each process has some amount of voting power (voting power of a process can be 0). Processes in our
% model are not part of a single administrative domain; therefore we cannot enforce a direct network connectivity
% between all processes. Instead, we assume that each process is connected to a subset of processes called peers,
% such that there is an indirect communication channel between all correct processes. Communication between
% processes is established using a gossip protocol

Para lograr homogeneidad con el modelo de computación presentado en ~\ref{sec:model_system},
consideramos que el poder de voto de cada validador es igual a 1, y por lo tanto existen exactamente
$n$ procesos en total y $f$ procesos bizantinos.
 
\subsection{Propiedades nativas}
Se presentan dos propiedades fundamentales de Tendermint, debidamente demostradas en \cite{Buchman.2018.Tendermint}.

\newcounter{prop:tendermint}
\setcounter{prop:tendermint}{\value{property}}

\begin{property}[Tendermint-Agreement]\label{tendermint:agreement}
  Dos procesos correctos nunca deciden valores distintos.
\end{property}

\begin{property}[Tendermint-Termination]\label{tendermint:termination}
    Todos los procesos correctos deciden eventualmente un valor.
\end{property}

A continuación se presenta una propiedad que se desprende de las propiedades anteriores
y de la concepción de Tendermint
como máquina de replicación de estados.

\begin{property}[Tendermint-Global-Requests]\label{tendermint:global-delivery}
  Si un sevidor correcto recibe una cadena de peticiones
  \BeginBlock, [\DeliverTx, ...], \EndBlock, \Commit,
  entonces todos los servidores correctos reciben dichas peticiones en el mismo orden.
  Aún más, esta serie de peticiones arriban a todos los servidores correctos en
  el mismo orden con respecto a cualquier otra cadena de peticiones.
  % Si un servidor correcto $v$ recibe una petición $\<DeliverTx>(t)$, entonces todos
  % los servidores correctos reciben la misma petición $\<DeliverTx>(t)$ eventualmente.
  % Aún más, esta petición se hace en el mismo orden con respecto a otras
  % peticiones $\<DeliverTx>$ en todos los servidores correctos.
\end{property}

% \begin{proof}
%   Si un servidor correcto $v$ recibe una petición $\<DeliverTx>(t)$ entonces, por definición
%   de \<DeliverTx>, $t$ fue decidida como una transacción parte de un bloque.
%   Por las propiedades \emph{Tendermint-Agreement}~\ref{tendermint:agreement} y
%   \emph{Tendermint-Termination}~\ref{tendermint:termination},
%   podemos asegurar que todos los servidores correctos deciden a $t$ como
%   parte del mismo bloque y, por lo tanto, todos recibirán la petición \<DeliverTx>($t$),
%   incluso 
%   en el mismo orden respecto a las otras peticiones \<DeliverTx> para el mismo bloque
%   - ya que estas peticiones se
%   realizan de forma asíncrona pero en orden, una vez que las transacciones dentro del bloque
%   ya fueron decididas y ordenadas.
% \end{proof}


\subsection{Propiedades adicionales}
En el contexto de estas demostraciones
se asume una mempool de tamaño tal que nunca se satura.
Es decir, el ratio de invocaciones \<add> por parte de los clientes nunca
llena la capacidad de la mempool, de modo que los elementos enviados por clientes
nunca son descartados por mempool llena.
De aquí se desprenden las siguientes propiedades.

\begin{property}[Tendermint-Eventual-CheckTx]\label{tendermint:eventual-checktx}
  Sea $t$ una transacción.
  Si se invoca \<broadcastTx>($t$) sobre un servidor correcto $v$, luego
  $v$ recibirá la petición \<CheckTx>($t$).
\end{property}

\begin{property}[Tendermint-Eventual-Injection]\label{tendermint:eventual-injection}
  Sea $t$ una transacción en la mempool de un servidor correcto.
  Eventualmente, $t$ formará parte de un bloque.
\end{property}

% No la necesito si add-before-get se cumple trivialmente
% \begin{property}[Tendermint-DeliverTx-BroadcastTx]\label{tendermint:delivery-broadcast}
%   Sea $v$ un servidor correcto.
%   Si $v$ recibe una petición $\<DeliverTx>(t)$,
%   entonces necesariamente, en el pasado, 
%   \<broadcastTx>($t$) fue invocado en algún servidor.
%   % NOTA: acá no se puede asegurar que fue invocado en un servidor correcto,
%   % ni tampoco que la transacción es válida.
% \end{property}

% \begin{property}[Tendermint-Delivery-Mempool]\label{tendermint:delivery-mempool}
%   Sea $v$ un servidor correcto.
%   Si $v$ recibe una petición $\<DeliverTx>(t)$,
%   entonces necesariamente, en el pasado, $t$ formó parte de la mempool de un
%   servidor correcto y fue propuesto para formar parte de un bloque.
% \end{property}

% \begin{property}[Tendermint-Mempool-Broadcast]\label{tendermint:mempool-broadcast}
%   Sea $v$ un servidor correcto.
%   Si $t$ forma parte de la mempool de $v$, entonces necesariamente, en el pasado,
%   \<broadcastTx>($t$) fue invocado en algún servidor correcto y la petición
%   $\<CheckTx>(t)$ en $v$ retornó \texttt{True}.
% \end{property}

\section{Pruebas de correctitud}
\subsection{Consideraciones generales}\label{subsection:proof-comments}
Muchas propiedades de \setchain razonan sobre el resultado $(S, H, h)$ de una invocación
a $v.\<get>$ sobre un servidor $v$.
%
Siendo que en todas las implementaciones \<get> retorna los valores actuales de $v.\THESET$, $v.\HISTORY$ y $v.\EPOCH$,
en las demostraciones se razonará directamente sobre la construcción de ellos, omitiendo la invocación
a \<get>.
%
Esto requiere un tratamiento especial para \hashchain, siendo que allí la definición de \<Query>~\ref{alg:hash_query}
(asociada a \<get>) no solo se limita a retornar la terna mencionada sino que además inicia la construcción
de $\HISTORY$ y difunde las pruebas de época, a través de la llamada a \<updateHistory>~\ref{alg:hash_update_history}.
%
Sin embargo, como fue mencionado en la sección \ref{subsec:hashchain-algorithms} ambos procesos podrían
ser ejecutados periódicamente mediante planificación, independientemente de la petición de consultas.

A su vez, para todo servidor correcto $v$, $v.\HISTORY$ y $v.\THESET$ son estructuras que solo crecen, por lo que una
vez que un elemento fue añadido a ellas, nunca será removido.

Por otro lado, como se presentó en el modelo de computación definido en la sección \ref{sec:model_system},
un nodo bizantino no puede generar un elemento válido.
%
Es decir, los elementos válidos son solo creados por clientes (no por procesos bizantinos).
%
Además, suponemos que los elementos solo pueden ser añadidos mediante una llamada a \<add>.
%
Por lo tanto, la propiedad \emph{Add-before-get} \ref{api:get->add} de \setchain se cumple
trivialmente para todas las soluciones presentadas en este trabajo.
%
Notemos que consideramos de la misma manera a los elementos de cliente como a los elementos
de prueba de época. Es decir, incluso un elemento de prueba de época $e'$ fue añadido mediante una llamada
a $w.\<add>(e')$ en algún servidor correcto $w$, donde el \textit{cliente} que invocó tal llamada fue el
mismo proceso que corre la ABCI correspondiente al servidor $w$.

\subsection{\vanilla}\label{subsec:proof-vanilla}

\newcounter{lemma:vanilla}
\setcounter{lemma:vanilla}{\value{lemma}}

\begin{lemma}\label{lemma:vanilla-consistent-set}
  La implementación \vanilla cumple la propiedad ~\ref{api:consistent-set} \textit{Consistent Sets} de \setchain,
  que establece que las épocas solo contienen elementos que provienen del conjunto de solo crecimiento.
\end{lemma}

\begin{proof}
  Esto es trivialmente correcto debido a que la
  construcción de $\THESET$ y de $\HISTORY$ se hace a partir de los mismos elementos, como se puede ver en las líneas
  \ref{line:abci-vanilla-set} y \ref{line:abci-vanilla-history} en el Algoritmo~\ref{alg:abci-vanilla}.
\end{proof}

\begin{lemma}
  La implementación \vanilla cumple la propiedad ~\ref{api:history->theset-local} \textit{Add-Get-Local} de \setchain,
  que declara que todo elemento válido añadido a un
  servidor correcto $v$ es eventualmente retornado en todas las llamadas futuras a $v$.\<get>.
\end{lemma}

\begin{proof}
  Sea $e \in $ \(V\) y $v$ un servidor correcto, para los cuales un cliente invoca $v.\<add>(e)$.
  Como $e \in $ \(V\), sabemos que $e$ es válido ($\bigstar$).
  Si $e \in v.\HISTORY$, entonces, por la propiedad anterior, $e \in v.\THESET$ y la demostración es trivial.
  Suponemos entonces $e \notin v.\HISTORY$ ($\clubsuit$).
  Por definción de \<add> en el algoritmo \ref{alg:api-vanilla}, la invocación
  $v.\<add>(e)$ se traduce en la llamada \<broadcastTx>($t$) sobre el servidor $v$,
  donde $t = e$. Por la propiedad~\ref{tendermint:eventual-checktx} \emph{Tendermint-Eventual-CheckTx},
  $v$ recibirá eventualmente la petición \<CheckTx>($t$).
  Por $\bigstar$ se tiene que $e$ es un elemento válido y por $\clubsuit$, que $e \notin \HISTORY$,
  por lo cual, siendo $t = e$
  y considerando la definición de \<CheckTx> en la línea \ref{alg:van_check_tx} del algoritmo
  \ref{alg:abci-vanilla}, se deduce que la llamada a \<CheckTx>($t$) retornará \texttt{True}
  en $v$, por lo cual $t$ pasará a formar parte de la mempool de $v$.
  Luego, por la propiedad~\ref{tendermint:eventual-injection} \emph{Tendermint-Eventual-Injection} sabemos
  que la transacción $t$ formará parte de un bloque.
  Por definición de \<DeliverTx>, dado que $t$ es una transacción parte de un bloque,
  el servidor $v$ eventualmente recibirá una petición \<DeliverTx>($t$),
  en donde $t = e$.
  Como se puede observar
  en la condición \texttt{if} de la línea~\ref{line:abci-vanilla-if} y en la definición de
  \<isValidTransaction>~\ref{alg:van_is_valid_tx}
  en el Algoritmo~\ref{alg:abci-vanilla},
  se analizan dos aspectos: que $e$ sea válido
  y que $e$ no forme parte de $v.\HISTORY$.
  Ambas condiciones se cumplen por $\bigstar$ y $\clubsuit$, respectivamente.
  Por lo tanto, $e$ será añadido a $v.\THESET$ como se indica en la
  línea ~\ref{line:abci-vanilla-set}, y retornado como parte de él en todas las futuras invocaciones a
  $v$.\<get>.
\end{proof}

\begin{lemma}\label{proof:get-global}
  La implementación \vanilla cumple la propiedad~\ref{api:history->theset} \textit{Get-Global} de \setchain,
  que establece que los elementos presentes en un servidor correcto
  son propagados a todos los servidores correctos.
\end{lemma}

\begin{proof}
  Sean $v$ y $w$ dos servidores correctos y $e \in $ \(U\), tal que  $e \in v.\THESET$.
  Debemos probar que eventualmente se cumplirá $e \in w.\THESET$.
  Si $e \in v.\THESET$, entonces, por construcción de $\THESET$, necesariamente $v$ en algún momento recibió una
  petición \<DeliverTx>($t$), donde $t = e$, y $e$ es válido.
  Si un servidor correcto recibe una petición \<DeliverTx>($t$), necesariamente lo hace como parte de una cadena de
  peticiones \BeginBlock, [\DeliverTx, ...], \EndBlock, \Commit.
  Luego, por la propiedad~\ref{tendermint:global-delivery}
  \emph{Tendermint-Global-Requests} todos los
  servidores correctos, y en particular $w$, reciben la misma serie de peticiones, incluyendo la petición \<DeliverTx>($t$).
  Dado que $e$ es válido, se tienen dos opciones: o bien $e$ es parte de $w.\HISTORY$,
  en cuyo caso ya se cumple $e \in w.\THESET$ (ver lema \ref{lemma:vanilla-consistent-set}), o bien $e$ es un elemento
  nuevo ($e \notin w.\HISTORY$).
  En este último caso,
  $e$ será añadido a
  $w.\THESET$, como se muestra en la línea~\ref{line:abci-vanilla-set} en el Algoritmo~\ref{alg:abci-vanilla}.
  Por lo tanto, eventualmente se cumplirá $e \in w.\THESET$.

  % Si un servidor correcto recibe una petición \<DeliverTx>($t$), por la propiedad
  % \textbf{Tendermint-Global-Requests}
  % todos los
  % servidores correctos, y en particular $w$, reciben dicha petición \<DeliverTx>($t$),
  % añadiéndo a $e$ a su conjunto
  % $\THESET$, como se muestra en la línea


  % Esto implica que el servidor decidió dicha transacción, puesto que la petición \<DeliverTx> se
  % envía una vez que las transacciones fueron decididas y ordenadas.
  % Luego, por la propiedad de \textit{Tendermint-Agreement}, todos los nodos correctos decidirán
  % la misma transacción asociada al elemento $e$, y por tanto recibirán la petición \<DeliverTx>
  % para ella, añadiéndo a $e$ a su conjunto $\THESET$, como se muestra en la línea
  % \ref{line:abci-vanilla-set} en el Algoritmo~\ref{alg:abci-vanilla}.
\end{proof}

\begin{lemma}
  La implementación \vanilla cumple la propiedad~\ref{api:theset->history} \textit{Eventual-Get}
  de \setchain, que establece que todos los elementos añadidos son eventualmente estampados
  con un número de época.
\end{lemma}

\begin{proof}
  Sea $v$ un servidor correcto y $e \in U$. Si $e \in v.\THESET$, entonces por construcción de
  $\THESET$, necesariamente $v$ en algún momento recibió una
  petición \<DeliverTx>($t$), donde $t = e$, y $e$ es válido.
  Como se puede ver en las líneas ~\ref{line:abci-vanilla-set} y ~\ref{line:abci-vanilla-history} en
  el Algoritmo~\ref{alg:abci-vanilla}, inmediatamente después de agregar un elemento a $v.\THESET$,
  el elemento se agrega a $v.\HISTORY$. Por lo tanto, con seguridad, el elemento $e$ será eventualmente
  agregado a $v.\HISTORY$.
  Si un servidor correcto recibe una petición \<DeliverTx>($t$), necesariamente lo hace como parte de una cadena de
  peticiones \BeginBlock, [\DeliverTx, ...], \EndBlock, \Commit.
  De aquí se desprende entonces que eventualmente $v$ recibirá una petición \EndBlock que, como se muestra
  en la línea \ref{line:epoch-increment}, incrementa el número de época, provocando que, a partir de ese momento, todas las invocaciones
  a \<get> incluyan la nueva época que contiene a $e$ como parte de ella.

\end{proof}

\begin{lemma}
  La implementación \vanilla cumple la propiedad~\ref{api:local_unique_stamp} \textit{Unique Epoch} de \setchain,
  que establece que un elemento puede estar en a lo sumo una época.
\end{lemma}

\begin{proof}
  Sea $v$ un servidor correcto e ${i,i'\leq v.\EPOCH}$ con ${i\neq i'}$.
  Queremos probar que $v.\HISTORY(i)\cap{}v.\HISTORY(i')=\emptyset$.
  %

  Esto es trivialmente correcto debido a que un elemento $e$ se agrega a \HISTORY[j]
  únicamente si no pertenece a $\HISTORY[j']$ con $j' \in \{0, 1, ..., j -1\}$.
  Esto se puede ver en la cláusula \texttt{if} de la línea \ref{line:abci-vanilla-if}
  y en la definición de \<isValidTransaction> de la línea \ref{alg:van_is_valid_tx},
  presentes en el Algoritmo \ref{alg:abci-vanilla}.
  % La demostración se hará por el absurdo.
  % Sin pérdida de generalidad, suponemos que ${i < i'}$. Sea $e$ un elemento $\in v.\HISTORY(i)$ .
  % Por lo tanto, $e$ fue estampado con el número de época $i$.
  % %
  % Suponemos ahora que también se cumple $e \in v.\HISTORY(i')$. Es decir, 
  % $e$ fue estampado con el número de época $i'$.
  % Para que $e$ también haya sido añadido a la época $i'$, necesariamente debe haber llegado
  % una petición \<DeliverTx> para la transacción asociada al elemento $e$ que, al procesarse,
  % incluyó a $e$ como parte de $v.\HISTORY(i')$.
  % Por construcción, esto necesariamente debe haber pasado en la línea
  % \ref{line:abci-vanilla-history} en el Algoritmo~\ref{alg:abci-vanilla},
  % con lo cual se concluye que la condición del \texttt{if} de
  % la línea ~\ref{line:abci-vanilla-if} evaluó a \texttt{True} ($\star$), ya que de otro modo
  % el elemento no hubiese sido añadido.
  % %

  % Dado que ${i < i'}$ y las épocas se construyen ordenadamente, al momento de la
  % petición \<DeliverTx> mencionada anteriormente, la época $i$ ya fue construida,
  % y por lo tanto se tiene
  % $e \in v.\HISTORY(i)$.
  % Por este motivo, al procesar dicha petición \<DeliverTx>, la condición \texttt{if} de
  % la línea ~\ref{line:abci-vanilla-if} evaluará a \texttt{false}, dado que 
  % \texttt{not} $e$ \texttt{in} $v.\HISTORY$ es falso, puesto que $e \in v.\HISTORY(i)$ .
  % Esto lleva a una contradicción por ($\star$).
  % La contradicción proviene de suponer que $e \in v.\HISTORY(i')$.
  % Por lo tanto, se concluye que $e \not\in v.\HISTORY(i')$. Luego se tiene
  % $v.\HISTORY(i)\cap{}v.\HISTORY(i')=\emptyset$.
\end{proof}

\begin{lemma}
  La implementación \vanilla cumple la propiedad \ref{api:consistent-gets} \textit{Consistent Gets} de \setchain,
  que establece que los servidores están de acuerdo en el contenido
  de las épocas.
\end{lemma}

\begin{proof}
  Sean $v$ y $w$ dos servidores correctos, e $i\leq \min(v.\EPOCH, w.\EPOCH)$.
  Queremos probar que $v.\HISTORY(i)=w.\HISTORY(i)$.

  Se demostrará $v.\HISTORY(i)=w.\HISTORY(i)$ por inducción fuerte en el número de época.

  \textbf{Caso base}. 
  % Se seguirá un razonamiento similar al presentado en la demostración del
  % Lema ~\ref{proof:get-global}.
  Sea un elemento $e$ $\in v.\HISTORY(1)$~\footnote{Por definición de \setchain, $\HISTORY$ empieza en la época 1.}. 
  Por construcción de $\HISTORY$, necesariamente $v$ en un momento determinado recibió una
  petición \<DeliverTx>($t$), donde $t = e$, y $e$ es válido.
  Naturalmente, esta petición \<DeliverTx>($t$) arriba a $v$ como parte de una cadena de
  peticiones \BeginBlock, [\DeliverTx, ...], \EndBlock, \Commit.
  Esto produjo que $e$ fuera añadido a $v.\HISTORY(1)$,
  como se muestra en la línea
  \ref{line:abci-vanilla-history} en el Algoritmo~\ref{alg:abci-vanilla}.
  Posteriormente, $v.\EPOCH$
  fue incrementado de 0 a 1 como parte del proceso de la petición \EndBlock,
  como se muestra en la línea \ref{line:epoch-increment}.
  De esta forma, $e$ fue estampado con el número de época 1 en $v$.
  % Esto implica que el servidor decidió dicha transacción, puesto que la petición \<DeliverTx> se
  % envía una vez que las transacciones fueron decididas y ordenadas.
  % Luego, por la propiedad de \textit{Tendermint-Agreement}, todos los nodos correctos y, en particular $w$,
  % deciden la misma transacción asociada al elemento $e$ (en el mismo orden que $v$).
  Por la propiedad~\ref{tendermint:global-delivery} \emph{Tendermint-Global-Requests} podemos
  asegurar que todos los servidores correctos,
  y en particular $w$, reciben la cadena de peticiones \BeginBlock, [\DeliverTx, ...], \EndBlock, \Commit
  en el mismo orden en que lo hace $v$.
  % petición \<DeliverTx>($t$) en el mismo orden en que lo hace $v$, es decir,
  % durante la construcción de la época $i$,
  Por lo tanto, de forma análoga, $w$ añadirá a $e$ a $w.\HISTORY(1)$.
  Posteriormente, $w.\EPOCH$ será incrementado de 0 a 1 como parte del proceso de la petición
  \EndBlock.
  De esta forma, $e$ es estampado con el número de época 1 en $w$.
  Así queda demostrado $v.\HISTORY(1) \subseteq w.\HISTORY(1)$.
  El mismo razonamiento se puede seguir para demostrar $w.\HISTORY(1) \subseteq v.\HISTORY(1)$.
  Luego, queda demostrada la propiedad para el caso base: $v.\HISTORY(1)=w.\HISTORY(1)$.
  
  \textbf{Caso inductivo}.
  Sea un elemento $e$ $\in v.\HISTORY(n+1)$. Análogamente a lo mencionado anteriormente,
  a $v$ llegó una cadena de peticiones \BeginBlock, [\DeliverTx, ...], \EndBlock, \Commit
  que provocó que $e$ fuera añadido a $v.\HISTORY(n+1)$ y, posteriormente, que $v.\EPOCH$
  fuera incrementado de n a n+1 como parte del proceso de la petición \EndBlock.
  Por la propiedad~\ref{tendermint:global-delivery} \emph{Tendermint-Global-Requests}
  podemos asegurar que la misma cadena de peticiones llegó a $w$.
  A su vez, por la hipótesis inductiva fuerte, sabemos que
  $v.\HISTORY(j)=w.\HISTORY(j), \forall j \leq n$.
  Por lo tanto, podemos asegurar que al procesarse la petición $w.\<DeliverTx>(t)$, donde
  $t = e$, la definición de \<isValidTransaction>
  de la línea \ref{alg:van_is_valid_tx} retorna \texttt{True}, puesto que
  $e \notin w.\HISTORY(j) \forall j \leq n$.
  De este modo, $e$ es añadido a $w.\HISTORY(n+1)$, y, posteriormente, $v.\EPOCH$
  se incrementa de n a n+1 como parte del proceso de la petición \EndBlock.
  Así queda demostrado $v.\HISTORY(n+1) \subseteq w.\HISTORY(n+1)$.
  El mismo razonamiento se puede seguir para demostrar $w.\HISTORY(n+1) \subseteq v.\HISTORY(n+1)$.
  Luego, queda demostrada la propiedad para el caso inductivo: $v.\HISTORY(n+1)=w.\HISTORY(n+1)$.
\end{proof}

\begin{lemma}\label{lemma:vanilla-add-before-get}
  La implementación \vanilla cumple la propiedad~\ref{api:get->add} \textit{Add-before-Get} de \setchain,
  que determina que todo elemento en $\THESET$ proviene del resultado de un cliente
  añadiendo un elemento.
\end{lemma}

\begin{proof}
  Se cumple trivialmente por lo mencionado en \ref{subsection:proof-comments}.
\end{proof}

\subsection{\compresschain}\label{sec:proof-compresschain}

\begin{lemma}\label{lemma:compresschain-consistent-set}
  La implementación \compresschain cumple la propiedad~\ref{api:consistent-set} \textit{Consistent Sets} de \setchain.
  %que establece que las épocas solo contienen elementos que provienen del conjunto de solo crecimiento.
\end{lemma}

\begin{proof}
  Esto es trivialmente correcto debido a que la
  construcción de $\THESET$ y de $\HISTORY$ se hace a partir de los mismos elementos, como se puede ver en las líneas
  \ref{line:abci-compresschain-the-set} y \ref{line:abci-compresschain-history} en el Algoritmo~\ref{alg:abci-brotli}.
\end{proof}

\begin{lemma}
  La implementación \compresschain cumple la propiedad~\ref{api:history->theset-local} \textit{Add-Get-Local} de \setchain.
  %que declara que todo elemento válido añadido a un
  %servidor correcto $v$ es eventualmente retornado en todas las llamadas futuras a $v$.\<get>.
\end{lemma}

\begin{proof}
  Sea $e \in $ \(V\) y $v$ un servidor correcto, para los cuales un cliente invoca $v.\<add>(e)$.
  Como $e \in $ \(V\), sabemos que $e$ es válido ($\bigstar$).
  Si $e \in v.\HISTORY$, entonces, por la propiedad anterior, $e \in v.\THESET$ y la demostración es trivial.
  Suponemos entonces $e \notin v.\HISTORY$ ($\clubsuit$).
  Por definción de \<add> en el algoritmo \ref{alg:api-brotli}, la invocación
  $v.\<add>(e)$ se traduce en la llamada \texttt{CompressCollector.AddElement($e$)}.
  Como se puede observar en el algoritmo ~\ref{alg:collector-brotli}, dado que $e$ es válido,
  será codificado adecuadamente y añadido a un lote $t$ que eventualmente estará listo\footnote{Como
  se mencionó en la sección ~\ref{subsec:compresschain-algorithms},
  un lote se considera listo para ser enviado una vez que, o bien alcanza un tamaño
  máximo, o bien una cantidad razonable de tiempo transacurrió desde que el primer
  elemento llegó. Por este motivo podemos asegurar que todo lote eventualmente será
  considerado listo.} y se invocará \<broadcastTx>($t$) sobre el servidor $v$.
  Por la propiedad~\ref{tendermint:eventual-checktx} \emph{Tendermint-Eventual-CheckTx}, $v$
  recibirá eventualmente la petición \<CheckTx>($t$).
  Por $\bigstar$ se tiene que $e$ es un elemento válido y por $\clubsuit$, que $e \notin \HISTORY$,
  por lo cual, siendo $e$ un elemento de $t$
  y considerando la definición de \<CheckTx> en la línea \ref{alg:brotli_check_tx} del algoritmo
  \ref{alg:abci-brotli}, se deduce que la llamada a \<CheckTx>($t$) retornará \texttt{True}
  en $v$, por lo cual $t$ pasará a formar parte de la mempool de $v$.
  Luego, por la propiedad~\ref{tendermint:eventual-injection} \emph{Tendermint-Eventual-Injection} sabemos que la
  transacción $t$ formará parte de un bloque.
  Por definición de \<DeliverTx>, dado que $t$ es una transacción parte de un bloque,
  el servidor $v$ eventualmente recibirá una petición \<DeliverTx>($t$).
  Como se puede observar
  en la condición \texttt{if} de la línea ~\ref{line:abci-compresschain-if} en el Algoritmo~\ref{alg:abci-brotli},
  se analizan dos aspectos: que $e$ sea válido
  y que $e$ no forme parte de $\HISTORY$.
  Ambas condiciones se cumplen por $\bigstar$ y $\clubsuit$, respectivamente.
  Por lo tanto, $e$ será añadido a $\THESET$ como se indica en la
  línea ~\ref{line:abci-compresschain-the-set}, y retornado como parte de él en todas las futuras invocaciones a
  $v$.\<get>.
\end{proof}

\begin{lemma}\label{proof:compresschain-get-global}
  La implementación \compresschain cumple la propiedad~\ref{api:history->theset} \textit{Get-Global} de \setchain.
  %que establece que los elementos presentes en un servidor correcto
  %son propagados a todos los servidores correctos.
\end{lemma}

\begin{proof}
  Sean $v$ y $w$ dos servidores correctos y $e \in U$, tal que  $e \in v.\THESET$.
  Debemos probar que eventualmente se cumplirá $e \in w.\THESET$.
  Si $e \in v.\THESET$, entonces, por construcción de $\THESET$, necesariamente $e$ es válido y
  $v$ en algún momento recibió una
  petición \<DeliverTx>($t$), donde $t$ es un lote comprimido de elementos, y uno de esos elementos
  es $e$.
  Si un servidor correcto recibe una petición \<DeliverTx>($t$), necesariamente lo hace como parte de una cadena de
  peticiones \BeginBlock, [\DeliverTx, ...], \EndBlock, \Commit.
  Luego, por la propiedad~\ref{tendermint:global-delivery}
  \emph{Tendermint-Global-Requests} todos los
  servidores correctos, y en particular $w$, reciben la misma serie de peticiones, incluyendo la petición \<DeliverTx>($t$).
  Dado que $e$ es válido, se tienen dos opciones: o bien $e$ es parte de $w.\HISTORY$,
  en cuyo caso ya se cumple $e \in w.\THESET$ (ver lema \ref{lemma:compresschain-consistent-set}), o bien $e$ es un elemento
  nuevo ($e \notin w.\HISTORY$).
  En este último caso,
  $e$ será añadido a
  $w.\THESET$, como se muestra en la línea
  \ref{line:abci-compresschain-the-set} en el Algoritmo~\ref{alg:abci-brotli}.
  Por lo tanto, se cumplirá eventualmente $e \in w.\THESET$.
\end{proof}

\begin{lemma}
  La implementación \compresschain cumple la propiedad~\ref{api:theset->history} \textit{Eventual-Get}
  de \setchain.
\end{lemma}

\begin{proof}
  Sea $v$ un servidor correcto y $e \in U$. Si $e \in v.\THESET$, entonces por construcción de
  $\THESET$, necesariamente $e$ es válido y $v$ en algún momento recibió una
  petición \<DeliverTx>($t$), donde $t$ es un lote comprimido de elementos, y uno de esos elementos
  es $e$.
  Como se puede ver en la definición de \<DeliverTx> y en las líneas \ref{line:abci-compresschain-the-set}
  y \ref{line:abci-compresschain-history} de \texttt{newEpoch} en
  el Algoritmo~\ref{alg:abci-brotli}, inmediatamente después de agregar un elemento a $\THESET$,
  el elemento se agrega a $\HISTORY$. Eventualmente la iteración sobre los elementos del lote termina y,
  como se muestra en la línea \ref{line:abci-compresschain-epoch-increment} se incrementa $\EPOCH$.
  Por lo tanto, con seguridad, el elemento $e$ será
  agregado a $v.\HISTORY$ y se producirá el incremento de $v.\EPOCH$, estampando a $e$ con un número de época.
\end{proof}

\begin{lemma}
  La implementación \compresschain cumple la propiedad~\ref{api:local_unique_stamp} \textit{Unique Epoch} de \setchain.
\end{lemma}

\begin{proof}
  Sea $v$ un servidor correcto e ${i,i'\leq v.\EPOCH}$ con ${i\neq i'}$.
  Queremos probar que $v.\HISTORY(i)\cap{}v.\HISTORY(i')=\emptyset$.
  %

  Esto es trivialmente correcto debido a que un elemento $e$ se agrega a \HISTORY[j]
  únicamente si no pertenece a $\HISTORY[j']$ con $j' \in \{0, 1, ..., j -1\}$.
  Esto se puede ver en la cláusula \texttt{if} de la línea \ref{line:abci-compresschain-if}
  presente en el Algoritmo \ref{alg:abci-brotli}.
  % La demostración se hará por el absurdo.
  % Sin pérdida de generalidad, suponemos que ${i < i'}$. Sea $e$ un elemento $\in \HISTORY(i)$ .
  % Por lo tanto, $e$ fue estampado con el número de época $i$.
  % %
  % Suponemos ahora que también se cumple $e \in \HISTORY(i')$. Es decir, 
  % $e$ fue estampado con el número de época $i'$.
  % Para que $e$ también haya sido añadido a la época $i'$, necesariamente debe haber llegado
  % una petición \<DeliverTx>$(t)$, donde $t$ es un lote comprimido de elementos, y uno de esos
  % elementos es $e$. Al procesarse dicha transacción, se debe haber
  % incluido a $e$ como parte de $\HISTORY(i')$.
  % %
  % Por construcción, esto necesariamente debe haber pasado en la línea
  % \ref{line:abci-compresschain-history} en el Algoritmo~\ref{alg:abci-brotli},
  % con lo cual se concluye que la condición del \texttt{if} de
  % la línea ~\ref{line:abci-compresschain-if} evaluó a \texttt{True} ($\star$),
  % ya que de otro modo el elemento no hubiese sido añadido.
  % %

  % Dado que ${i < i'}$ y las épocas se construyen ordenadamente, al momento de la
  % petición \<DeliverTx>$(t)$, la época $i$ ya fue construida, y por lo tanto se tiene
  % $e \in \HISTORY(i)$.
  % Por este motivo, al procesar la petición \<DeliverTx>$(t)$, la condición \texttt{if} de
  % la línea ~\ref{line:abci-compresschain-if} evaluará a \texttt{false}, dado que 
  % \texttt{not} $e$ \texttt{in} $\HISTORY$ es falso, puesto que $e \in \HISTORY(i)$.
  % Esto lleva a una contradicción por ($\star$).
  % La contradicción proviene de suponer que $e \in \HISTORY(i')$.
  % Por lo tanto, se concluye que $e \not\in \HISTORY(i')$. Esto vale para cualquier elemento $e$.
  % Luego se tiene $\HISTORY(i)\cap{}\HISTORY(i')=\emptyset$.
\end{proof}

\begin{lemma}
  La implementación \compresschain cumple la propiedad~\ref{api:consistent-gets} \textit{Consistent Gets} de \setchain.
\end{lemma}

\begin{proof}
  % Sea $e$ un elemento $\in v.\HISTORY(i)$. 
  % Por construcción de $\HISTORY$, necesariamente $e$ es válido y $v$ en un momento determinado recibió una
  % petición \<DeliverTx>($t$), donde $t$ es un lote comprimido de elementos, y uno de esos
  % elementos es $e$. Como consecuencia de esta petición,
  % $e$ fue estampado con el número de época $i$.
  % Por la propiedad \emph{Tendermint-Global-Requests}~\ref{tendermint:global-delivery} podemos asegurar que todos los servidores correctos,
  % y en particular $w$, reciben la petición \<DeliverTx>($t$) en el mismo orden en que lo hace $v$, es decir,
  % durante la construcción de la época $i$, añadiendo a $e$ a $w.\HISTORY(i)$, como se muestra en la línea
  % \ref{line:abci-compresschain-history} en el Algoritmo~\ref{alg:abci-brotli}.
  % El mismo razonamiento se puede hacer para un elemento $e \in w.\HISTORY(i)$.
  % Luego, $v.\HISTORY(i)=w.\HISTORY(i)$.
  %%%%%%%%%%%%%%%%%%%%%%%%%%%%%%%%%%%%%%%%%%%%%%%%%%%%%%%%%%%%
  Sean $v$ y $w$ dos servidores correctos, e $i\leq \min(v.\EPOCH, w.\EPOCH)$.
  Queremos probar que $v.\HISTORY(i)=w.\HISTORY(i)$.

  Se demostrará $v.\HISTORY(i)=w.\HISTORY(i)$ por inducción fuerte en el número de época.

  \textbf{Caso base}. 
  % Se seguirá un razonamiento similar al presentado en la demostración del
  % Lema ~\ref{proof:get-global}.
  Sea un elemento $e$ $\in v.\HISTORY(1)$.
  Por construcción de $\HISTORY$, necesariamente $e$ es válido y $v$ en un momento determinado recibió una
  petición \<DeliverTx>($t$), donde $t$ es un lote comprimido de elementos, y uno de esos
  elementos es $e$.
  % Por construcción de $\HISTORY$, necesariamente $v$ en un momento determinado recibió una
  % petición \<DeliverTx>($t$), donde $t = e$, y $e$ es válido.
  Esto produjo que $e$ fuera añadido a $v.\HISTORY(1)$ como se muestra en la línea
  \ref{line:abci-compresschain-history}.
  Posteriormente, cuando la iteración sobre los elementos del lote finaliza,
  $v.\EPOCH$ es incrementado de 0 a 1, como se muestra en la línea
  \ref{line:abci-compresschain-epoch-increment}
  en el Algoritmo~\ref{alg:abci-brotli}.
  De esta forma, $e$ fue estampado con el número de época 1 en $v$.
  Naturalmente, esta petición \<DeliverTx>($t$) arriba a $v$ como parte de una cadena de
  peticiones \BeginBlock, [\DeliverTx, ...], \EndBlock, \Commit.
  % Esto implica que el servidor decidió dicha transacción, puesto que la petición \<DeliverTx> se
  % envía una vez que las transacciones fueron decididas y ordenadas.
  % Luego, por la propiedad de \textit{Tendermint-Agreement}, todos los nodos correctos y, en particular $w$,
  % deciden la misma transacción asociada al elemento $e$ (en el mismo orden que $v$).
  Por la propiedad~\ref{tendermint:global-delivery} \emph{Tendermint-Global-Requests} podemos
  asegurar que todos los servidores correctos,
  y en particular $w$, reciben la cadena de peticiones \BeginBlock, [\DeliverTx, ...], \EndBlock, \Commit
  en el mismo orden en que lo hace $v$.
  % petición \<DeliverTx>($t$) en el mismo orden en que lo hace $v$, es decir,
  % durante la construcción de la época $i$,
  Por lo tanto, de forma análoga, $w$ añadirá a $e$ a $w.\HISTORY(1)$ y, posteriormente,
  $w.\EPOCH$ será incrementado de 0 a 1.
  De esta forma, $e$ es estampado con el número de época 1 en $w$.
  Así queda demostrado $v.\HISTORY(1) \subseteq w.\HISTORY(1)$.
  El mismo razonamiento se puede seguir para demostrar $w.\HISTORY(1) \subseteq v.\HISTORY(1)$.
  Luego, queda demostrada la propiedad para el caso base: $v.\HISTORY(1)=w.\HISTORY(1)$.
  
  \textbf{Caso inductivo}.
  Sea un elemento $e$ $\in v.\HISTORY(n+1)$. Análogamente a lo mencionado anteriormente,
  a $v$ llegó una cadena de peticiones \BeginBlock, [\DeliverTx, ...], \EndBlock, \Commit
  que provocó que $e$ fuera añadido a $v.\HISTORY(n+1)$ y que $v.\EPOCH$
  fuera incrementado de n a n+1.
  Por la propiedad~\ref{tendermint:global-delivery} \emph{Tendermint-Global-Requests}
  podemos asegurar que la misma cadena de peticiones llegó a $w$.
  A su vez, por la hipótesis inductiva fuerte, sabemos que
  $v.\HISTORY(j)=w.\HISTORY(j), \forall j \leq n$.
  Por lo tanto, podemos asegurar que al procesarse la petición $w.\<DeliverTx>(t)$,
  donde $t$ es un lote comprimido de elementos, y uno de esos
  elementos es $e$, la cláusula \texttt{if} de la línea \ref{line:abci-compresschain-if}
  retorna \texttt{True}, puesto que
  $e \notin w.\HISTORY(j) \forall j \leq n$.
  % $t = e$, la definición de \<isValidTransaction>
  % de la línea \ref{alg:van_is_valid_tx} retorna \texttt{True}, puesto que
  % $e \notin w.\HISTORY(j) \forall j \leq n$.
  De este modo, $e$ es añadido a $w.\HISTORY(n+1)$, y $v.\EPOCH$ se incrementa de n a n+1.
  Así queda demostrado $v.\HISTORY(n+1) \subseteq w.\HISTORY(n+1)$.
  El mismo razonamiento se puede seguir para demostrar $w.\HISTORY(n+1) \subseteq v.\HISTORY(n+1)$.
  Luego, queda demostrada la propiedad para el caso inductivo: $v.\HISTORY(n+1)=w.\HISTORY(n+1)$.
\end{proof}

\begin{lemma}
  La implementación \compresschain cumple la propiedad~\ref{api:get->add} \textit{Add-before-Get} de \setchain.
\end{lemma}

\begin{proof}
  Se cumple trivialmente por lo mencionado en \ref{subsection:proof-comments}.
\end{proof}

\subsection{\hashchain}\label{sec:proof-hashchain}

\subsubsection{Consideraciones generales}\label{subsection:hashchain-proof-comments}
Para estas demostraciones consideramos que dos lotes de elementos distintos nunca
tienen el mismo hash asociado, es decir, no se considera la posibilidad de colisiones.

\subsubsection{Propiedades derivadas}
Con el objetivo de hacer las demostraciones más sencillas, se definen y demuestran las siguientes
propiedades derivadas.

\begin{property}[Haschain-Consolidation]\label{tendermint:hashchain-consolidation}
  Si un hash $h$ consolida en un servidor correcto, eventualmente consolidará en todos
  los servidores correctos.
\end{property}
  
\begin{proof}
  Sea $v$ un servidor correcto.
  Sea $h$ un hash que consolida en $v$.
  Por lo tanto, \<shouldConsolidateHash>($h$) retornó \texttt{True}, es decir, se recolectaron
  \SPH firmas para $h$.
  Dado que en \texttt{hash\_to\_signatures} solo se añaden firmas en la línea \ref{alg:hash-new-sign}
  de \<DeliverTx> tras chequear que la misma sea válida (ver línea \ref{line:hash-check-sign}),
  concluimos que $v$ recibió \SPH peticiones \<DeliverTx> para $h$ con distintas firmas válidas.
  Dichas peticiones \<DeliverTx> arriban como parte de una cadena
  de peticiones \BeginBlock, [\DeliverTx, ...], \EndBlock, \Commit.
  Por la propiedad~\ref{tendermint:global-delivery} \emph{Tendermint-Global-Requests}
  podemos asegurar que la misma cadena de peticiones llegará a todos los servidores correctos.
  De este modo, todos los servidores correctos recibirán \SPH firmas válidas y distitnas para $h$,
  de modo que dicho hash consolidará.
\end{proof}

\begin{property}[Haschain-Consolidation]\label{tendermint:hashchain-consolidation}
  Si un hash $h$ consolida en un servidor correcto $v$, eventualmente todos los elementos
  asociados a $h$ son añadidos a $v.\HISTORY$.
\end{property}
  
\begin{proof}
  Sea $v$ un servidor correcto.
  Sea $h$ un hash que consolida en $v$.
  Por lo tanto, \<shouldConsolidateHash>($h$) retornó \texttt{True}, es decir, se recolectaron
  \SPH firmas válidas y distintas para $h$.
  Dado que $\SPH \geq f + 1$, podemos asegurar que al menos un servidor correcto (llamemoslo $w$) firmó $h$.
  Con seguridad, dicho servidor correcto conoce el hash y, por lo tanto, garantiza que lo revertirá
  cuando $v$ procese la petición \<DeliverTx> asociada a $h$ y a la firma de $w$ (si es que $v$
  aún no tenía el lote asociado a $h$). 
  De este modo, se asegura que $v$ tiene los elementos asociados a $h$ y, por lo tanto, podrá añadirlos
  a \HISTORY cuando se procese \<updateHistory>.
  El mismo razonamiento se puede utilizar para las épocas anteriores.
  
  NOTA: habría que usar inducción en el nro de época.
\end{proof}

\begin{property}[Haschain-Broadcast-Consolidation]\label{tendermint:hashchain-broadcast-consolidation}
Sea $v$ un servidor correcto. Sea $e$ un elemento válido, tal que $e \in l$, y $h$, el hash asociado al lote $l$.
Si se invoca \<broadcastTx>($h, s$) en $v$, donde $s$ es la firma de $h$ con la clave privada de $v$,
entonces eventualmente $e \in v.\HISTORY$.
\end{property}

\begin{proof}
  Supongamos que se invoca \<broadcastTx>($h, s$) en $v$.
  Por la propiedad~\ref{tendermint:eventual-checktx} \emph{Tendermint-Eventual-CheckTx}, $v$
  recibirá eventualmente la petición \<CheckTx>($h, s$).
  Siguiendo la definición de \<CheckTx>~\ref{alg:hash_check_tx}, si la firma es válida ($s$ lo es puesto que
  $v$ es correcto) se abren dos casos: que $v$ conozca el hash o que aún no lo conozca.
  Si $v$ no tiene el lote $l$ asociado al hash $h$, entonces trivialmente retornará \texttt{True}.
  Si $v$ tiene el lote $l$ asociado al hash $h$, entonces si $l$ pasa \<isValidBatch>, retornará \texttt{True}.
  Por el contrario, si $l$ no pasa \<isValidBatch>, siendo que $e \in l$ y $e$ es válido, necesariamente tenemos
  que $e \in v.\HISTORY$. En tal caso, la propiedad queda demostrada.

  Suponemos entonces que $v$.\<CheckTx>($h, s$) retorna \texttt{True}.
  Por lo tanto, $t$ pasará a formar parte de la mempool de $v$.
  Luego, por la propiedad~\ref{tendermint:eventual-injection} \emph{Tendermint-Eventual-Injection}
  sabemos que la transacción $t$ formará parte de un bloque.
  Por definición de \<DeliverTx>, dado que $t$ es una transacción parte de un bloque,
  el servidor $v$ eventualmente recibirá una petición \<DeliverTx>($t$) como parte de una cadena
  de peticiones \BeginBlock, [\DeliverTx, ...], \EndBlock, \Commit.
  Por la propiedad~\ref{tendermint:global-delivery} \emph{Tendermint-Global-Requests}
  podemos asegurar que la misma cadena de peticiones llegará a todos los servidores correctos.
  Como parte de la ejecución de \<DeliverTx>($t$), si aún no tienen el lote $l$ asociado,
  los servidores correctos intentarán revertir $h$ comunicándose con el \hcollector
  de $v$, quien, con seguridad, devolverá el lote asociado a $h$.
  Así, como parte de \<tryReverse> todos los servidores correctos firmarán $h$ con
  su propia clave privada.
  Siguiendo un razonamiento idéntico al ya explicado, podemos concluir que se recibirán peticiones
  \<DeliverTx> para todas las transacciones del tipo $(h, s')$, donde $s'$ es una firma
  válida para $h$ generada con la clave privada de un servidor correcto, haciendo que el hash
  consolide en $v$ y, por tanto, todos sus elementos sean añadidos a $v.\HISTORY$ (incluyendo $e$).
  La única excepción a esto se dará en el caso en que \<CheckTx>($h, s'$) retorne \texttt{False},
  significando que $e in w.\HISTORY$ para algún servidor $w$ correcto.
  En ese caso, un hash $h'$ que contiene a $e$ consolidó en $w$ y, por la propiedad
  \ref{tendermint:hashchain-consolidation}, eventualmente consolidará en $v$, añadiendo a
  $e$ a $v.\HISTORY$.
\end{proof}

% \begin{property}[Haschain-Delivery]\label{tendermint:hashchain-delivery}
%   % Sea $e$ un elemento válido y $v$ un servidor correcto. Luego de cada invocación
%   % $v.\<add>(e)$ por parte de un cliente, eventualmente se produce
%   % una petición $v.\<DeliverTx>(t)$, en donde.

%   Sea $v$ un servidor correcto y $t$ una tupla $(h, s)$ donde
%   \begin{itemize}
%     \item $h$ es un lote hasheado de elementos, donde uno de esos elementos es válido,
%     \item y $s$ es una firma válida de dicho hash realizada con la clave privada de $v$.
%   \end{itemize}
%   Si $v$ invoca \<broadcastTx>($t$), eventualmente todos los servidores
%   correctos conocen el lote asoaciado a $h$.%se produce una petición $v.\<DeliverTx>(t)$.
 
% \end{property}

% \begin{proof}
%   Supongamos que $v$ invoca \<broadcastTx>($t$).
%   Siempre que un servidor correcto invoca \<broadcastTx> para un hash firmado con su propia clave privada,
%   es porque su propio \hcollector persistió en su base de datos el lote asociado a dicho hash previamente.
%   Por la propiedad~\ref{tendermint:eventual-checktx} \emph{Tendermint-Eventual-CheckTx}, $v$
%   recibirá eventualmente la petición \<CheckTx>($t$).
%   Siguiendo la definición de \<CheckTx>~\ref{alg:hash_check_tx}, si la firma es válida ($s$ lo es por hipótesis)
%   se abren dos casos: que $v$ conozca el hash o que aún no lo conozca.
  
%   Si $v$ no tiene el lote asociado al hash $h$, entonces trivialmente retornará \texttt{True}.
%   Por lo tanto, $t$ pasará a formar parte de la mempool de $v$.
%   Luego, por la propiedad~\ref{tendermint:eventual-injection} \emph{Tendermint-Eventual-Injection}
%   sabemos que la transacción $t$ formará parte de un bloque.
%   Por definición de \<DeliverTx>, dado que $t$ es una transacción parte de un bloque,
%   el servidor $v$ eventualmente recibirá una petición \<DeliverTx>($t$), como parte de una cadena
%   de peticiones \BeginBlock, [\DeliverTx, ...], \EndBlock, \Commit.
%   Por la propiedad~\ref{tendermint:global-delivery} \emph{Tendermint-Global-Requests}
%   podemos asegurar que la misma cadena de peticiones llegará a todos los servidores correctos.
%   Como parte de la ejecución de \<DeliverTx>($t$), intentarán revertir $h$ comunicándose con el \hcollector
%   de $v$, quien, con seguridad, devolverá el lote asociado a $h$.


%   Por otro lado, si $v$ tiene el lote asociado a $h$, entonces es porque necesariamente previamente lo revirtió,
%   mediante la ejecución de \<tryReverse>. Sin embargo, 
%   , entonces retornará \texttt{True} puesto que es un hash asociado a un lote válido
%   por hipótesis (al menos uno de sus elementos es válido y nuevo).
  
% \end{proof}

% \begin{property}[Haschain-Eventual-Consolidation]\label{tendermint:hashchain-eventual-consolidation}
%   Sean $v$ y $w$ dos servidores. Sea $t$ una tupla $(h, s)$ donde
%   \begin{itemize}
%     \item $h$ es un lote hasheado de elementos con al menos un elemento válido y nuevo,
%     \item y $s$ es una firma válida de dicho hash realizada con la clave privada de $v$.
%   \end{itemize}
%   Supongamos que $v$ está dispuesto a revertir el hash al menos a $w$.
%   Luego, si $w$ es un servidor correcto y recibe una petición $\<DeliverTx>(t)$,
%   eventualmente el hash $h$ consolidará.
%   % Sea $e$ un elemento válido y $v$ un servidor correcto. Si un cliente invoca
%   % $v.\<add>(e)$, eventualmente un hash proveniente de un lote que contiene a $e$ consolidará.
% \end{property}

% \begin{proof}
%   % Sea $e$ un elemento válido y $v$ un servidor correcto.
%   % Si un cliente invoca $v.\<add>(e)$, sabemos por la propiedad \textit{Haschain-Delivery} que,
%   % eventualmente $v$ recibirá la petición $\<DeliverTx>(t)$, en donde
%   % $t$ es una tupla $(h, s)$, siendo
%   % \begin{itemize}
%   %   \item $h$ un lote hasheado de elementos, donde uno de esos elementos es $e$,
%   %   \item y $s$ una firma válida de dicho hash realizada con la clave privada de $v$.
%   % \end{itemize}
%   % Por la propiedad \textit{Tendermint-Global-Requests}, sabemos que todos los nodos correctos
%   % recibirán eventualmente la misma petición $\<DeliverTx>(t)$.

%   Sea $w$ un nodo correcto que recibe $\<DeliverTx>(t)$. Como lo indica la definición presentada
%   en el Algoritmo~\ref{alg:abci-hash1},
%   $w$ chequeará que $s$ sea una firma válida (lo cual vale por hipótesis) y luego sumará la nueva
%   firma a las firmas para el hash $h$.
%   Seguido de esto, si $w$ no conoce aún el reverso de $h$, entonces lanzará una acción asíncrona
%   para revertirlo
%   (ver línea \ref{alg:spawn-reverse}).
%   Si $w$ ya conoce el reverso de $h$ es porque ya invocó al proceso de inversión de hashes previamente.

%   Supongamos que $w$ no conoce el reverso de $h$.
%   Como se puede ver en la definición de \<tryReverse>~\ref{alg:hash_revert},
%   $w$ se conectará con su propio \hcollector (mediante la invocación
%   \<Reverse> del algoritmo ~\ref{alg:collector-hash}) para obtener el lote asociado a $h$.
%   Con seguridad obtendrá el reverso puesto que su \hcollector se comunicará con el \hcollector de $v$ (que
%   por hipótesis está dispuesto a revertir $h$) porque $s$ es una firma asociada a $v$.
%   Por su parte, el \hcollector de $v$ ejecutará \<Reverse>,
%   obteniendo el lote desde su base de datos y devolviéndolo.

%   Siguiendo la definición de \<tryReverse> en $w$, como se muestra en las líneas \ref{alg:hash-sign-hash}
%   y \ref{alg:hash-broadcast-hash}, una vez que obtiene el lote asociado a $h$,
%   firmará $h$ con su propia clave privada y lo difundirá como
%   una nueva transacción invocando \<broadcastTx>($t'$), donde $t' = (h, s')$, siendo $s'$ la firma
%   generada con la clave privada de $w$.
%   Por la propiedad \emph{Haschain-Delivery}~\ref{tendermint:hashchain-delivery},
%   eventualmente llegará una nueva petición $\<DeliverTx>(t')$.
%   Luego, por la propiedad \emph{Tendermint-Global-Requests}~\ref{tendermint:global-delivery},
%   todos los servidores correctos
%   recibirán $\<DeliverTx>(t')$, agregando una nueva firma para el hash $h$,
%   como se muestra en la línea \ref{alg:hash-new-sign}.

%   Dado que este proceso se repetirá para todo servidor correcto, y considerando que
%   \SPH es menor o igual a la cantidad de nodos
%   correctos, con certeza $h$ obtendrá las \SPH firmas (puesto que cada servidor colaborará
%   con su firma) y, por lo tanto, consolidará.
% \end{proof}

\subsubsection{Lemas}

\begin{lemma}
  La implementación \hashchain cumple la propiedad~\ref{api:consistent-set} \textit{Consistent Sets} de \setchain.
  %que establece que las épocas solo contienen elementos que provienen del conjunto de solo crecimiento.
\end{lemma}

\begin{proof}
  Sea $v$ un servidor correcto sobre el que se invoca $v.\<get>$.
  Por definición de \<get> en el Algoritmo~\ref{alg:api-hashchain}, esto se traduce como una llamada a
  \<Query> en el Algoritmo~\ref{alg:abci-hash1}.
  Sea $(\THESET, \HISTORY, \EPOCH)$ el resultado de dicha invocación.
  Queremos probar que para todo $i \leq \EPOCH, \HISTORY(i) \subseteq \THESET$.
  Sea $e$ un elemento $\in \HISTORY(i)$ para algún $i \leq \EPOCH$.
  Por construcción de \HISTORY, $e$ fue necesariamente agregado en la línea \ref{line:abci-hashchain-history}
  como parte de la ejecución de \<updateHistory>.
  Se deduce entonces que $e$ es un elemento válido proveniente de un lote asociado al hash correspondiente a
  la época $i$.

  Dado que la variable \texttt{hash\_to\_batch} solo se modifica en la línea \ref{line:abci-hashchain-hash-to-batch} como
  parte de \<tryReverse>~\ref{alg:hash_revert},
  se concluye que dicho código tiene que haber sido ejecutado previamente, lo cual implica también que
  $e$ fue añadido a $\THESET$ con anterioridad.
  Por lo tanto, se tiene $e \in \THESET$, concluyendo que $\HISTORY(i) \subseteq \THESET$ para todo $i \leq \EPOCH$.
  % Esto es trivialmente correcto debido a que la
  % construcción de $\THESET$ y de $\HISTORY$ se hacen a partir de los mismos elementos (ver definición de
  % \<tryReverse> y \<Query> en el Algoritmo~\ref{alg:abci-hash1} y ~\ref{alg:abci-hash2}).
\end{proof}

\begin{lemma}
  La implementación \hashchain cumple la propiedad~\ref{api:history->theset-local} \textit{Add-Get-Local} de \setchain.
  %que declara que todo elemento válido añadido a un
  %servidor correcto $v$ es eventualmente retornado en todas las llamadas futuras a $v$.\<get>.
\end{lemma}

\begin{proof}
  Sea $e$ un elemento válido y $v$ un servidor correcto, para los cuales un cliente invoca $v.\<add>(e)$.
  Por definción de \<add> en el algoritmo \ref{alg:api-hashchain}, la invocación
  $v.\<add>(e)$ se traduce en la llamada \texttt{HashCollector.AddElement($e$)}.
  Como se puede observar en el algoritmo ~\ref{alg:collector-hash}, $e$ será codificado
  adecuadamente y añadido a un lote $l$ que eventualmente estará listo, se hasheará,
  se firmará, y se invocará \<broadcastTx>($t$) sobre el servidor $v$, en donde $t = (h, s)$
  será una tupla conformada por el hash del lote que contiene a $e$
  y una firma válida para el mismo ($\bigstar$). A su vez, la correspondencia entre el hash y el lote
  será añadido a una base de datos local.

  Por la propiedad~\ref{tendermint:eventual-checktx} \emph{Tendermint-Eventual-CheckTx}, $v$
  recibirá eventualmente la petición \<CheckTx>($t$).
  Consideremos la definición de la línea \ref{alg:hash_check_tx} en el Algoritmo~\ref{alg:abci-hash1}.
  Dado que la firma $s$ es válida por $\bigstar$, se abren dos casos:
  que el lote asociado a $h$ ya sea conocido o no.

  Si el lote asociado a $h$ ya es conocido, es decir, si $h$ \texttt{in hash\_to\_batch},
  entonces con certeza previamente se recibió una petición \<CheckTx> o \<DeliverTx> asociada al
  hash $h$,
  $h$ fue revertido, y se añadió su lote asociado\footnote{Por lo mencionado en
  las consideraciones generales de \hashchain \ref{subsection:hashchain-proof-comments}, dicho lote es $l$,
  puesto que no se considera la posibilidad de colisiones.} a \texttt{hash\_to\_batch}.
  Esto únicamente se hace como parte de \<tryReverse>~\ref{alg:hash_revert},
  en donde, como se puede ver en la línea
  \ref{line:abci-hashchain-the-set}, $e$ tuvo que haber sido añadido a \THESET,
  puesto que es un elemento válido por hipótesis.

  Si, por el contrario,
  el lote asociado a $h$ no es aún conocido, es decir, si \texttt{not} $h$ \texttt{in hash\_to\_batch},
  entonces se invocará a \<tryReverse>.
  De este modo, se comunicará con su propio \hcollector para revertir el hash (la firma $s$ está asociada a él).
  Con seguridad, el \hcollector tendrá el lote asociado a dicho hash en su base de datos local
  (puesto que fue el \hcollector creador de dicho hash) y lo retornará, de manera que el
  elemento $e$ (parte del lote) será añadido a $\THESET$
  de acuerdo a la línea \ref{line:abci-hashchain-the-set} en el Algoritmo~\ref{alg:abci-hash2}.

  Por lo tanto, todas las invocaciones a $v.\<get>$ eventualmente tendrán al elemento $e$
  como parte de \THESET.
\end{proof}

\begin{lemma}
  La implementación \hashchain cumple la propiedad~\ref{api:history->theset} \textit{Get-Global} de \setchain.
  %que establece que los elementos presentes en un servidor correcto
  %son propagados a todos los servidores correctos.
\end{lemma}

\begin{proof}
  % Sean $v$ y $w$ dos servidores correctos y $e \in U$, tal que  $e \in v.\THESET$.
  % Debemos probar que eventualmente se cumplirá $e \in w.\THESET$.
  % Si $e \in v.\THESET$, entonces, por construcción de $\THESET$, necesariamente $v$ en algún momento
  % ejecutó \<tryReverse>($h, s$)~\ref{alg:hash_revert}, donde $h$ es el hash asociado a un lote $l$ que contiene a $e$
  % y $s$ es una firma válida para ese hash.
  % De este modo, revirtió $h$ correctamente y, como consecuencia,
  % añadió a $e$ a $\THESET$ (ver línea \ref{line:abci-hashchain-the-set} en el Algoritmo~\ref{alg:abci-hash2}).

  % A su vez, como parte de \<tryReverse>, $v$ difundió $h$ con su propia firma $s'$ (a menos que $s$ ya fuera la firma de $v$,
  % en cuyo caso ya lo había difundido previamente).
  % En cualquier caso, podemos asegurar que se invocó 
  % \<broadcastTx>($h, s'$) en $v$.
  % Siempre que un servidor correcto invoca \<broadcastTx> para un hash firmado con su propia clave privada,
  % es porque su propio \hcollector persistió en su base de datos el lote asociado a dicho hash previamente.
  % Por la propiedad~\ref{tendermint:eventual-checktx} \emph{Tendermint-Eventual-CheckTx}, $v$
  % recibirá eventualmente la petición \<CheckTx>($h, s'$).
  % Siguiendo la definición de \<CheckTx>~\ref{alg:hash_check_tx}, si la firma es válida ($s'$ lo es puesto que
  % $v$ es correcto) se abren dos casos: que $v$ conozca el hash o que aún no lo conozca.
  % Si $v$ no tiene el lote $l$ asociado al hash $h$, entonces trivialmente retornará \texttt{True}.
  % Si $v$ tiene el lote $l$ asociado al hash $h$, entonces si $l$ pasa \<isValidBatch>, retornará \texttt{True}.
  % Por el contrario, si $l$ no pasa \<isValidBatch>, siendo que $e \in l$ y $e$ es válido, necesariamente tenemos
  % que $e \in v.\HISTORY$. En tal caso, por la propiedad 
  % Por lo tanto, $t$ pasará a formar parte de la mempool de $v$.
  % Luego, por la propiedad~\ref{tendermint:eventual-injection} \emph{Tendermint-Eventual-Injection}
  % sabemos que la transacción $t$ formará parte de un bloque.
  % Por definición de \<DeliverTx>, dado que $t$ es una transacción parte de un bloque,
  % el servidor $v$ eventualmente recibirá una petición \<DeliverTx>($t$), como parte de una cadena
  % de peticiones \BeginBlock, [\DeliverTx, ...], \EndBlock, \Commit.
  % Por la propiedad~\ref{tendermint:global-delivery} \emph{Tendermint-Global-Requests}
  % podemos asegurar que la misma cadena de peticiones llegará a todos los servidores correctos.
  % Como parte de la ejecución de \<DeliverTx>($t$), intentarán revertir $h$ comunicándose con el \hcollector
  % de $v$, quien, con seguridad, devolverá el lote asociado a $h$.

  % La función \<tryReverse> puede ser invocada como consecuencia de una petición \<CheckTx> o como consecuencia
  % de una petición \<DeliverTx>.
  % Si proviene de la ejecución de \<CheckTx>, entonces como se muestra en la línea \ref{line:spawn-reverse-checktx}
  % de su definición, inmediatamente después de lanzar una llamada asíncrona a \<tryReverse>, la petición retornó
  % \texttt{True}.
  % Como consecuencia de esto, la transacción en cuestión (que llamamos $t = (h,s)$) fue añadida a la mempool de $v$ y,
  % por la propiedad~\ref{tendermint:eventual-injection} \emph{Tendermint-Eventual-Injection} sabemos que eventualmente
  % $v$ recibirá una petición $\<DeliverTx>(t)$.
  % Aún más, por la propiedad~\ref{tendermint:global-delivery} \emph{Tendermint-Global-Requests} todos los servidores
  % correctos recibirán $\<DeliverTx>(t)$.

  % Por lo tanto, podemos concluir que todos los servidores correctos y, en particular $w$, recibirán la petición $\<DeliverTx>(t)$.
  % Como consecuencia de esta ejecutarán \<tryReverse> (si es que ya no lo hicieron) añadiendo a $e$ a \THESET. 
\end{proof}

\begin{lemma}
  La implementación \hashchain cumple la propiedad \textit{Eventual-Get}~\ref{api:theset->history}
  de \setchain.
\end{lemma}

\begin{proof}
  Sea $v$ un servidor correcto y $e \in U$. Si $e \in v.\THESET$ y $e \in \HISTORY$, luego
  la propiedad vale trivialmente. Suponemos que $e \notin \HISTORY$.
  Si $e \in v.\THESET$, entonces por construcción de
  $\THESET$, necesariamente $e$ es válido y $v$ en algún momento ejecutó \<tryReverse>~\ref{alg:hash_revert} para un
  hash $h$, donde el lote $l$ asociado a $h$ contenía a $e$ como elemento.
  Como parte de \<tryReverse>, $v$ difundió $h$ con su propia firma $s'$ (a menos que $s$ ya fuera la firma de $v$,
  en cuyo caso ya lo había difundido previamente).
  En cualquier caso, podemos asegurar que se invocó 
  \<broadcastTx>($h, s'$) en $v$.
  Luego, por la propiedad \ref{tendermint:hashchain-broadcast-consolidation}, podemos asegurar que eventualmente
  $e \in v.\HISTORY$.
  %Siempre que un servidor correcto invoca \<broadcastTx> para un hash firmado con su propia clave privada,
  %es porque su propio \hcollector persistió en su base de datos el lote asociado a dicho hash previamente.
  % Por la propiedad~\ref{tendermint:eventual-checktx} \emph{Tendermint-Eventual-CheckTx}, $v$
  % recibirá eventualmente la petición \<CheckTx>($h, s'$).
  % Siguiendo la definición de \<CheckTx>~\ref{alg:hash_check_tx}, si la firma es válida ($s'$ lo es puesto que
  % $v$ es correcto) se abren dos casos: que $v$ conozca el hash o que aún no lo conozca.
  % Si $v$ no tiene el lote $l$ asociado al hash $h$, entonces trivialmente retornará \texttt{True}.
  % Si $v$ tiene el lote $l$ asociado al hash $h$, entonces si $l$ pasa \<isValidBatch>, retornará \texttt{True}.
  % Por el contrario, si $l$ no pasa \<isValidBatch>, siendo que $e \in l$ y $e$ es válido, necesariamente tenemos
  % que $e \in v.\HISTORY$. En tal caso, la propiedad queda demostrada.

  % Suponemos entonces que $v$.\<CheckTx>($h, s'$) retorna \texttt{True}.
  % Por lo tanto, $t$ pasará a formar parte de la mempool de $v$.
  % Luego, por la propiedad~\ref{tendermint:eventual-injection} \emph{Tendermint-Eventual-Injection}
  % sabemos que la transacción $t$ formará parte de un bloque.
  % Por definición de \<DeliverTx>, dado que $t$ es una transacción parte de un bloque,
  % el servidor $v$ eventualmente recibirá una petición \<DeliverTx>($t$), como parte de una cadena
  % de peticiones \BeginBlock, [\DeliverTx, ...], \EndBlock, \Commit.
  % Por la propiedad~\ref{tendermint:global-delivery} \emph{Tendermint-Global-Requests}
  % podemos asegurar que la misma cadena de peticiones llegará a todos los servidores correctos.
  % Como parte de la ejecución de \<DeliverTx>($t$), si aún no tienen el lote $l$ asociado,
  % intentarán revertir $h$ comunicándose con el \hcollector
  % de $v$, quien, con seguridad, devolverá el lote asociado a $h$.
  % Así, como parte de \<tryReverse> todos los servidores correctos firmarán $h$ con su propia clave privada,

  % La función \<tryReverse> pudo ser invocada como consecuencia de una petición \<CheckTx> o como consecuencia
  % de una petición \<DeliverTx>.

  % Si proviene de la ejecución de \<CheckTx>, entonces como se muestra en la línea \ref{line:spawn-reverse-checktx}
  % de su definición, inmediatamente después de lanzar una llamada asíncrona a \<tryReverse>, la petición retornó
  % \texttt{True}.
  % Como consecuencia de esto, la transacción en cuestión (llamémosla $t = (h,s)$) fue añadida a la mempool de $v$ y,
  % por el lema~\ref{tendermint:eventual-injection} \emph{Tendermint-Eventual-Injection} sabemos que
  % eventualmente $v$ recibirá una petición $\<DeliverTx>(t)$.

  % Por lo tanto, podemos concluir que eventualmente $v$ recibirá una petición \<DeliverTx>($t$).
  % Luego, considerando que el servidor que generó la firma $s$ estuvo dispuesto a revertir $h$ a $v$,
  % por la propiedad \emph{Haschain-Eventual-Consolidation}\ref{tendermint:hashchain-eventual-consolidation}
  % podemos concluir que el hash $h$ consolidará y, finalmente, $v$ añadirá a $e$ a $\HISTORY$ 
  % como se muestra en la línea \ref{line:abci-hashchain-history} en el Algoritmo~\ref{alg:abci-hash1}.
\end{proof}

\begin{lemma}
  La implementación \hashchain cumple la propiedad \textit{Unique Epoch}~\ref{api:local_unique_stamp} de \setchain.
\end{lemma}

\begin{proof}
  Sea $v$ un servidor correcto e ${i,i'\leq v.\EPOCH}$ con ${i\neq i'}$.
  Queremos probar que $v.\HISTORY(i)\cap{}v.\HISTORY(i')=\emptyset$.
  %

  Esto es trivialmente correcto debido a que un elemento $e$ se agrega a \HISTORY[j]
  únicamente si no pertenece a $\HISTORY[j']$ con $j' \in \{0, 1, ..., j -1\}$, como se puede
  ver en la cláusula \texttt{if} de la línea \ref{line:abci-hashchain-history-if}
  en el Algoritmo \ref{alg:abci-hash1}.
  %
\end{proof}

\begin{lemma}
  La implementación \hashchain cumple la propiedad \textit{Consistent Gets}~\ref{api:consistent-gets} de \setchain.
\end{lemma}

\begin{proof}
  % Sean $v$ y $w$ dos servidores correctos, e $i\leq \min(v.\EPOCH, w.\EPOCH)$.
  % Queremos probar que $v.\HISTORY(i)=w.\HISTORY(i)$.

  % Sea $e$ un elemento $\in v.\HISTORY(i)$. 
  % Por construcción de $\HISTORY$, necesariamente $e$ es válido y pertenece al lote asociado a un hash $h$ que
  % ya consolidó.
  % Notemos que para añadir un nuevo elemento a $\HISTORY$, el hash en cuestión debe tener una época asignada
  % (a través de la variable \texttt{epoch\_to\_hash}) como se muestra en la línea \ref{alg:hash-epoch-to-hash-use}
  % en el Algoritmo~\ref{alg:abci-hash1}.
  % Solo se asignan épocas a hashes cuando estos consolidan como se ve en la línea \ref{alg:hash-epoch-to-hash}.

  % De la definición de \<shouldConsolidateHash>\ref{alg:hash_consolidated} presentada
  % en el Algoritmo~\ref{alg:abci-hash2} y del hecho de que las firmas se agregan
  % a \texttt{hash\_to\_signatures} solo en la línea \ref{alg:hash-new-sign} de \<DeliverTx>, se concluye que
  % tienen que haber llegado al menos \SPH peticiones \<DeliverTx> para el hash $h$ con firmas de distintos servidores.
  % Por la propiedad \emph{Tendermint-Global-Requests}~\ref{tendermint:global-delivery} podemos asegurar que todos
  % los servidores correctos, y en particular $w$, reciben las mismas peticiones \<DeliverTx>
  % y en el mismo orden en que lo hace $v$.
  % A su vez, podemos asegurar que todos los nodos correctos pueden revertir $h$ puesto que $v$ es un servidor
  % correcto que conoce el lote asociado y revertirá el hash a quien se lo pida.
  % De este modo, el hash $h$ consolida en todos los servidores
  % durante la construcción de la época $i$, añadiendo a $e$ a $w.\HISTORY(i)$, como se muestra en la línea
  % \ref{line:abci-hashchain-history} en el Algoritmo~\ref{alg:abci-hash1}.
  % El mismo razonamiento se puede hacer para un elemento $e \in w.\HISTORY(i)$.
  % Luego, $v.\HISTORY(i)=w.\HISTORY(i)$.
\end{proof}

\begin{lemma}
  La implementación \hashchain cumple la propiedad \textit{Add-before-Get}~\ref{api:get->add} de \setchain.
\end{lemma}

\begin{proof}
  Se cumple trivialmente por lo mencionado en \ref{subsection:proof-comments}.
\end{proof}

\section{Conclusión}
Habiendo demostrado que \vanilla, \compresschain y \hashchain cumplen todas las propiedaes
de \setchain presentadas en la sección \ref{sec:setchain-properties}, queda formalmente probado
que las tres soluciones propuestas en este trabajo son implementaciones correctas de \setchain.
% \begin{proof}
%   Sea $v$ un servidor correcto y $e \in \THESET$.
%   ...
%   ¿un servidor maligno no podría invocat Tendermint.Broadcast(h,s)?
% \end{proof}

% \subsection{Otras cosas}
% \begin{property}[Safety]\label{tendermint:safety}
%   Si hay menos de 1/3 del poder de voto en manos bizantinas y al menos un validador correcto
%   decide el bloque $B$, entonces ningún validador correcto decidirá un bloque distinto a $B$.
% \end{property}

% \begin{property}[Liveness]\label{tendermint:liveness}
%   Si hay menos de 1/3 de poder de voto en manos bizantinas, luego el procolo de Tendermint
%   no se bloquea (es decir, no hay \textit{deadlock}).
% \end{property}

% En el contexto de Tendermint, denotamos con \textit{n} el poder de voto total de los procesos
% en el sistema, y asumimos una cota superior \textit{f} en el poder de voto total proveniente de
% los procesos bizantinos.
% %
% El algoritmo de consenso de Tendermint asume que $n > 3f$, es decir, requiere que el poder de voto
% en manos de procesos bizantinos sea menor a 1/3 del poder de voto total.
% Por simplicidad, en ocasiones se considera $n = 3f + 1$.

% Tendermint resuelve el problema de la replicación de máquinas de estados ejecutando instancias de
% consenso secuencialmente para acordar cada bloque de transacciones que serán luego ejecutadas por el
% servicio siendo replicado.
% Se considera una variante del problema de consenso bizantino llamado \textit{Validity Predicate-based Byzantine consensus}
% que es motivado por los sistemas de blockchain~\cite{Crain2017LeaderRandomizationSignaturefreeBC}.
% El problema es definido por propiedades de acuerdo, terminación, y validés.

% Esta variante del problema de consenso bizantino tiene un predicado específico a la aplicación
% denotado \textit{valid()} que indica si un valor es válido. En el contexto de sistemas de blockchain, por ejemplo,
% un valor no es válido si no contiene un hash apropiado del último bloque añadido a la blockchain.

% \begin{property}[Agreement]\label{tendermint:agreement1}
%   Dos procesos correctos nunca deciden valores distintos.
% \end{property}

% \begin{property}[Termination]\label{tendermint:termination1}
%     Todos los procesos correctos deciden eventualmente un valor.
% \end{property}

% \begin{property}[Validity]\label{tendermint:validity}
%     Un valor decidido es válido. Es decir, satisface el predicado predefinido
%     denotado con \textit{valid}.
% \end{property}
% %

% Las propiedades ~\ref{tendermint:safety} y ~\ref{tendermint:liveness} son demostradas en ~\cite{Kwon2014TendermintC}.
% Las propiedades ~\ref{tendermint:agreement1}, ~\ref{tendermint:termination1} y ~\ref{tendermint:validity}
% son demostradas en ~\cite{Buchman.2018.Tendermint}.

% En esta sección se introducen las propiedades que deben cumplir las implementaciones
% correctas de \setchain.
% La especificación dada asume que los clientes interactúan con servidores correctos,
% aunque en realidad los clientes no pueden estar seguros de que el servidor que están
% contactando no sea bizantino.
% Como ya fue mencionado anteriormente, las propiedades siempren razonan sobre los servidores
% correctos, dado que los servidores bizantinos no proveen ninguna garantía.
% %

% Empezamos requiriendo que en \setchain toda operación \<add> y \<get> ejecutada sobre un servidor
% correcto eventualmente termine.
% Se dice que un elemento $e$ está en la época $i$ en el historial $H$ (retornado por una invocación
% a \<get>) si $e \in H(i)$.
% Se dice que un elemento $i$ está en $H$ is existe una época $i$ tal que $e \in H(i)$.

% % ? Esta es la property que Marga indicó que falta en el preliminar de Setchain
% La primera propiedad establece que las épocas solo contienen elementos que provienen del conjunto
% de solo crecimiento.

% % 1. Cada elemento válido agregado en un servidor correcto eventualmente es retor-
% % nado en todas las futuras invocaciones a get hechas sobre servidores correctos.
% La segunda propiedad declara que todo elemento añadido a un servidor correcto es eventualmente
% retornado en todas las llamadas a \<get> futuras.

% % ?
% La siguiente propiedad establece que los elementos presentes es un servidor correcto son propagados
% a todos los servidores correctos.

% Se asume en el resto de este trabajo que en cualquier punto de tiempo hay un instante futuro en el
% cual una nueva época será generada. Esto es un supuesto razonable en cual escenario práctico.

% % 2. Todos los elementos válidos agregados en un servidor correcto deben eventual-
% % mente ser estampados con un número de época en todos los servidores correctos.
% La siguiente propiedad establece que todos los elementos añadidos son eventualmente estampados
% con un número de época.

% % ?
% La siguiente propiedad establece que un elemento puede estar en a lo sumo una época,
% y que ningún elemento puede estar en dos épocas diferentes incluso si los conjuntos historiales se
% obtienen de invocaciones \<get> a distintos servidores (ambos correctos).

% % 4. Dos servidores correctos cualesquieras están de acuerdo en el contenido de todas
% % las épocas que hayan sido computadas
% La siguiente propiedad establece que los servidores están de acuerdo en el contenido
% de las épocas.

% % 5. Cada elemento que se estampa con una época proviene del resultado de un cliente
% % añadiendo el elemento.
% Finalmente, se requiere que todo elemento in el historia provenga del resultado de un cliente
% añadiendo un elemento.


% Faltarían las siguientes propiedades
