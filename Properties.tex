\section{Propiedades}\label{sec:properties}
En esta sección se introducen las propiedades que deben cumplir las implementaciones
correctas de \setchain.
La especificación dada asume que los clientes interactúan con servidores correctos,
aunque en realidad los clientes no pueden estar seguros de que el servidor que están
contactando no sea bizantino.
Como ya fue mencionado anteriormente, las propiedades siempren razonan sobre los servidores
correctos, dado que los servidores bizantinos no proveen ninguna garantía.
%

Empezamos requiriendo que en \setchain toda operación \<add> y \<get> ejecutada sobre un servidor
correcto eventualmente termine.
Se dice que un elemento $e$ está en la época $i$ en el historial $H$ (retornado por una invocación
a \<get>) si $e \in H(i)$.
Se dice que un elemento $i$ está en $H$ is existe una época $i$ tal que $e \in H(i)$.

% ? Esta es la property que Marga indicó que falta en el preliminar de Setchain
La primera propiedad establece que las épocas solo contienen elementos que provienen del conjunto
de solo crecimiento.

% 1. Cada elemento válido agregado en un servidor correcto eventualmente es retor-
% nado en todas las futuras invocaciones a get hechas sobre servidores correctos.
La segunda propiedad declara que todo elemento añadido a un servidor correcto es eventualmente
retornado en todas las llamadas a \<get> futuras.

% ?
La siguiente propiedad establece que los elementos presentes es un servidor correcto son propagados
a todos los servidores correctos.

Se asume en el resto de este trabajo que en cualquier punto de tiempo hay un instante futuro en el
cual una nueva época será generada. Esto es un supuesto razonable en cual escenario práctico.

% 2. Todos los elementos válidos agregados en un servidor correcto deben eventual-
% mente ser estampados con un número de época en todos los servidores correctos.
La siguiente propiedad establece que todos los elementos añadidos son eventualmente estampados
con un número de época.

% ?
La siguiente propiedad establece que un elemento puede estar en a lo sumo una época,
y que ningún elemento puede estar en dos épocas diferentes incluso si los conjuntos historiales se
obtienen de invocaciones \<get> a distintos servidores (ambos correctos).

% 4. Dos servidores correctos cualesquieras están de acuerdo en el contenido de todas
% las épocas que hayan sido computadas
La siguiente propiedad establece que los servidores están de acuerdo en el contenido
de las épocas.

% 5. Cada elemento que se estampa con una época proviene del resultado de un cliente
% añadiendo el elemento.
Finalmente, se requiere que todo elemento in el historia provenga del resultado de un cliente
añadiendo un elemento.


% Faltarían las siguientes propiedades
