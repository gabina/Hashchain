A continuación se presentan las definciones de algunos conceptos que fueron utilizados
en inglés a lo largo del trabajo, debido a la dificultad de ser traducidos razonablemente.
\begin{itemize}
     \item \textit{Application Programming Interface (API).}  Pieza de código que permite a diferentes
     aplicaciones comunicarse entre sí, compartiendo información y funcionalidades.
     \item \textit{Commit.} Acción de guardar o confirmar un nuevo bloque en la
     blockchain, haciéndolo parte permanente de ella.
     \item \textit{Commited by the network.} Un bloque se considera así cuando una cantidad necesaria de nodos
     en la red lo validaron y agregaron a la blockchain mediante el mecanismo de consenso que utilizan,
     asegurando la inmutabilidad. 
     \item \textit{Deployment.} Mecanismo por el cual un determinado software se disponibiliza para los usuarios
     en un entorno particular.
     \item \textit{Distributed grown-only sets.} Conjuntos distribuidos que solo crecen.
     \item \textit{Endpoint.} Una URL específica asociada a un recurso particular. Cuando se interactúa con una API,
     en general los endpoints ejecutan actividades específicas, como solicitar datos o iniciar procesos.
     \item \textit{Fork.} El fork de un proyecto ocurre cuando desarrolladores toman una copia del código fuente de
     un software particular, y comienzan un desarrollo independiente a partir de él, creando una pieza de software
     distinta.
     \item \textit{Front-running attack.} Ataque de adelantamiento. Estrategia maliciosa en la cual un actor
     observa una transacción que está pendiente de ser confirmada en una blockchain y, basándose en esta
     información, introduce una transacción propia con el objetivo de adelantarse a la transacción
     original para obtener un beneficio.
     \item \textit{Gossip protocol.} Protocolo de comunicación que permite intercambiar información
     entre entidades de un sistema distribuido, de modo que la información se disemina a todos los miembros de la red,
     sin necesidad de comunicación directa entre todos ellos.
     \item \textit{Liveness.} Las propiedades de liveness son aquellas que establecen que \emph{cosas buenas} ocurren
     en cada ejecución o, equivalentemente, describen algo que debe obligatoriamente pasar durante cada ejecución.
     \item \textit{Optimistic Rollup.} Protocolos de capa 2 diseñados para ampliar el rendimiento de la capa base.
     Reducen el cómputo en la cadena principal al procesar las transacciones fuera de la cadena, ofreciendo mejoras
     significativas en la velocidad de procesamiento.
     \item \textit{Proof-of-work}. Prueba de trabajo. Mecanismo por el cual el probador demuestra que
     realizó un esfuerzo computacional lo suficientemente costoso.
     \item \textit{Proof-of-stake}. Prueba de participación. Mecanismo por el cual el probador demuestra la propiedad de
     algún bien valioso, en general, alguna cryptomoneda
     \item \textit{Recursive Length Prefix (RLP).} Algoritmo de codificación estándar de Ethereum para la transferencia de datos
     entre nodos en formato eficiente en espacio.
     \item \textit{Remote Procedure Call (RPC).} Llamada a procesamiento remoto. Es una forma de comunicación inter-procesos,
     en donde un programa hace que un procedimiento se ejecute en un espacio de memoria diferente (comúnmente en otra computadora
     en una red compartida). El procedimiento  está escrito como si fuese una llamada a un procedimiento local, sin necesidad de
     que el programador dé los detalles explícitos de la interacción remota.
     \item \textit{Safety.} Las propiedades de safety son aquellas que proscriben \emph{cosas malas} durante
     la ejecución. Una propiedad de safety caracteriza lo que está permitido a través de declarar lo que está
     prohibido. 
     \item \textit{Timeout.} Período de tiempo máximo que un proceso está dispuesto a esperar por una determinada
     operación o evento. En general tiene asociado una acción a realizar si dicho período de tiempo se termina antes
     de que la operación o el evento esperado ocurra.
     \item \textit{Transacction fees.} Impuesto de transacción. Es una pequeña cantidad que se cobra a los
     usuarios que envían transacciones a la red blockchain.  
\end{itemize}