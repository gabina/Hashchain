\section{Objetivo del capítulo}\label{sec:benchmark}
En este capítulo se introducen los resultados de la evaluación empírica para las implementaciones
prototípicas de las tres alternativas propuestas a lo largo del trabajo: \vanilla, \compresschain,
y finalmente \hashchain.
La idea de los experimentos realizados es determinar la cantidad de elementos que pueden ser
añadidos a la \setchain por segundo, de modo de evaluar las siguientes hipótesis:
% Write hypothesis
\begin{itemize}
	\item La implementación de \compresschain presenta mejor rendimiento que la de \vanilla.
	\item La implementación de \hashchain presenta mejor rendimiento que la de \compresschain.
\end{itemize}

\section{Configuración}
%
La evaluación fue ejecutada sobre un servidor con 2 procesadores CPU Intel Xeon de 3GHz
con 36 núcleos y 256GB de RAM, corriendo Ubuntu 18.04 Linux64.

%
Todas las implementaciones fueron escritas en Golang y residen en un único ejecutable, el cual se encarga
de inicializar los siguiente servicios.
\begin{itemize}
	\item Un nodo de Tendermint adjunto a una versión específica de la aplicación (la cual impelementa los métodos
	de la ABCI). Las versiones disponibles de la aplicación son: \vanilla, \compresschain, y \hashchain. Todas estas
	versiones necesitan de una base de datos persistente para alojar la \setchain en sí.
	\item Para \compresschain y \hashchain, un servidor \collector RPC. Este servidor se encarga de recibir transacciones,
	armar los lotes adecuados (siguiendo la lógica correspondiente), y difundirlos mediante comunicación con el nodo
	de Tendermint. Por su parte, el \hcollector también requiere una base de datos persistente para mapear hases a lotes
	de elementos.
	\item Un cliente, el cual se encarga de leer e interpretar transacciones, y enviarlas (comunicándose con el nodo de
	Tendermint en la versión \vanilla, o con el \collector correspondiente en los otros casos). Cada cliente se comunica
	con un único nodo servidor.
\end{itemize}
Para las bases de datos persistentes se utilizó \textit{BadgerDB}\cite{db.badger}.
Para las funciones criptográficas se utilizó el paquete crypto de ethereum. 

%
Cada instancia fue empaquetada en un contenedor Docker sin límite de uso en CPU
o RAM.
%
La cantidad de nodos utilizados varió según cada experimento.
%
Fue utilizada la herramienta Docker-Compose para iniciar y correr los contenedores necesarios,
y proveer una red entre ellos de manera sencilla.


\section{Experimentos}
El funcionamiento de los experimentos llevados a cabo sigue la lógica presentada a continuación.
%
En principio, se inician las \textit{N} instancias que correspondan, según la configuración del experimento.
%
Como ya se mencionó, las instancias involucran tanto el nodo de Tendermint, como el \collector (en caso de
ser necesario), y el cliente.
%

Una vez inicializados los \textit{N} contenedores, los clientes comienzan a procesar los elementos asignados a cada
uno y a invocar \<add> para enviarlos.
%
Cada proceso cliente hace las invocaciones \<add> en el servidor que reside en su mismo contenedor. 
%
Esto implica (indirectamente) comunicarse con el nodo de Tendermint en el caso de \vanilla, y
comunicarse con el \collector correspondiente en el caso de \compresschain o \hashchain.

Los experimentos funcionan en rondas.
%
Cada experimento ejecuta X rondas, de Y segundos cada una.
%
Los clientes envían elementos cada Z segundos.
%
Luego de cada ronda, cada cliente envía una petición \<get> para registrar cuántas épocas y cuántos
elementos fueron efectivamente añadidos a la \setchain.
%
Los distintos servidores pueden diferir en la respuesta (ya que algunos pueden encontrarse atrasados
respecto a otros), por lo que se realiza un promedio de estos valores.
%
De este modo se logra estimar la cantidad de elementos añadidos por segundo.

Tendermint posee diversos parámetros que pueden configurarse.
Por ejemplo, el tamaño de la mempool
(en cantidad de transacciones y en cantidad de bytes) o el tamaño del bloque.
El valor de dichos parámetros fue tomado de \cite{tendermint.design}.


\subsection{Elementos}
% También se puede citar Arbitrum Nitro: A Second-Generation Optimistic Rollup
Los elementos utilizados por los procesos clientes durante los experimentos son transacciones provenientes
de Arbitrum~\cite{Kalodner2018Arbitrum}, una tecnología de tipo \textit{Optimistic Rollup}~\footnote{
Arbitrum es una cadena lateral que corre en paralelo a la red principal de Ethereum.
También conocida como una solución de escalabilidad de capa 2 (L2), Arbitrum mejora la velocidad de las transacciones así
como los costos en comparación a la red principal.
Arbitrum fue fundada en 2021 y es al momento una de las tecnologías de capa 2 más populares en el mercado.}.
%

La elección de las transacciones fue tomada priorizando que los experimentos sean fieles a un contexto
real, respetando el contenido de las transacciones.
%
Fueron descargadas utilizando la API de Alchemy\footnote{Disponible en \url{https://www.alchemy.com/}}, la
cual permite conectarse directamente a un nodo específico de Arbitrum que es parte de la red,
de modo de interactuar con los datos en la cadena y hacer las peticiones necesrias para descargar las transacciones.

%
Al analizar el conjunto de transacciones descargadas (alrededor de 123456 transacciones) se encontró que el tamaño de
estas presentaba algunos valores extremos hacia la derecha.
Es decir, mientras que la mediana rondaba los 350 bytes, y la desviación estándar los 600 bytes, los valores mínimos y máximos eran de
110 bytes y  50 mil bytes, respectivamente.
Por tal motivo se decidió remover los valores extremadamente altos.
Siguiendo los valores modelo presentados en \cite{tendermint.design}, se limitó a utilizar transacciones de hasta 1KB de tamaño.
De este modo, analizando aproximadamente 100 mil transacciones provenientes del conjunto de datos usado
durante los experimentos (solo transacciones de hasta 1KB de tamaño), se obtuvieron los siguientes resultados.
\begin{itemize}
	\item Mediana: 340 bytes
	\item Desviación estándar: 220 bytes
	\item Min: 110 bytes
	\item Max: 1024 bytes
\end{itemize}
%

\section{Resultados}



%%% Local Variables:
%%% TeX-master: "article.tex"
%%% TeX-PDF-mode: t
%%% End:
