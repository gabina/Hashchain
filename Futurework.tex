\section{Desarrollo de \hashchain}
Una posible línea de trabajo futuro es la continuación del desarrollo
de \hashchain.
Esto involucra, por un lado, implementar la versión de \hashchain que utiliza
\textit{consolidación de época de primera vista} (ver Sección~\ref{subsubsec:consolidation}),
ya que en este trabajo se implementó únicamente la versión
de \textit{consolidación de época actual}, con la cual se realizó la evaluación
empírica.

%
Otra posibilidad de continuación de \hashchain se da a través del desarrollo de una
\textit{Hashchain pesimista}.
Esta nueva versión cambiaría el comportamiento por defecto del resultado de \<CheckTx>
ante la imposibilidad de correr el chequeo de un hash debido a la ausencia
del lote original.
Actualmente, el comportamiento por defecto es considerar un hash válido.
Una \hashchain pesimista consideraría inválidos a los hashes como conducta predeterminada.
Esto cambiaría el funcionamiento de la \hashchain ya que, por ejemplo, los elementos serían
posiblemente rechazados en un principio, pero una vez consensuados podrían ser agregados
a la \setchain quizás sin necesidad de una estrategia de consolidación.

\section{\textit{Deployments} de \hashchain}
Otro aspecto de posible trabajo futuro incluye la evaluación empírica con
\textit{deployments}
de \hashchain similares a los presentados en \cite{tendermint.design}.
Es decir, utilizando un entorno distribuído geográficamente, con nodos desparramados
uniformemente por distintas regiones en todos los continentes.

De esta forma podrían analizarse cuestiones que quedaron por fuera del alcance de las
evaluaciones realizadas en este trabajo. Por ejemplo, cómo reaccionan las distintas
soluciones a problemas en la red (menos probables de ocurrir cuando se está utilizando
docker compose en una misma máquina), o cómo se ve afectado el rendimiento cuando
los nodos se encuentran a una distancia física considerable. 

\section{Aplicación de \hashchain}
Por último, una dirección de trabajo adicional comprende el desarrollo de las
aplicaciones que motivan la existencia de la \setchain: aquellas que no requieren de
un orden total.
Por ejemplo, la implementación de mempools y sistemas de
\textit{optimistic rollup} de capa 2 a través de Setchains.

A su vez, la \setchain puede ser usada como un mecanismo para mitigar ataques de tipo
\textit{front-running}. La mempool almacena transacciones solicitadas por los usuarios,
por lo que observar
la mempool permite predecir las operaciones futuras, e inyectar maliciosamente transacciones
para que se ejecuten antes que las observadas (pagando una tarifa más alta a los mineros).
Esto es lo que se conoce como ataque de \textit{front-running}.

\setchain puede usarse para detectar ataques de este tipo, dado que puede servir
como un
mecanismo básico para construir una mempool que sea eficiente y sirva como un registro de
solicitudes. Además, las Setchains pueden utilizarse como un bloque de construcción para
evitar los ataques de \textit{front-running}, donde los usuarios cifran sus solicitudes
utilizando un esquema de
cifrado de firma múltiple. Una vez que el orden de las solicitudes fue fijado, los servidores
de desencriptación de los participantes desencriptan
las solicitudes.