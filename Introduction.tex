\chapter{Introduction}
% What is the problem. Justifiy that the problem is a problem.
%Blockchains
Las \textit{bitácoras distribuídas} (también conocidas como blockchains) tomaron
popularidad con la implementación de Bitcoin, propuesta por Nakamoto en
2009~\cite{nakamoto06bitcoin}.
Allí se presentó como un método para eliminar terceras partes confiables en sistemas
de pago electrónico.
%
Las blockchains modernas incorporan contratos inteligentes~\cite{szabo96smart,ethereum},
los cuales son programas de estado inmutable alojados en la blockchain. Dichos
programas describen la funcionalidad de las transacciones, incluyendo el intercambio
de criptomonedas.
%
Los contratos inteligentes permiten describir funcionalidades sofisticadas, habilitando
diversas aplicaciones en finanzas descentralizadas (DeFi)\footnote{En diciembre de 2021,
el valor monetario alojado en DeFi estaba estimado en alrededor de \$100B, de acuerdo a Statista
\url{https://www.statista.com/statistics/1237821/defi-market-size-value-crypto-locked-usd/}.},
gobierno decentralizado, Web3, etc.
%
El elemento principal de todas las  bitácoras distribuídas es la cadena de bloques en donde
se encuentran las transacciones.
%
La blockchain es un \textit{objeto distribuído} que contiene las transacciones realizadas
en nombre de los usuarios, empaquetadas en bloques, y toalmente
ordenadas~\cite{anta2018formalizing,anta2021principles}.
%
En implementaciones de mundo real, el objeto blockchain es mantenido por múltiples servidores
sin una autoridad central, usando \emph{algoritmos de consenso} que son resilientes a los
ataques bizantinos.
%

Un aspecto clave en la adopción de las tecnologías blockchain es \emph{su rendmiento}, medido
en términos del número de transacciones por segundo.
%
Consecuentemente, muchas técnicas para mejorar la escalabilidad de las blockchains
está siendo desarrollada.
%
Las blockchains actuales requieren de algoritmos de consenso que garanticen que las
transacciones, empaquetadas en bloques, estén totalmente ordenadas.
%
Esta imposición de un orden total puede ser innecesario para algunas aplicaciones.


%
Un enfoque prometedor para mejorar la escalabilidad es \textit{Setchain}, un tipo de
datos concurrente que implementa conjuntos de solo crecimiento distribuídos. A su vez,
es tolerante a comportamientos bizantinos, y provee barreras o puntos de sincronización
(llamados épocas).
%
\setchain relaja el requerimiento de orden total y, por lo ranto, puede lograr mayor
rendimiento y escalabilidad.
%
Las \setchains se pueden usar para aquellas aplicaciones, como los registros digitales,
en donde los elementos en la blockchain no necesitan estar ordenados, excepto a través
de barreras ocasionales.

%

Distintos algoritmos bizantinos distribuídos que implementan \setchain fueron
propuestos, pero no existía al momento ninguna implementacion eficiente compatible con
los requerimientos de una aplicación del mundo real.

%% Motivate the use of tendermint. Why tendermint is relevant
Hoy en día, las implementaciones de bitácoras distribuídas evitan el uso de
\emph{pruebas de trabajo}.
%
Las primeras implementaciones de bitácoras distribuídas (Bitcoin~\cite{nakamoto06bitcoin}
y Ethereum~\cite{ethereum}) están basadas en algoritmos de consenso que funcionan con
pruebas de trabajo, enfatizando un problema energérico conocido\cite{energy.problem.bitcoin}.
%
Otros algoritmos de consenso, como aquellos que utilizan \emph{pruebas de apuestas},
fueron propuestos para intentar desligarse de este problema.
%
Sin embargo, existen otros problemas a resolver: la finalización probabilística.
%
Los bloques, una vez que fueron aceptados, requieren rondas de confirmación, lo que da
lugar a la red a decidir otra rama en la evolución de la  bitácora.


Tendermint~\cite{Buchman.2018.Tendermint} es una plataforma novedoza y popular que resuelve
algunos de los problemas técnicos que implementaciones previas de blockchain presentaban,
como el problema energético y la finalización probabilística.
%
Tendermint presenta una separación clara entre las capas de bajo nivel de la blockchain,
tales como un protocolo \textit{gossip} y un algoritmo de consenso, y los conceptos de alto
nivel relacionados a la estructura de datos que la blockchain mantiene.
%
Aún más, Tendermint es una solución robusta a las bitácoras distribuídas implementada
en \emph{Go}~\cite{donovan15go}.

%

%
% What is the solution.
%
En este trabajo se propone una familia de implementaciones robustas de \setchain,
construídas sobre la \emph{plataforma de aplicación de blockchain Tendermint},
siendo la contribución principal \textbf{Hashchain}.
%
Hashchain explota el poder de compresión de las funciones hash para reducir la
comunicación necesaria durante difusión y consenso, comunicando un hash de tamaño
fijo en lugar de cientos o miles de elementos.
%
El precio a pagar es una algoritmo distribuído adicional para obtener el conjunto de
elementos desde un hash, garantizando tolerancia a servidores bizantinos.
%
Se sigue un enfoque incremental, proveyendo diversas aproximaciones a la solución
final y más compleja. Se presentan:
\begin{enumerate}
  \item una primera implementación ingenua de Setchain,
  \item una implementación intermedia utilizando un algoritmo de compresión, y
  \item una solución usando funciones hash.
\end{enumerate}
%
%
Cada una de las implementaciones está construída sobre Tendermint.

% The article is structured as follows: in Section~\ref{sec:prelim}, we present a
% brief description of Setchain and Tendermint; in Section~\ref{sec:impl}, we
% describe three implementations of Setchain on top of Tendermint; and in
% Section~\ref{sec:conclusion}, we conclude our work.
% %
% This article is presented as a work-in-progress and shows partial results, and
% thus, the experimental section is lacking.

% We prove how Hashchain can implement Setchain correctly.

%%% Local Variables:
%%% TeX-master: "article.tex"
%%% TeX-PDF-mode: t
%%% End:
