\section{Introduction}
% What is the problem. Justifiy that the problem is a problem.
%Blockchains
\emph{Distributed ledgers} (also known as \emph{blockchains}) were first proposed
by Nakamoto in 2009~\cite{nakamoto06bitcoin} in the implementation of Bitcoin,
as a method to eliminate trustable third parties in electronic payment systems.
%
Modern blockchains incorporate smart
contracts~\cite{szabo96smart,ethereum}, 
which are immutable state-full
programs stored in the blockchain that describe functionality of transactions,
including the exchange of cryptocurrency.
%
Smart contracts allow to describe sophisticated functionality, enabling
many applications in decentralized finances (DeFi)\footnote{As of December 2021, the monetary value locked in DeFi
  was estimated to be around \$100B, according to Statista
  \url{https://www.statista.com/statistics/1237821/defi-market-size-value-crypto-locked-usd/}.},
  decentralized governance,
Web3, etc.

The main element of all distributed ledgers is the ``blockchain''.
%
The blockchain is a \emph{distributed object} containing, packed in blocks, the
totally ordered list of transactions performed on behalf of the
users~\cite{anta2018formalizing,anta2021principles}.
%
In real-world implementations, the blockchain object is maintained by multiple
servers without a central authority, using \emph{consensus algorithms} that are
resilient to Byzantine attacks.

%
A key aspect of the adoption of blockchain technologies is \emph{their
performance}~\footnote{Mesuare in terms of the number of transactions per
second.}.
%
Consequently, many techniques to improve the scalability of
blockchains are being developed.
%
Current blockchains require consensus algorithms to guarantee that
transactions, batched as blocks, are totally ordered.
%
Imposing a total order, although it is safe, may be unnecessary for some
applications.
%
A promising approach to improve scalability is \emph{Setchain}, a distributed
concurrent data type that implements Byzantine tolerant distributed grown-only
sets with barriers (called epochs).
%
Setchain relaxes the total order requirement, and thus, can achieve higher
throughput and scalability.
%
Setchains can be used for those applications, like digital
registries, where different elements in the blockchain need not be
ordered except across infrequent barriers.
%
Several Byzantine distributed algorithms that implement Setchain
have been proposed but no efficient real-world implementations
exists.

%% Motivate the use of tendermint. Why tendermint is relevant
Nowadays, implementations of distributed ledgers avoid the use of
\emph{proof-of-work}.
%
The first distributed ledges implementations (Bitcoin~\cite{nakamoto06bitcoin}
and Ethereum~\cite{ethereum}) are based on proof-of-work consensus algorithms
\gabina{Revisar referencia a Ethereum Yellow paper porque parece repetida}
leading to a known energetic problem~\cite{energy.problem.bitcoin}.
\martin{Cite energetic problem}
\marga{puse un paper que calcula la huella de carbon de bitcoin en 2018}
%
Other consensus algorithms, as \emph{proof-of-stake}, were proposed trying to
solve this problem.
%
However, after solving the energetic problem, there was another unsolved
problem: probabilistic finality.
%
Blocks, after being accepted, require confirmation rounds leaving room for the
network to decide on another branch in the evolution of the ledger.
%
\martin{This is a known problem in bitcoin, but I don't remember the name}

Tendermint~\cite{Buchman.2018.Tendermint} is a novel and popular
platform that solves some of the technical problems of previous blockchain implementations,
i.e energetic problem and probabilistic finality.
%
Tendermint presents a clear separation between the low-level basic blockchain
blocks, such as a gossip protocol and a consensus algorithm, and high-level
concepts related to the data structure the blockchain maintains.
%
Moreover, Tendermint is a real-world distributed ledger solution implemented in
\emph{Go}~\cite{donovan15go}.
%

%
% What is the solution.
%
In this work in progress, we propose a family
of real-world \setchain implementations built on top of the \emph{Tendermint
blockchain application platform}, with our primary contribution being \textbf{Hashchain}.
%
Hashchain exploits the compression
power of hash functions to reduce the communication necessary during broadcasts
and consensus, communicating a fixed-sized hash instead of hundreds or thousands of
elements.
%
The prize to pay is an additional distributed algorithm to obtain
the set of elements from a hash, guaranteeing tolerance against
Byzantine servers.
%
We follow an incremental approach providing several approximations to the
final and most complex solution. We present:
%
\begin{enumerate}
        \item a naive implementation of Setchain,
        \item an intermediate implementation using a compression algorithm, and
        \item a solution using hash functions.
\end{enumerate}
%
Each implementation is based on top of Tendermint.

The article is structured as follows: in Section~\ref{sec:prelim}, we present a
brief description of Setchain and Tendermint; in Section~\ref{sec:impl}, we
describe three implementations of Setchain on top of Tendermint; and in
Section~\ref{sec:conclusion}, we conclude our work.
%
This article is presented as a work-in-progress and shows partial results, and
thus, the experimental section is lacking.

% We prove how Hashchain can implement Setchain correctly.

%%% Local Variables:
%%% TeX-master: "article.tex"
%%% TeX-PDF-mode: t
%%% End:
