%\section{Resumen y conclusión final}\label{sec:conclusion}
%
\section{Conclusiones}
Las tecnologías blockchain están atravesando un problema de escalabilidad.
%
Comparado a otros sistemas basasdos en transacciones, las blockchains son simplemente
demasiado lentas.
%
Existen, por lo menos, dos formas posibles de resolver este problema: o bien
diseñando algoritmos de consenso más rápidos, o bien colocando soluciones por
encima de la blockchain.

%
Un ejemplo de lo último es lo que se conoce como soluciones de \emph{Layer-2}
(capa 2),
donde se tiene una máquina de computación sobre la blockchain que consume menos
recursos que si se actuara directamente sobre ella.
%
Un ejemplo de lo primero es \setchain.

\setchain presenta un algoritmo de consenso más rápido
%basado en
%\textit{set-consensus}\cite{redbelly},
que abandona el clásico orden total de las transacciones en las blockchains.
%
El orden parcial de \setchain establece que no es posible comparar los elementos
en una misma época.
%
Sin embargo, se puede decir que los elementos de una época ocurrieron antes o después
que los elementos de otra.
%
Esto es, \setchain propone un orden relativo, en donde los elementos
de distintas épocas se pueden relacionar, pero no es posible relacionar elementos
en la misma época.

Aunque \setchain es una idea inteligente, carecía al momento de una implementación
de mundo real.
%
Los autores de \setchain dieron una implementación a modo de prueba de concepto,
que les permitió observar que, en condiciones de laboratorio,
era más rápido que otros algoritmos de consenso.
%

% Implementing and verifying such claims are the goals of this work in progress.
La implementación de sistemas distribuidos es difícil, por lo que en este trabajo
se decidió utilizar una solución prefabricada: Tendermint.
Tendermint ofrece una implementación
de mundo real de una estructura basada en blockchain, presentando una separación clara
entre las primitivas de la blockchain y la interfaz definida por el usuario.


Como contribuciones originales de este trabajo se presentaron tres alternativas posibles para la implementación
de \setchain construidas sobre Tendermint. Dado que los algoritmos de \setchain
presentados en \cite{Capretto.2022.Setchain} emplean bloques primitivos como
\emph{Set Byzantine Consensus}
no es posible replicar los mismos comportamientos.
Sin embargo, tomamos un enfoque ligeramente diferente,
intentando obtener tanto como se pueda de lo que Tendermint es capaz de ofrecer.
Cada alternativa presentada trata de acercarse a la implementación ideal de \setchain,
explotando la idea de que cada época es un conjunto de elementos.

%
La primera solución (ver Sección~\ref{sec:vanilla}) presenta una simple y
correcta pero ineficiente implementación de \setchain donde Tendermint solo mantiene
un conjunto de solo crecimiento.

%
La segunda solución (ver Sección~\ref{sec:compresschain}) mejora la alternativa anterior
usando un algoritmo de compresión: las transacciones de Tendermint ahora son conjuntos
comprimidos de elementos.

%
Finalmente, la tercera solución (ver Sección~\ref{sec:hashchain}) va más allá de la
segunda alternativa y usa funciones hash para reducir la sobrecarga en la red.
Aunque reduzca el tráfico haciendo consenso sobre transacciones pequeñas
y de tamaño fijo (es decir, sobre hashes), el uso de los mismos requiere de una forma de
invertirlos para así obtener los elementos originales enviados por los clientes.
Para resovler esto, se diseñó
un algoritmo distribuido, tolerante a fallas bizantinas, que trabaja como un objeto
distribuido de resolución de hashes.
De este modo se conforma una implementación de \setchain de mundo real.

%
% \hashchain es la solución más cercana a \setchain sin usar un algoritmo de set-consensus.
% Dado que los nodos no conocen los elementos dentro de un hash \emph{a priori}, el
% consenso consiste en elegir cuál época irá próxima.
%

% No entiendo a qué va esta idea:
%However, since in Tendermint we have a mempool and a gossip network, elements
%are gossiped around, and hashes have the biggest set of elements known to the
%node proposing the next epoch.

Los números provistos por la evaluación empírica son prometedores, indicando que,
efectivamente, \compresschain permite un rendimiento mejor que \vanilla y, a su vez,
\hashchain presenta un rendimiento superior a \compresschain.
Estos resultados se presentan a favor de la hipótesis de que hacer consenso sobre
elementos agrupados de forma eficiente interpretados como unidad representa una
mejora en comparación con el consenso usual, hecho sobre elementos individuales. 

\section{Trabajo futuro}

\subsection{Desarrollo de \hashchain}
Una posible línea de trabajo futuro es la continuación del desarrollo
de \hashchain.
Esto se puede abordar, por un lado, implementando la versión de \hashchain que utiliza
\textit{consolidación de época de primera vista} (ver Sección~\ref{subsubsec:consolidation}),
ya que en este trabajo se implementó únicamente la versión
de \textit{consolidación de época actual}, con la cual se realizó la evaluación
empírica.

%
Otra posibilidad de continuación de \hashchain se da a través del desarrollo de una
\textit{Hashchain pesimista}.
Esta nueva versión cambiaría el comportamiento por defecto del resultado de \<CheckTx>
ante la imposibilidad de correr el chequeo de un hash debido a la ausencia
del lote original.
Actualmente, el comportamiento por defecto es considerar a un hash como válido.
Una \hashchain pesimista consideraría inválidos a los hashes como conducta predeterminada.
Esto cambiaría el funcionamiento de la \hashchain ya que, por ejemplo, los elementos serían
posiblemente rechazados en un principio, pero una vez consensuados podrían ser agregados
a la \setchain quizás sin necesidad de una estrategia de consolidación.

%
Por último, en la sección \ref{subsubsec:vanilla-implementantion} se presentó 
a la estrategia de almacenamiento de elementos y épocas en la base de datos
como un aspecto con múltiples abordajes: dependiendo del caso de uso de \setchain
se opta por una forma u otra,
con el objetivo de mantener eficientes las búsquedas.
%
Una perspectiva de desarrollo interesante es ir incluso más allá de la definición primitiva de \<get>
(que se limita a retornar la terna $\THESET, \HISTORY, \EPOCH$) y tener una \hashchain que provea
a los usuarios diversos métodos de hacer consultas simples sobre el contenido de la \setchain.
%
Ejemplos de estos nuevos métodos son:
\begin{itemize}
    \item \textit{find\_element}: dado un elemento retorna un booleano indicando si el elemento
    pertenece o no a alguna época, junto con una prueba de ello.
    \item \textit{get\_epoch}: dado un número de época, retorna los elementos de la misma.
\end{itemize}
%
La idea principal es poder desarrollar métodos de búsqueda \emph{confiables}, es decir, donde además
de retornar el objeto pedido, se presenta una prueba de pertenencia que el cliente puede verificar
sencillamente mediante el uso de \emph{estructuras de datos autenticadas}~\cite{authenticated-data-struct}.
%


\subsection{\textit{Deployments} de \hashchain}
Otro aspecto de posible trabajo futuro incluye la evaluación empírica con
\textit{deployments}
de \hashchain similares a los presentados en \cite{tendermint.design}.
Es decir, utilizando un entorno distribuido geográficamente, con nodos desparramados
uniformemente por distintas regiones en todos los continentes.

De esta forma podrían analizarse cuestiones que quedaron por fuera del alcance de las
evaluaciones realizadas en este trabajo. Por ejemplo, cómo reaccionan las distintas
soluciones a problemas en la red (menos probables de ocurrir cuando se está utilizando
docker compose en una misma máquina), o cómo se ve afectado el rendimiento cuando
los nodos se encuentran a una distancia física considerable. 

\subsection{Aplicación de \hashchain}
Una dirección de trabajo adicional comprende el desarrollo de las
aplicaciones que motivan la existencia de la \setchain: aquellas que no requieren de
un orden total.
Por ejemplo, la implementación de mempools y sistemas de
\textit{optimistic rollup} de capa 2 a través de Setchains.

A su vez, la \setchain puede ser usada como un mecanismo para mitigar ataques de tipo
\textit{front-running}. La mempool almacena transacciones solicitadas por los usuarios,
por lo que observar
la mempool permite predecir las operaciones futuras. \emph{Front-running} es la acción
de inyectar maliciosamente transacciones
para que se ejecuten antes que las observadas~\cite{Daian2020FlashBoys, Ferreira2021Frontrunner}
(pagando una tarifa más alta a los mineros).
%
\setchain puede usarse para detectar ataques de este tipo, dado que puede servir
como un
mecanismo básico para construir una mempool que sea eficiente y sirva como un registro de
solicitudes.
%
Además, las Setchains pueden utilizarse como un bloque de construcción para
evitar los ataques de \textit{front-running}, donde los usuarios cifran sus solicitudes
utilizando un esquema de
cifrado de firma múltiple.
%
Una vez que el orden de las solicitudes fue fijado, los servidores
de desencriptación de los participantes desencriptan
las solicitudes.

%
Por último, estudiar cómo equipar las blockchains existentes con una \setchain (sincronizando bloques
y épocas) para permitir a los contratos inteligentes acceder a la \setchain como parte de su almacenamiento
es otra ruta de trabajo futura.