\documentclass{book}


\pdfobjcompresslevel0
\usepackage{fancyvrb}
\usepackage{amssymb}
\usepackage{amsmath}
\usepackage{algorithm}
\usepackage{algpseudocode}
\usepackage{cesar-llncs}
%
% T_{fSeq} to T_{path} Auxiliary Functions
%
\newcommand{\TPathFig}{
\renewcommand{\arraystretch}{1}
\centering
\begin{tabular}{|c|}
\hline
$\fApp : \sSeq \times \sSeq \to \sSeq$ \\
\hline
$\fApp (\fNil, l) = l$ \\
$\fApp(\fCons(a,l), l') = \fCons(a, \fApp(l, l'))$ \\
\hline
\hline
$\fSeqToSet : \sSeq \to \sSet$ \\
\hline
$\fSeqToSet (\fNil) = \emptyset$ \\
$\fSeqToSet (\fCons(a,l)) = \{ a \} \cup \fSeqToSet(l)$ \\
\hline
\hline
$\fIsPath : \sSeq$ \\
\hline
$\fIsPath(\fNil)$ \\
$\fIsPath(\fCons(a, \fNil))$ \\
$\{a\} \nsubseteq \fSeqToSet(l) \land \fIsPath(l) \Impl
	\fIsPath(\fCons(a,l))$ \\
\hline
\hline
$\fLast : \sSeq \to \sAddr$ \\
\hline
$\fLast(\fCons(a, \fNil)) = a$ \\
$l \neq \fNil \Impl \fLast(\fCons(a, l)) = \fLast(l)$ \\
\hline
\hline
$\pIsReachK : \sMem \times \sAddr \times \sAddr \times \sLevelK$ \\
\hline
$\pIsReachK(m, a, a, l)$ \\
$m[a].\fNext[l] = a' \land \pIsReach(m, a', b, l)
	\Impl \pIsReach(m, a, b, l)$ \\
\hline
\hline
$\pIsReachKP : \sMem \times \sAddr \times \sAddr \times \sLevelK
	\times \sSeq$ \\
\hline
$\pIsReachKP(m, a, a, l, \fNil)$ \\
$m[a].\fNext[l] = a' \land \pIsReachP(m, a', b, l, p) \Impl
	\pIsReachP(m, a, b, l, \fCons(a, p))$ \\
\hline
\hline
$\fFirstMarked : \sMem \times \sSeq \times \sLevelK \times \sAddr$ \\
\hline
$\fFirstMarked (m, \fNil, l, \fNull)$ \\
$p \neq \fNil \land p = \fCons(j, q) \land m[j].\fLockID[l] \neq \oslash
	\Impl \fFirstMarked(m, p, l, j)$ \\
$p \neq \fNil \land p = \fCons(j, q) \land m[j].\fLockID[l] = \oslash
	\land \fFirstMarked(m, q, l, i)
		\Impl \fFirstMarked(m, p, l, i)$ \\
\hline
\hline
$\pOrdPath : \sMem \times \sSeq$ \\
\hline
$\pOrdPath(h, \fNil)$ \\
$h[a].\fNext[\fLevel{0}] = a' \land h[a].\fKey \pOrd h[a'].\fKey \land
	p = \fCons(a, q) \land \pOrdPath(h, q) \Impl \pOrdPath(h, p)$ \\
\hline
\end{tabular}
}


%
% TLLpL Auxiliary Functions
%
\newcommand{\AuxiliaryFunctions}{
\centering
\begin{tabular}{|c|}
\hline
$\List : \sMem \times \sAddr \times \sSet$ \\
\hline
$\begin{array}{lcl}
	\List(h,a,r) & \Iff & \fNull \in \fAddrToSet(h, a)
		\land r = \fPathToSet(\fGetp(h, a, \fNull)) \\
\end{array}$\\
\hline
\hline
$\fa : \sMem \times \sAddr \to \sPath$ \\
\hline
$\fa(h,n) = \begin{cases}
		\fEpsilon & \trm{if } n = \fNull \\
		\fGetp(h, h[n].\fNext, \fNull) & \trm{if } n \neq \fNull \\
		\end{cases}$ \\
\hline
%
% Begin: Function removed to reduce space
%
%\hline
%$\fb : \sMem \times \sAddr \times \sAddr \times \sPath$ \\
%\hline
%$\begin{array}{lcl}
%\fb(h, n, m, p) & \Iff & (n = \fNull \land p = \fEpsilon)
%	\lor \; \big(n \neq \fNull \land \pReach(h,n,m,\tilde{p})\\
%		&& \hs{3.5} \land \; m = \fNull
%			\Impl \pAppend(p, \fNull, \tilde{p}) \\
%		&& \hs{3.5} \land \; m \neq \fNull
%			\Impl \pAppend(p, [m], \tilde{p}) \big) \\
%\end{array}$ \\
%\hline
\hline
$\LastMarked : \sMem \times \sPath \to \sAddr$ \\
\hline
$\LastMarked(m, p) = \fFirstLocked(m, \fRev(p))$ \\
\hline
\hline
$\NoMarks : \sMem \times \sPath$ \\
\hline
$\NoMarks(m,p) \Iff \fFirstLocked(m,p) = \fNull$ \\
\hline
\hline
$\SomeMark : \sMem \times \sPath$ \\
\hline
$\SomeMark(m,p) \Iff \fFirstLocked(m,p) \neq \fNull$ \\
\hline
\end{tabular}
}



%
% TSLK Signature
%
\newcommand{\TSLKSignature}{
\centering
\begin{tabular}{|c|c|c|c|}
\hline
Signature & \hs{0.1} Sorts \hs{0.1} & Functions & Predicates \\
\hline
\hline
\sigLevelK &
	$\begin{array}{c}
		\sLevelK
	\end{array}$ &
	$\begin{array}{lcl}
		0,1, \ldots, \K-1 & : & \sLevelK \\
	\end{array}$ &
	$\begin{array}{lcl}
		< : \sLevelK \times \sLevelK
	\end{array}$\\
\hline
\sigOrd &
	$\begin{array}{c}
		\sOrd
	\end{array}$ &
	$\begin{array}{lcl}
		\fZero & : & \sOrd \\
		\fInfty & : & \sOrd \\
	\end{array}$ &
	$\begin{array}{lcl}
		\pOrd & : & \sOrd \times \sOrd \\
	\end{array}$ \\
\hline
\sigThId &
	$\begin{array}{c}
		\sThId
	\end{array}$ &
	$\begin{array}{lcl}
		\oslash & : & \sThId \\
	\end{array}$ & \\
\hline
\hline
\sigCells &
	$\begin{array}{c}
		\sCell \\ \sElem \\ \sOrd \\ \sAddr \\ \sThId
	\end{array}$ &
	$\begin{array}{lcl}
		\fError & : & \sCell \\
		\fMkcell & : & \sElem \times \sOrd \times \sAddr^{\K}
			\times \sThId^{\K} \to \sCell \\
		\_.\fData & : & \sCell \to \sElem \\
		\_.\fKey & : & \sCell \to \sOrd \\
		\_.\fNext \lbrack \_ \rbrack & : & \sCell \times \sLevelK
			\to \sAddr \\
		\_.\fLockID \lbrack \_ \rbrack & : & \sCell
			\times \sLevelK \to \sThId \\
		\_.\fLock \lbrack \_ \rbrack & : & \sCell \times \sLevelK	
			\to \sThId \to \sCell \\
		\_.\fUnlock \lbrack \_ \rbrack & : & \sCell
			\times \sLevelK \to \sCell \\
	\end{array}$ & \\
\hline
\sigMemory &
	$\begin{array}{c}
		\sMem \\ \sAddr \\ \sCell
	\end{array}$ &
	$\begin{array}{lcl}
		\fNull & : & \sAddr \\
		\_[\_] & : & \sMem \times \sAddr \to \sCell \\
		\fUpd & : & \sMem \times \sAddr \times \sCell \to \sMem
	\end{array}$ & \\
\hline
\sigReach &
	$\begin{array}{c}
		\sMem \\ \sAddr \\ \sPath
	\end{array}$ &
	$\begin{array}{lcl}
		\fEpsilon & : & \sPath \\
		\lbrack \_ \rbrack & : & \sAddr \to \sPath
	\end{array}$ &
	$\begin{array}{lcl}
		\pAppend & : & \sPath \times \sPath \times \sPath \\
		\pReachK & : & \sMem \times \sAddr \times \sAddr \\
				&& \mhs \times \; \sLevelK \times \sPath \\
	\end{array}$ \\
\hline
\hline
\sigSets &
	$\begin{array}{c}
		\sAddr \\ \sSet
	\end{array}$ &
	$\begin{array}{lcl}
		\emptyset & : & \sSet \\
		\{ \_ \} & : & \sAddr \to \sSet \\
		\cup, \cap, \setminus & : & \sSet \times \sSet \to \sSet
	\end{array}$ &
	$\begin{array}{lcl}
		\in & : & \sAddr \times \sSet \\
		\subseteq & : & \sSet \times \sSet
	\end{array}$ \\
\hline
\sigSetsT &
	$\begin{array}{c}
		\sThId \\ \sSetT
	\end{array}$ &
	$\begin{array}{lcl}
		\stEmpty & : & \sSetT \\
		\stSingle{\_} & : & \sThId \to \sSetT \\
		\stCup, \stCap, \stSubst & : & \sSetT \times \sSetT \to \sSetT
	\end{array}$ &
	$\begin{array}{lcl}
		\stIn & : & \sThId \times \sSetT \\
		\stSubset & : & \sSetT \times \sSetT
	\end{array}$ \\
\hline
\sigMrgn &
	$\begin{array}{c}
		\sMrgn, \\ \sAddr \\ \sLevelK
	\end{array}$ &
	$\begin{array}{lcl}
		\mrgnEmpty & : & \sMrgn \\
		\mrgnSingle{\_, \_} & : & \sAddr \times \sLevelK \to \sMrgn \\
		\mrgnCup, \mrgnCap, \mrgnMinus & : & \sMrgn \times \sMrgn
			\to \sMrgn
	\end{array}$ &
	$\begin{array}{lcl}
		\mrgnIn & : & \sAddr \times \sLevelK \times \sMrgn \\
		\mrgnSubset & : & \sMrgn \times \sMrgn \\
		\mrgnDisj & : & \sMrgn \times \sMrgn
	\end{array}$ \\
\hline
\hline
\sigBridge &
	$\begin{array}{c}
		\sMem \\ \sAddr \\ \sSet \\ \sPath
	\end{array}$ &
	$\begin{array}{lcl}
		\fPathToSet & : & \sPath \to \sSet \\
		\fAddrToSetK & : & \sMem \times \sAddr \times \sLevelK
			\to \sSet \\
		\fGetpK & : & \sMem \times \sAddr \times \sAddr \times
			\sLevelK \to \sPath \\
		\fFirstLockedK & : & \sMem \times \sPath \times \sLevelK
			\to \sAddr
	\end{array}$ &
	$\begin{array}{lcl}
		\pOrdList & : & \sMem \times \sPath \\
	\end{array}$ \\
\hline
\end{tabular}
}


%
% TSLK Interpretations
%
\newcommand{\TSLKInterpretationsOne}{
\centering
\begin{tabular}{|c|p{10.5cm}|}
\hline
\multicolumn{2}{|c|}{Interpretation of sort symbols: \sAddr, \sElem, 
\sThId, \sLevelK, \sOrd, \sCell, \sMem, \sPath, \sSet, \sSetT and \sMrgn}
\\
\hline
\hline
\multicolumn{2}{|l|}{
Each sort $\sigma$ in \sigTSLK is mapped to a non-empty set $\Ais{\sigma}$ 
such that:} \\
\multicolumn{2}{|l|}{
\begin{tabular}{clcl}
	(a) &$\Ais{\sAddr}$ and $\Ais{\sElem}$ are discrete sets &
	(b) &$\Ais{\sThId}$ is a discrete set containing symbol $\oslash$ \\
	(c) &$\Ais{\sLevelK}$ is the finite collection 0,\ldots,\K-1 &
	(d) &$\Ais{\sOrd}$ is a collection of total ordered identifiers \\
	(e) &$\Ais{\sCell} = \Ais{\sElem} \times \Ais{\sOrd} \times 
		\Ais{\sAddr}^{\K} \times \Ais{\sThId}^{\K}$ &
	(f) &$\Ais{\sMem} = \Ais{\sCell}^{\Ais{\sAddr}}$ \\
	(g) &$\Ais{\sPath}$ is the set of all finite sequences of &
	(h) &$\Ais{\sSet}$ is the power-set of $\Ais{\sAddr}$ \\
		& \hs{0.5} (pairwise) distinct elements of $\Ais{\sAddr}$ &
	(i) &$\Ais{\sSetT}$ is the power-set of $\Ais{\sThId}$ \\
	(j) &$\Ais{\sMrgn}$ is the power-set of
		$\Ais{\sAddr} \times \Ais{\sLevelK}$ &
\end{tabular}
} \\
\hline
\hline
Signature & Interpretation \\
\hline
%\hline
%\sigLevelK &
%	\begin{tabular}{p{9.7cm}}
%		$\inter{\fLevel{i}}{\Ai} \neq \inter{\fLevel{j}}{\Ai}$
%			when $i \neq j$
%	\end{tabular} \\
\hline
\sigOrd &
	\begin{tabular}{p{9.7cm}}
		$x \inter{\pOrd}{\Ai} y \land y \inter{\pOrd}{\Ai} x
			\Impl x = y$ for any $x, y \in \Ais{\sOrd}$ \\
		$x \inter{\pOrd}{\Ai} y \land y \inter{\pOrd}{\Ai} z
			\Impl x \inter{\pOrd}{\Ai} z$
			for any $x, y, z \in \Ais{\sOrd}$ \\
		$x \inter{\pOrd}{\Ai} y \lor y \inter{\pOrd}{\Ai} x$
			for any $x, y \in \Ais{\sOrd}$ \\
		$\inter{\fZero}{\Ai} \inter{\pOrd}{\Ai} x \land
		x \inter{\pOrd}{\Ai} \inter{\fInfty}{\Ai}$
			for any $x \in \Ais{\sOrd}$ \\
%		$x \neq \inter{\fInfty}{\Ai} \Impl
%			x \inter{\pOrd}{\Ai} \inter{\fSucc}{\Ai}(x)$
%			for any $x \in \Ais{\sOrd}$ \\
	\end{tabular} \\
\hline
\hline
\sigCells &
	\begin{tabular}{p{11.5cm}}
		\begin{compactitem}
		\item $\inter{\fMkcell}{\Ai} (e, k, a_0, \ldots, a_{\K-1},
						t_0, \ldots, t_{\K-1}) =
			\langle e, k, a_0, \ldots, a_{\K-1},
				t_0, \ldots, t_{\K-1} \rangle$
			for each
			$e \in \Ais{\sElem}$, $i \in \Ais{\sOrd}$,
			$a_0, \ldots, a_{\K-1} \in \Ais{\sAddr}$
			and $t_0, \ldots, t_{\K-1} \in \Ais{\sThId}$
		\item $\langle e, k, \vect{a}, \vect{t} \rangle
				.\inter{\fData}{\Ai} = e$
			for each $e \in \Ais{\sElem}$,
			$k \in \Ais{\sOrd}$,
			$\vect{a} \in \Ais{\sAddr}^{\K}$
			and $\vect{t} \in \Ais{\sThId}^{\K}$
		\item $\langle e, k, \vect{a}, \vect{t} \rangle
				.\inter{\fKey}{\Ai} = k$
			for each $e \in \Ais{\sElem}$,
			$k \in \Ais{\sOrd}$,
			$\vect{a} \in \Ais{\sAddr}^{\K}$
			and $\vect{t} \in \Ais{\sThId}^{\K}$
		\item $\langle e, k, a_0, \ldots, a_{\K-1}, \vect{t} \rangle
				.\inter{\fNext}{\Ai}[j] = a_j$
			for each $e \in \Ais{\sElem}$,
			$k \in \Ais{\sOrd}$,
			$a_0 \ldots, a_{\K-1} \in \Ais{\sAddr}$,
			$j \in \Ais{\sLevelK}$
			and $\vect{t} \in \Ais{\sThId}^{\K}$ and
		\item $\langle e, k, \vect{a}, t_0, \ldots, t_{\K-1} \rangle
				.\inter{\fLockID}{\Ai}[j] = t_{j}$
			for each $e \in \Ais{\sElem}$,
			$k \in \Ais{\sOrd}$,
			$\vect{a} \in \Ais{\sAddr}^{\K}$,
			$j \in \Ais{\sLevelK}$
			and $t_0, \dots, t_{\K-1} \in \Ais{\sThId}$
		\item $\langle e, k, \vect{a}, t_0, \ldots, t_{\K-1} \rangle
				.\inter{\fLock}{\Ai}[j](t') =
			\langle e, k, \vect{a},
				t_0, \ldots, t_{j-1}, t',
				t_{j+1}, \ldots, t_{\K-1}
			\rangle$
			for each $e \in \Ais{\sElem}$,
			$k \in \Ais{\sOrd}$,
			$\vect{a} \in \Ais{\sAddr}^{\K}$,
			$j \in \Ais{\sLevelK}$
			and $t_0, \ldots, t_{\K-1}, t' \in \Ais{\sThId}$
		\item $\langle e, k, \vect{a}, t_0, \ldots, t_{\K-1} \rangle
				.\inter{\fUnlock}{\Ai}[j] =
			\langle e, k, \vect{a},
				t_0, \ldots, t_{j-1}, \oslash,
				t_{j+1}, \ldots, t_{\K-1}
			\rangle$
			for each $e \in \Ais{\sElem}$,
			$k \in \Ais{\sOrd}$,
			$\vect{a} \in \Ais{\sAddr}^{\K}$,
			$j \in \Ais{\sLevelK}$
			and $t_0, \ldots, t_{\K-1} \in \Ais{\sThId}$
		\item $\inter{\fError}{\Ai}.\inter{\fNext}{\Ai} =
			\inter{\fNull}{\Ai}$
		\end{compactitem}
	\end{tabular} \\
\hline
\sigMemory &
	\begin{tabular}{p{9.7cm}}
		\begin{compactitem}
		\item $m\inter{[a]}{\Ai} = m(a)$ for each $m \in \Ais{\sMem}$
			and $a \in \Ais{\sAddr}$
		\item $\inter{\fUpd}{\Ai}(m,a,c) = m_{a \mapsto c}$ for each
			$m \in \Ais{\sMem}$, $a \in \Ais{\sAddr}$ and
			$c \in \Ais{\sCell}$
		\item $\inter{m}{\Ai}(\inter{\fNull}{\Ai}) = 
			\inter{\fError}{\Ai}$
			for each $m \in \Ais{\sMem}$
		\end{compactitem}
	\end{tabular} \\
\hline
\sigReach &
	\begin{tabular}{p{11cm}}
		\begin{compactitem}
		\item $\inter{\fEpsilon}{\Ai}$ is the empty sequence
		\item $\inter{[i]}{\Ai}$ is the sequence containing $i \in 
			\Ais{\sAddr}$ as the only element
		\item $\left(\left[i_1,\ldots, i_n\right], \left[j_1, \ldots, 
			j_m\right], \left[i_1,\ldots, i_n, j_1,
				\ldots, j_m\right]
			\right) \in \inter{\pAppend}{\Ai}$ iff $i_k$
		and $j_l$ are all distinct
		\item $(m, \aInit, \aEnd, l, p) \in \inter{\pReachK}{\Ai}$
			iff $\aInit=\aEnd$ and $p = \fEpsilon$, or
		there exist addresses $a_1, \ldots, a_n \in \Ais{\sAddr}$ such 
		that:
		\end{compactitem} \\
		\hs{1.5} $\begin{array}{ll}
			\trm{(a) } p = [a_1, \ldots, a_n]  \phantom{aaaa}&
			\trm{(c) } m(a_r).\inter{\fNext}{\Ai}[l]
				= a_{r+1}, \trm{for } 1 \leq r < n \\
			\trm{(b) } a_1 = \aInit &
			\trm{(d) } m(a_n).\inter{\fNext}{\Ai}[l]
				= \aEnd
		\end{array}$
	\end{tabular} \\
\hline
\hline
\sigSets &
	\begin{tabular}{p{11cm}}
		The symbols $\emptyset$, $\{ \_ \}$, $\cup$, $\cap$, 
		$\setminus$, $\in$ and $\subseteq$ are interpreted 
according to
		their standard interpretation over sets of addresses.
	\end{tabular} \\
\hline
\sigSetsT &
	\begin{tabular}{p{11cm}}
		The symbols $\stEmpty$, $\stSingle{\_}$, $\stCup$, $\stCap$, 
		$\stSubst$, $\stIn$ and $\stSubset$ are interpreted according
		to their standard interpretation over sets of thread
		identifiers.
	\end{tabular} \\
\hline
\end{tabular}
}


\newcommand{\TSLKInterpretationsTwo}{
\centering
\begin{tabular}{|c|p{11.7cm}|}
\hline
Signature & Interpretation \\
\hline
\hline
\sigMrgn &
	\begin{tabular}{p{11cm}}
		\begin{compactitem}
		\item $\mrgnEmptyI{\Ai} = \emptyset$
		\item $\mrgnSingleI{a,j}{\Ai} = \{ (a,j) \}$
			for each $a \in \Ais{\sAddr}$
			and $j \in \Ais{\sLevelK}$
		\item $m_1 \mrgnCupI{\Ai} m_2 = m_1 \cup m_2$
			for each $m_1, m_2 \in \Ais{\sMrgn}$
		\item $m_1 \mrgnCapI{\Ai} m_2 = m_1 \cap m_2$
			for each $m_1, m_2 \in \Ais{\sMrgn}$
		\item $m_1 \mrgnMinusI{\Ai} m_2 = m_1 \setminus m_2$
			for each $m_1, m_2 \in \Ais{\sMrgn}$
		\item $m_1 \mrgnSubsetI{\Ai} m_2 \Iff m_1 \subseteq m_2$
			for each $m_1, m_2 \in \Ais{\sMrgn}$
		\item $(a,j) \mrgnInI{\Ai} m \Iff (a,j) \in m$
			for each $a \in \Ais{\sAddr}$,
			$j \in \Ais{\sLevelK}$ and $m \in \Ais{\sMrgn}$
		\item $m_1 \mrgnDisjI{\Ai} m_2 \Iff m_1 \mrgnCapI{\Ai} m_2
			= \mrgnEmptyI{\Ai}$
			for each $m_1, m_2 \in \Ais{\sMrgn}$
		\end{compactitem}
	\end{tabular} \\
\hline
\hline
\sigBridge &
	\begin{tabular}{p{11cm}}
		\begin{compactitem}
		\item $\inter{\fPathToSet}{\Ai}(p) = \{ a_1, \ldots, 
			a_n\}$ for $p = [a_1, \ldots, a_n] \in
			\Ais{\sPath}$
		\item $\inter{\fAddrToSetK}{\Ai}(m, a, l) = \big\{
			a' \in \Ais{\sAddr} \:|\: \exists p \in 
			\Ais{\sPath}$ s.t. $(m,a,a',l,p) \in \pReachK
			\big\}$
		\item $\inter{\fGetpK}{\Ai}(m,\aInit,\aEnd,l) =
			\begin{cases}
				p & \text{if } (m,\aInit,\aEnd,l,p)
					\in \inter{\pReachK}{\Ai} \\
				\fEpsilon & \text{otherwise}
			\end{cases}$
		\end{compactitem} \\
			\hspace{1cm} for each $m \in \Ais{\sMem}$,
			$p \in \Ais{\sPath}$, $l \in \Ais{\sLevelK}$
			and $\aInit, \aEnd \in \Ais{\sAddr}$
		\begin{compactitem}
		\item $\inter{\fFirstLocked}{\Ai}\left(m,[a_1, \ldots,
			a_n], l \right) =\left\{
			\begin{array}{ll}
			a_k &
			\trm{if there is } 1 \leq k \leq n \trm{ such that } \\
			& \trm{for all } 1 \leq j < k,
				m[a_j].\fLockID[l] = \oslash \\
			& \trm{and } m[a_k].\fLockID[l] \neq \oslash \\
			\fNull & \trm{otherwise }
			\end{array}
			\right.$
		\item $\inter{\pOrdList}{\Ai}\left(m,p \right)$
			iff $p=\fEpsilon$ or $p = \lbrack a \rbrack$ or
			$p = [a_1, \ldots, a_n]$ with $n \geq 2$ and
			$m(a_i).\inter{\key}{\Ai} \pOrd
				m(a_{i+1}).\inter{\key}{\Ai}$
			for all $1 \leq i < n$,
			for any $m \in \Ais{\sMem}$
		\end{compactitem}
	\end{tabular} \\
\hline
\end{tabular}
}

\usepackage{xspace}
\usepackage{amssymb}
%\newcommand{\val}[1]{\ensuremath{[\![ #1 ]\!]}}
\newcommand{\der}[1]{\ensuremath{{\mathop{{ %
            \longrightarrow}}\limits^{{#1}}}\!}} %
\newcommand \tpl[1]{\langle #1 \rangle}
%\newcommand{\RE}{\textsf{RE}\xspace}
\newcommand{\NVPA}{\textsf{NVPA}\xspace}
\newcommand{\VPA}{\textsf{VPA}\xspace}
\newcommand{\NFA}{\textsf{NFA}\xspace}
\newcommand{\VPL}{\textsf{VPL}\xspace}
\newcommand{\CARET}{\textsf{CaRet}\xspace}
\newcommand{\VPLs}{\textsf{VPL}s\xspace}
\newcommand{\VRL}{\textsf{VRL}\xspace}
\newcommand{\VRE}{\textsf{VRE}\xspace}
\newcommand{\EMSO}{\textsf{MSO$_{\mu}$}\xspace}
\newcommand{\MSO}{\textsf{MSO}\xspace}
\newcommand{\XML}{\textsf{XML}\xspace}

\newcommand{\EXPTIME}{{\sffamily EXPTIME}\xspace}

\newcommand{\WM}{\ensuremath{\mathit{WM}}\xspace}
\newcommand{\MR}{\ensuremath{\mathit{MR}}\xspace}
\newcommand{\MC}{\ensuremath{\mathit{MC}}\xspace}
\newcommand{\SWM}{\ensuremath{\mathit{MWM}}\xspace}
\newcommand{\Pu}{\ensuremath{\mathcal{P}}\xspace}
%\newcommand{\Lang}{\ensuremath{\mathcal{L}}\xspace}
\newcommand{\Au}{\ensuremath{\mathcal{A}}}
\newcommand{\Msubst}{$M$-substitution\xspace}
\newcommand{\Mclosure}{$M$-closure\xspace}
\newcommand{\Sclosure}{$S$-closure\xspace}


\newcommand{\Stilde}{\widetilde{\Sigma}\xspace}
\newcommand{\Scall}{\Sigma_{\mathit{call}}\xspace}
\newcommand{\Sret}{\Sigma_{\mathit{ret}}\xspace}
\newcommand{\Sint}{\Sigma_{\mathit{int}}\xspace}

\newcommand{\SubstOp}[1]{\mathrel{\curvearrowleft_{#1}}}
\newcommand{\SubstBox}{\SubstOp{\Box}}
\newcommand{\MClos}[2]{#1^{\SubstOp{#2}}}
\newcommand{\LangClos}{\MClos{\Lang}{\Box}}
\newcommand{\SClos}[2]{{#1}^{\circlearrowleft_{#2}}}
\newcommand{\SClosure}[1]{\SClos{#1}{\Box}}
\newcommand{\LangSClos}{\SClos{\Lang}{\Box}}



%\newcommand{\DefinedAs}{\,\stackrel{\text{def}}{=}\,}

%\newcommand{\DefOR}{\ensuremath{\;\;\;\big|\;\;\;}}
\newcommand{\DefORmini}{\ensuremath{\;\;\big|\;\;}}

\newcommand{\details}[1]{}

\newcommand{\Outcm}{c\,b\,r}
\newcommand{\UOutcm}{\underline{\Outcm}}

\newcommand{\LSet}{R}
\newcommand{\SSet}{S}

\newcommand{\qin}{\ensuremath{q_{\textit{in}}}}
%%%% LTL Symbols %%%%%
\makeatletter
\DeclareFontFamily{OMX}{MnSymbolE}{}
\DeclareSymbolFont{MnLargeSymbols}{OMX}{MnSymbolE}{m}{n}
\SetSymbolFont{MnLargeSymbols}{bold}{OMX}{MnSymbolE}{b}{n}
\DeclareFontShape{OMX}{MnSymbolE}{m}{n}{
    <-6>  MnSymbolE5
   <6-7>  MnSymbolE6
   <7-8>  MnSymbolE7
   <8-9>  MnSymbolE8
   <9-10> MnSymbolE9
  <10-12> MnSymbolE10
  <12->   MnSymbolE12
}{}
\DeclareFontShape{OMX}{MnSymbolE}{b}{n}{
    <-6>  MnSymbolE-Bold5
   <6-7>  MnSymbolE-Bold6
   <7-8>  MnSymbolE-Bold7
   <8-9>  MnSymbolE-Bold8
   <9-10> MnSymbolE-Bold9
  <10-12> MnSymbolE-Bold10
  <12->   MnSymbolE-Bold12
}{}

\let\llangle\@undefined
\let\rrangle\@undefined
\DeclareMathDelimiter{\llangle}{\mathopen}%
                     {MnLargeSymbols}{'164}{MnLargeSymbols}{'164}
\DeclareMathDelimiter{\rrangle}{\mathclose}%
                     {MnLargeSymbols}{'171}{MnLargeSymbols}{'171}
\makeatother

% \newcommand{\mymid}{\mathbin{\mid}}
% \newcommand{\Power}[3]{{#2}\!\mymid\!{#1}\rangle\!\rangle{#3}}
% \newcommand{\WeakPower}[3]{{#2}\mymid\!{#1}\rangle{#3}} %warning: not dual power!!
% \newcommand{\PowerSub}[3]{{#2}\mymid{#1}\rangle\!\rangle{#3}}
% \newcommand{\WeakPowerSub}[3]{{#2}\mymid{#1}\rangle{#3}} %warning: not dual power!!

\newcommand{\Always}{\LTLsquare}
%\newcommand{\Always}{\mathbf{G}}
\newcommand{\Event}{\LTLdiamond}
\newcommand{\Eventually}{\Event}
\newcommand{\Next}{\LTLcircle}
\newcommand{\PrevLTL}{\LTLcircleminus}
\newcommand{\Previously}{\Prev}
\newcommand{\PrevNoFirst}{\LTLcircletilde}
\newcommand{\Before}{\PrevNoFirst}
\newcommand{\HasAlwaysBeen}{\LTLsquareminus}
\newcommand{\SoFar}{\HasAlwaysBeen}
\newcommand{\Once}{\LTLdiamondminus}
\newcommand{\Since}{\mathbin{\mathcal{S}}}
\newcommand{\BackTo}{\mathbin{\mathcal{B}}}
\newcommand{\WeakPrev}{\LTLcircletilde}
\newcommand{\UNTIL}{\mathbin{\mathcal{U}}}
\newcommand{\WaitFor}{\mathbin{\mathcal{W}}}


%\newenvironment{proposition}{\theoremlike{Proposition}}{\par\medskip}

\newcommand{\val}[1]{\ensuremath{[\![ #1 ]\!]}}

\newcommand{\der}[1]{\ensuremath{\;\;{\mathop{{ %
            \longrightarrow}}\limits^{{#1}}}\!}\;\;} %

\newcommand{\Der}[1]{\ensuremath{\rightarrow_{#1}}} %
\newcommand{\notDer}[1]{\ensuremath{\not\rightarrow_{#1}}} %
%\newcommand{\notDer}[1]{\ensuremath{\;\;\not\rightarrow_{#1}\;\;}} %


\newcommand \tpl[1]{\langle #1 \rangle}


\newcommand{\LTL}{\textsf{LTL}\xspace}
\newcommand{\ForSpec}{\textsf{ForSpec}\xspace}

\newcommand{\SVA}{\textsf{SVA}\xspace}
\newcommand{\VLTL}{\textsf{VLTL}\xspace}
\newcommand{\RLTL}{\textsf{RLTL}\xspace}
%\newcommand{\RE}{\textsf{RE}\xspace}
\newcommand{\NVPA}{\textsf{NVPA}\xspace}
\newcommand{\VPA}{\textsf{VPA}\xspace}
\newcommand{\NFA}{\textsf{NFA}\xspace}
\newcommand{\VPL}{\textsf{VPL}\xspace}
\newcommand{\CARET}{\textsf{CaRet}\xspace}
\newcommand{\VPLs}{\textsf{VPL}s\xspace}
\newcommand{\VRL}{\textsf{VRL}\xspace}
\newcommand{\VRE}{\textsf{VRE}\xspace}
\newcommand{\EMSO}{\textsf{MSO$_{\mu}$}\xspace}
\newcommand{\MSO}{\textsf{MSO}\xspace}
\newcommand{\FO}{\textsf{FO}\xspace}
\newcommand{\EFO}{\textsf{FO}$_{\mu}$\xspace}
\newcommand{\XML}{\textsf{XML}\xspace}
%\newcommand{\ETL}{\textsf{ETL}\xspace}
\newcommand{\PDL}{\textsf{PDL}\xspace}
\newcommand{\AJA}{\textsf{AJA}\xspace}
\newcommand{\AJT}{\textsf{AJT}\xspace}
\newcommand{\SAJA}{\textsf{SAJA}\xspace}
\newcommand{\MAJA}{\textsf{MAJA}\xspace}
\newcommand{\AFA}{\textsf{AFA}\xspace}
\newcommand{\AVPA}{\textsf{AVPA}\xspace}
\newcommand{\PSL}{\textsf{PSL}\xspace}

\newcommand{\NWTL}{\textsf{NWTL}\xspace}
\newcommand{\NWTLPLUS}{\textsf{NWTL}$^{+}$\xspace}


\newcommand{\TWOEXPTIME}{\textsf{2-EXPTIME}\xspace}
\newcommand{\EXPTIME}{\textsf{EXPTIME}\xspace}
\newcommand{\PTIME}{\textsf{PTIME}\xspace}

\newcommand{\WM}{\ensuremath{\mathit{WM}}}
\newcommand{\MR}{\ensuremath{\mathit{MR}}}
\newcommand{\MC}{\ensuremath{\mathit{MC}}\xspace}
\newcommand{\SWM}{\ensuremath{\mathit{MWM}}\xspace}
\newcommand{\Pu}{\ensuremath{\mathcal{P}}}
\newcommand{\Lang}{\ensuremath{\mathcal{L}}}
\newcommand{\Au}{\ensuremath{\mathcal{A}}}
\newcommand{\Gam}{\ensuremath{\mathcal{G}}}
\newcommand{\Tu}{\ensuremath{\mathcal{T}}}
\newcommand{\Stratum}{\ensuremath{\mathcal{S}}}
\newcommand{\RejMain}{\ensuremath{\bot_\MainStates}}
\newcommand{\RejSec}{\ensuremath{\bot_\SecStates}}

%%%%
%%%% The following three macros are confusing with MainStates and SecStates
%%%%
\newcommand{\Msubst}{$M$-substitution\xspace} % M-substitution CHANGE FONT
\newcommand{\Mclosure}{$M$-closure\xspace}    %
\newcommand{\Sclosure}{$S$-closure\xspace}    %

\newcommand{\Region}{\ensuremath{\mathcal{R}}}

%\newcommand{\MainStates}{\ensuremath{\textit{M}}}
%\newcommand{\MainStates}{\ensuremath{M}}
\newcommand{\MainStates}{\ensuremath{\mathsf{M}}}
%\newcommand{\SecStates}{\ensuremath{\textit{S}}}
%\newcommand{\SecStates}{\ensuremath{S}}
\newcommand{\SecStates}{\ensuremath{\mathsf{S}}}
%\newcommand{\FamStratum}{\ensuremath{\mathfrak{F}}}
\newcommand{\FamStratum}{\ensuremath{\mathsf{F}}}
\newcommand{\Acc}{\ensuremath{\textit{Acc}}}
\newcommand{\Inf}{\ensuremath{\textit{Inf}}}
\newcommand{\SF}{\ensuremath{\textit{SF}}}
\newcommand{\Model}{\ensuremath{\textit{X}}}



\newcommand{\Stilde}{\widetilde{\Sigma}\xspace}
\newcommand{\Scall}{\Sigma_{\mathit{call}}}
\newcommand{\call}{{\mathit{call}}}
\newcommand{\ret}{{\mathit{ret}}}
\newcommand{\intA}{{\mathit{int}}}
\newcommand{\Sret}{\Sigma_{\mathit{ret}}}
\newcommand{\Sint}{\Sigma_{\mathit{int}}}

\newcommand{\Nat}{\mathbb{N}}



\newcommand{\SubstOp}[1]{\mathrel{\curvearrowleft_{#1}}}
\newcommand{\SubstBox}{\SubstOp{\Box}}
\newcommand{\MClos}[2]{#1^{\SubstOp{#2}}}
\newcommand{\LangClos}{\MClos{\Lang}{\Box}}
\newcommand{\SClos}[2]{{#1}^{\circlearrowleft_{#2}}}
\newcommand{\SClosure}[1]{\SClos{#1}{\Box}}
\newcommand{\LangSClos}{\SClos{\Lang}{\Box}}

\newcommand{\caller}{\mathsf{c}}
\newcommand{\down}{\mathsf{+}}
\newcommand{\abs}{\mathsf{a}}
\newcommand{\absExt}{\mathsf{ea}}
\newcommand{\PosBool}{\ensuremath{\mathcal{B}^{+}}}

%\newcommand{\dirUp}{\ensuremath{\uparrow}}
\newcommand{\dirUp}{\ensuremath{\leftarrow}}
%\newcommand{\dirDown}{\ensuremath{\downarrow}}
\newcommand{\dirDown}{\ensuremath{\rightarrow}}
%\newcommand{\dirUpAbs}{\ensuremath{\uparrow^{\abs}}}
\newcommand{\dirUpAbs}{\ensuremath{\curvearrowleft}}
%\newcommand{\dirDownAbs}{\ensuremath{\downarrow_{\abs}}}
\newcommand{\dirDownAbs}{\ensuremath{\curvearrowright}}
%\newcommand{\dirDownAbsExt}{\ensuremath{\downarrow_{\absExt}}}
%\newcommand{\dirDownAbsExt}{\ensuremath{\curvearrowright_{\mkern-15.2mu\raisebox{.30em}{$\phantom{.}_{\bullet}$}}}}
\newcommand{\dirDownAbsExt}{\ensuremath{\curvearrowright_{\mkern-8.5mu{\bullet}}}}
%\newcommand{\dirUpAbsExt}{\ensuremath{\uparrow^{\absExt}}}
%\newcommand{\dirUpAbsExt}{\ensuremath{\curvearrowleft_{\mkern-30.2mu\raisebox{.30em}{$\phantom{.}_{\bullet}$}}}}
\newcommand{\dirUpAbsExt}{\ensuremath{\phantom{.}_\bullet{\mkern-8.2mu\curvearrowleft}}}

\newcommand{\DIR}{\textit{DIR}}
\newcommand{\Trans}{\texttt{t}}
\newcommand{\Moves}{\textit{Moves}}
%\newcommand{\dir}{\textit{dir}}
\newcommand{\SUCC}{\mathsf{succ}}
\newcommand{\NULL}{\mathsf{\bot}}
\newcommand{\acc}{\textit{acc}}
\newcommand{\rej}{\textit{rej}}
\newcommand{\rank}{\textit{rank}}
\newcommand{\widt}{\textit{width}}

\newcommand{\DefinedAs}{\,\stackrel{\text{def}}{=}\,}

%\newcommand{\DefOR}{\ensuremath{\;\;\;\big|\;\;\;}}
\newcommand{\DefORmini}{\ensuremath{\;\big|\;}}

\newcommand{\details}[1]{}

%%%%%% Temporal Modalities %%%%%%%%%%%%%%

\newcommand{\true}{\ensuremath{\textup{\texttt{true}}}}
\newcommand{\false}{\ensuremath{\textup{\texttt{false}}}}
\newcommand{\Power}[3]{\mathrel{{#1}|{#2}\rrangle{#3}}}
\newcommand{\Norm}[1]{\ensuremath{\|{#1}\|}}
\newcommand{\WeakPower}[3]{\mathrel{{#1}|{#2}\rangle{#3}}}
\newcommand{\PastPower}[3]{\mathrel{{#1}\llangle{#2}|{#3}}}
\newcommand{\PastWeakPower}[3]{\mathrel{{#1}\langle{#2}|{#3}}}
%\newcommand{\Next}{\ensuremath{\bigcirc}}
\newcommand{\Seq}{\ensuremath{;}}
%\newcommand{\Previous}{\ensuremath{{\mathrlap{\bigcirc}\,\minus\,}}}
\newcommand{\Previous}{\Prev}
%\newcommand{\until}{{\sf U}}
\newcommand{\until}{\Until}
%\newcommand{\since}{{\sf S}}
\newcommand{\since}{\Since}
%\newcommand{\sometimes}{\lozenge}
%\newcommand{\always}{\square}
\newcommand{\always}{\Always}

\newcommand{\comp}[1]{\ensuremath{\overline{#1}}} % complement automaton
\newcommand{\dual}[1]{\ensuremath{\widetilde{#1}}} % dualization

\newcommand{\Outcm}{c\,b\,r}
\newcommand{\UOutcm}{\underline{\Outcm}}

\newenvironment{changemargin}[2]{%
  \begin{list}{}{%
    \setlength{\topsep}{0pt}%
    \setlength{\leftmargin}{#1}%
    %\setlength{\footnotewidth}{#1}%
    \setlength{\rightmargin}{#2}%
    \setlength{\textheight}{21cm}%
   \setlength{\topmargin}{0cm}
   \setlength{\listparindent}{\parindent}%
    \setlength{\itemindent}{\parindent}%
    \setlength{\parsep}{\parskip}%
  }
  \item[]}{\end{list}}


\newcommand{\push}{\textit{push}}
\newcommand{\Regions}{\ensuremath{\textit{REG}}}
\newcommand{\pop}{\textit{pop}}
%\newcommand{\pending}{\ensuremath{\texttt{pending}}}

\newcommand{\Bu}{\ensuremath{\textsf{B}}}
\newcommand{\Co}{\ensuremath{\textsf{C}}}

\newcommand{\alphaANY}{\ensuremath{\alpha_{\mathit{ONE}}}}

%%%% ATTENTION HERE: DON'T USE OLD DEFINITION IN FSTTCS CLASS iS treated as a section %%%%
\newcommand{\myparagraph}[1]{\paragraph{\textup{\textbf{#1}}}}
%\newcommand{\myparagraph}[1]{\noindent\textbf{#1}}

\newcommand{\myparagraphApp}[1]{\vspace{0.5em}\noindent\textup{\textbf{#1}}}

\newcommand{\qin}{\ensuremath{q_{\textit{in}}}}

% MINE
\newcommand{\striver}{\textsf{Striver}\xspace}

\newcommand{\tessla}{TeSSLa\xspace}

\newcommand{\reserved}[1]{\ensuremath{\mathit{#1}}\xspace}
\newcommand{\Syntax}[1]{\keyword{#1}}


\newcommand{\wakeUpTicks}{\reserved{ticks^*}}
\newcommand{\ticks}{\reserved{ticks}}

\newcommand{\caseof}[1]{\mathtt{case}~#1~\mathtt{of}}
%\newcommand{\IF}{\mathtt{if}}  %% Defined in algorithmic
%\newcommand{\THEN}{\mathtt{then}}
%\newcommand{\ELSE}{\mathtt{else}}
%\newcommand{\ELSIF}{\mathtt{elsif}}

\newcommand{\IfThenElse}[3]{\mathtt{if}~#1~\mathtt{then}~#2~\mathtt{else}~#3}

\newcommand{\IntoP}{\mathrel{\rightharpoondown}}


\newcommand{\Vc}[1]{\ensuremath{\langle #1 \rangle}}

\newcommand{\default}{\reserved{default}}
\newcommand{\lift}{\reserved{lift}}
\newcommand{\delay}{\keyword{delay}}
\newcommand{\SHIFT}{\keyword{shift}}

%\newcommand{\at}{\keyword{delayLast}}
\newcommand{\at}{\keyword{at}}

\newcommand{\PrevVal}{\reserved{valprev}}
%\newcommand{\Prev}{\reserved{tprev^{\leftarrow}}}
%\newcommand{\Prev}{\reserved{prev}}
%\DeclareMathOperator*{\prevOp}{\reserved{prev}}
\DeclareMathOperator{\prevOp}{\reserved{prev}}
\DeclareMathOperator{\succOp}{\reserved{succ}}
%\newcommand{\Prev}{\ensuremath{\ensuremath{\reserved{prev}}\limits{\leq}}}
\newcommand{\Prev}{\ensuremath{\prevOp\limits_{<}}}
\newcommand{\PrevEqVal}{\reserved{valprev_\circ}}
%\newcommand{\PrevEq}{\reserved{tprev}}
%\newcommand{\PrevEq}{\reserved{prev_\circ}}
\newcommand{\PrevEq}{\ensuremath{\prevOp\limits_{\leq}}}
\newcommand{\Succ}{\ensuremath{\succOp\limits_{>}}}
\newcommand{\SuccEqVal}{\reserved{valsucc_\circ}}
%\newcommand{\SuccEq}{\reserved{tsucc}}
%\newcommand{\SuccEq}{\reserved{succ_\circ}}
\newcommand{\SuccEq}{\ensuremath{\succOp\limits_{\geq}}}

\newcommand{\PrevS}{\Syntax{\ll}}
%\newcommand{\PrevEqS}{\Syntax{<\sim}}
%\newcommand{\PrevEqS}{\Syntax{<\text{\raisebox{10ex}{$\sim$}}}}
%\newcommand{\PrevEqS}{\ensuremath{\mathrel{\ooalign{$<$\cr\hidewidth\raise{20ex}\hbox{$\sim$}\cr}}}}
\newcommand{\PrevEqS}{\ensuremath{\mathrel{\ooalign{$<$\cr\hidewidth\raisebox{-1.2ex}{$\sim$}\cr}}}}
\newcommand{\SuccS}{\Syntax{\gg}}
%\newcommand{\SuccEqS}{\Syntax{\sim>}}
\newcommand{\SuccEqS}{\ensuremath{\mathrel{\ooalign{$>$\cr\hidewidth\raisebox{-1.2ex}{$\sim$}\cr}}}}
\newcommand{\setS}[1]{\Syntax{\{} #1 \Syntax{\}}}
\newcommand{\valS}[1]{\textsf{(}#1\textsf{)}}
\newcommand{\Ticks}{\keyword{ticks}}
\newcommand{\ticksS}{\keyword{ticks}}
\newcommand{\Union}{\mathrel{\textsf{U}}}
%\newcommand{\delaySym}{\Syntax{delay}}
%\newcommand{\delayS}[1]{\delaySym{}(#1)}
\newcommand{\delayS}[1]{\delay{}(#1)}
%\newcommand{\atSym}{\Syntax{at}}
%\newcommand{\atS}[1]{\atSym{}(#1)}
\newcommand{\atS}[1]{\at{}(#1)}


\newcounter{Problem} \setcounter{Problem}{0}

\newcommand{\Prob}{\addtocounter{Problem}{1}\noindent\paragraph{\textbf{Problem
      \#\theProblem}}}

\newcommand{\Abs}[1]{\ensuremath{#1^\#\xspace}}

\newcommand{\LET}{\mathtt{let}}
\newcommand{\MAX}{\ensuremath{\mathit{max}}}
%\newcommand{\DefOr}{\ensuremath{\hspace{1em}\big|\hspace{1em}}}


\newcommand{\tprev}{\ensuremath{t_{\textit{prev}}}}

\newcommand{\Tstar}{\ensuremath{\mathbb{T}^*}\xspace}

\newcommand{\dom}{\ensuremath{\mathit{dom}}}

%%%%%%%%%%% FROM ANOTHER PAPER ON Nested SRV

%%%%%%%%%%%%%%%%%%%%%%%%%%%%%%%%%%%%%%%%%%%%%%%%%%
%%%% Runtime Verification Macros
%%%%%%%%%%%%%%%%%%%%%%%%%%%%%%%%%%%%%%%%%%%%%%%%%%

\newcommand{\Clock}{\textit{clock}}
\newcommand{\Xclock}{\ensuremath{X_\Clock}}
\newcommand{\SRV}{SRV\xspace}
\newcommand{\NRV}{NestedSRV\xspace}

%%%% types
\newcommand{\BoolT}{\textit{Bool}}
\newcommand{\NatT}{\textit{Nat}}
\newcommand{\IntT}{\textit{Int}}
\newcommand{\StackT}{\textit{Stack}}
\newcommand{\SetT}{\textit{Set}}
\newcommand{\ElemT}{\textit{Elem}}
\newcommand{\andsym}{\texttt{and}}
\newcommand{\orsym}{\texttt{or}}
\newcommand{\notsym}{\texttt{not}}
\newcommand{\truesym}{\texttt{true}}
\newcommand{\falsesym}{\texttt{false}}
\newcommand{\SP}{\hspace{0.03em}}

\newcommand{\ifthenelsesym}{
  $\begin{bmatrix}
    \sym{if}\SP\cdot\SP\sym{then}\SP\cdot\SP\\
        \hspace{0.12em}\sym{else}\SP\cdot\SP
  \end{bmatrix}$}
\newcommand{\ifthenelsetype}[1]{
  $\begin{array}{l}\BoolT \times #1 \times #1\\
    \hspace{3em}\Into #1
  \end{array}$
}
\newcommand{\BoolName}{\textit{Booleans}}
\newcommand{\BoolSorts}{\BoolT}
\newcommand{\NatName}{\textit{Naturals}}
\newcommand{\NatSorts}{\NatT, \BoolT}
\newcommand{\SetName}{\textit{Sets}}
\newcommand{\SetSorts}{\SetT, \BoolT, \ElemT}
\newcommand{\StackName}{\textit{Stacks}}
\newcommand{\StackSorts}{\StackT, \BoolT, \ElemT}

%%%%%

\newcommand{\Sres}{S_{\text{res}}}
\newcommand{\calS}{\mathcal{S}}
\newcommand{\calF}{\mathcal{F}}
\newcommand{\NExpr}{\ensuremath{\textit{NestedExpr}}}
\newcommand{\NestedAtomic}[1]{\ensuremath{\textit{NestedOffset}(#1)}}
\newcommand{\Atomic}[1]{\ensuremath{\textit{Offset}(#1)}}
%\newcommand{\In}{\mathcal{I}}
%\newcommand{\Out}{\mathcal{O}}
\newcommand{\Dom}{\mathcal{D}}
%\newcommand{\sym}[1]{#1}
%\newcommand{\sym}[1]{\ensuremath{\mathtt{#1}}}
\newcommand{\sym}[1]{\ensuremath{\mathtt{#1}}}
\newcommand{\symIF}{\sym{if}}
\newcommand{\symTHEN}{\sym{then}\xspace}
\newcommand{\symELSE}{\sym{else}\xspace}
\newcommand{\symTRUE}{\sym{true}\xspace}
\newcommand{\symFALSE}{\sym{false}\xspace}
\newcommand{\symf}{\sym{f}}
%\newcommand{\DOM}[1]{{#1}^{\Dom}}
\newcommand{\domf}{\DOM{f}}
\newcommand{\clock}[1]{\ensuremath{\text{$#1$}}\xspace}
\newcommand{\clockn}{\clock{n}}
\newcommand{\symDIV}{\sym{div}}
\newcommand{\otau}{\overline{\tau}}
\newcommand{\osigma}{\overline{\sigma}}
\newcommand{\anext}[3]{\ensuremath{#1 \leadsto_{#3} #2}}


\newcommand{\Pred}{\ensuremath{\textit{pred}}}
% Language name
\newcommand{\NSRV}{NestedSRV\xspace}
\newcommand{\LOLA}{LOLA\xspace}


% Translation stuff
\newcommand{\tigger}{trigger\xspace}
\newcommand{\tiggers}{triggers\xspace}
\newcommand{\Tiggers}{Triggers\xspace}

% To change!!!
\newcommand{\SIC}{\ensuremath{\parallel_{\mathcal{P}}}}

\newcommand{\ctrue}{\tbf{true}}
\newcommand{\cfalse}{\tbf{false}}

%\newcommand{\default}{c}

% Specification
\newcommand{\Spec}{\varphi}

% Input variables
\newcommand{\iv}{t}
\newcommand{\ivcall}{\iv_{call}}
\newcommand{\ivbeg}{\iv_{beg}}
\newcommand{\ivend}{\iv_{end}}
\newcommand{\ivret}{\iv_{ret}}
% Output variable
\newcommand{\ov}{s}
% Any stream variable 
\newcommand{\asv}{v}


% Output stream
\newcommand{\os}{\sigma}
% Input streams
\newcommand{\is}{\tau}
\newcommand{\iscall}{\is_{call}}
\newcommand{\isbeg}{\is_{beg}}
\newcommand{\isend}{\is_{end}}
\newcommand{\isret}{\is_{ret}}
% Any stream
\newcommand{\as}{\sigma}

% Abstract sum over execution model
\newcommand{\asum}[1]{\oplus_{#1}}
\newcommand{\aminus}[1]{\ominus_{#1}}

% Guess
\newcommand{\G}{\mathcal{G}}

\newcommand{\gaprev}[1]{\ensuremath{#1 . \ttt{prev}_{abs}}}
\newcommand{\gcprev}[1]{\ensuremath{#1 . \ttt{prev}_{con}}}
\newcommand{\gnow}[1]{\ensuremath{#1 . \ttt{now}}}
\newcommand{\gcnext}[1]{\ensuremath{#1 . \ttt{next}_{con}}}
\newcommand{\ganext}[1]{\ensuremath{#1 . \ttt{next}_{abs}}}

% Input
\newcommand{\inp}{\sigma}
\newcommand{\Inp}{\Sigma}

% Execution model
\newcommand{\exm}{\eta}
\newcommand{\Exm}{H}

% Automata
\newcommand{\LA}[1]{\mathcal{L}(#1)}

% Path
\newcommand{\path}{\alpha}

% Common types
\newcommand{\FNat}{\Nat_\bot}

% Brackets, tuples
\newcommand{\Bracket}[1]{\llbracket #1 \rrbracket}
\newcommand{\Tup}[1]{\ensuremath{\langle #1 \rangle}}

% Felipe's notes
\newcommand{\felitodo}[1]{\todo[linecolor=red,backgroundcolor=red!25,bordercolor=red]{Felipe: #1}}
\newcommand{\felitext}[1]{\todo[inline,linecolor=red,backgroundcolor=red!25,bordercolor=red]{Felipe: #1}}


\definecolor{darkblue}{rgb} {0.200, 0.200, 0.800}
\definecolor{darkred} {rgb} {0.800, 0.200, 0.200}
\definecolor{darkgreen}{rgb}{0.200, 0.800, 0.200}

\newcommand{\colorEv}[2]{\ensuremath{\textcolor{#2}{\llbracket} #1
    \textcolor{#2}{\rrbracket}}}
\newcommand{\colorSigmaEval}[3]{\ensuremath{\colorEv{#1}{#3}_{#2}}}
\newcommand{\colorEval}[2]{\colorEv{#1}{#2}}
\newcommand{\colorEvalSigma}[2]{\colorSigmaEval{#1}{\sigma}{#2}}
%\newcommand{\evalSpec}[1]{\colorSigmaEval{#1}{darkblue}}
\newcommand{\evalSpec}[1]{\colorEval{#1}{black}}
\newcommand{\evalSpecsigma}[2]{\colorEval{#1}{#2}{black}}
\newcommand{\evalE}[1]{\colorEval{#1}{darkred}}
\newcommand{\evalEsigma}[2]{\colorSigmaEval{#1}{#2}{darkred}}
%\newcommand{\evalTau}[1]{\colorSigmaEval{#1}{darkgreen}}
\newcommand{\evalTau}[1]{\colorEvalSigma{#1}{darkblue}}
\newcommand{\evalTauN}[1]{\colorEval{#1}{darkblue}}
\newcommand{\evalTausigma}[2]{\colorSigmaEval{#1}{#2}{darkblue}}
\newcommand{\evalTicks}[1]{\colorEvalSigma{#1}{brown}}
\newcommand{\evalTicksN}[1]{\colorEval{#1}{brown}}
\newcommand{\evalTickssigma}[2]{\colorSigmaEval{#1}{#2}{brown}}
\newcommand{\presentLabel}{\ensuremath{\mathcal{O}}}
\newcommand{\pastLabel}{\ensuremath{\prec}}
\newcommand{\edgesSpec}{\ensuremath{Edges_\spec}}
\newcommand{\interp}{\mathcal{I}}
\newcommand{\Val}{\mathit{V}}

%\newcommand{\undef}{\reserved{undef}}

\newcommand{\specialbot}[2]{\ensuremath{\bot^{#1}_{\text{#2}}}\xspace}
\newcommand{\untypedspecialbot}[1]{\ensuremath{\bot_{\text{#1}}}\xspace}
\newcommand{\D}{D}

% \newcommand{\maybe}[2]{\ensuremath{\mathtt{Maybe}}^#1_{\text{#2}}}
\newcommand{\T}{\ensuremath{\mathbb{T}}\xspace}
\newcommand{\outT}{\specialbot{\T}{outside}}
\newcommand{\outD}{\specialbot{\D}{outside}}
\newcommand{\notickD}{\specialbot{\D}{notick}}
\newcommand{\outDS}{\Syntax{outside}_\D}
\newcommand{\outTS}{\Syntax{outside}_{\T}}
\newcommand{\notickDS}{\Syntax{notick}_\D}
\newcommand{\maybe}[2]{\ensuremath{#1 \cup \{\specialbot{#1}{#2}\}}}
% \newcommand{\outT}{\untypedspecialbot{outside}}
% \newcommand{\outD}{\untypedspecialbot{outside}}
% \newcommand{\notickD}{\untypedspecialbot{notick}}
% \newcommand{\outDS}{\Syntax{outside}}
% \newcommand{\outTS}{\Syntax{outside}}
% \newcommand{\notickDS}{\Syntax{notick}}

\newcommand{\maybeOutT}{\maybe{\T}{outside}}
\newcommand{\maybeOutD}{\maybe{D}{outside}}
\newcommand{\maybeNotickD}{\maybe{D}{notick}}
\newcommand{\undef}{\reserved{undef}} % Might be confusing with bottom

\newcommand{\entails}{\vdash}
\newcommand{\nonZenoT}{2^{\T}_{nz}}
\newcommand{\offsetEps}{\T_\epsilon}

\newcommand{\lambdatS}[1]{\Syntax{(\text{\textlambda} t . }#1\Syntax{)}}
\newcommand{\evalAlphaT}[1]{\colorSigmaEval{#1}{black}\alpha_t}
\newcommand{\evalAlpha}[1]{\colorSigmaEval{#1}{black}\alpha}

\newcommand{\Eval}{\mathit{Eval}\xspace}
\newcommand{\EvalPhi}{\ensuremath{\Eval_\varphi}\xspace}

% Processors
\newcommand{\colorboxnode}[2]{%
    \colorlet{currentcolor}{.}%
    {\color{#1}%
    \fbox{\color{currentcolor}\strut \ensuremath{#2}}}%
    \xspace
}
\newcommand*{\tickNode}[1]{\colorboxnode{green}{#1}}
\newcommand*{\valueNode}[1]{\colorboxnode{blue}{#1}}
\newcommand*{\gateNode}[1]{\colorboxnode{black}{#1}}

\newcommand*{\delayTickNode}{\tickNode{delay}}
\newcommand*{\srcTickNode}{\tickNode{srcTick}}
\newcommand*{\constTickNode}{\tickNode{constTick~c}}
\newcommand*{\UTickNode}{\tickNode{UTick}}

\newcommand*{\prevNode}{\valueNode{prev}}
\newcommand*{\prevEqNode}{\valueNode{prevEq}}
\newcommand*{\prevValNode}{\valueNode{prevVal}}
\newcommand*{\prevEqValNode}{\valueNode{prevEqVal}}
\newcommand*{\funcNode}{\valueNode{func~f}}
\newcommand*{\tNode}{\valueNode{\mathbf{t}}}

\newcommand*{\coreNode}{\gateNode{core}}
\newcommand*{\inputNode}{\gateNode{in}}

\newcommand*{\StreamT}[1]{\ensuremath{\mathcal{S}_{#1}}}
\newcommand*{\StreamProgressT}[1]{\ensuremath{\mathcal{S}^{\progressEv}_{#1}}}
\newcommand*{\StreamUnit}{\StreamT{Unit}}
\newcommand*{\StreamProgressUnit}{\StreamProgressT{Unit}}

% Processor functions
\newcommand*{\voteFun}{\ensuremath{vote}\xspace}
\newcommand*{\execFun}{\ensuremath{exec}\xspace}
\newcommand*{\rinseFun}{\ensuremath{rinse}\xspace}
\newcommand*{\mayTickFun}{\ensuremath{mayTick}\xspace}
\newcommand*{\computeValFun}{\ensuremath{computeVal}\xspace}

% Progress events data
\newcommand*{\progressEv}{\untypedspecialbot{progress}}

% Translation phase
\newcommand*{\signalFun}{\ensuremath{signalNode}\xspace}
\newcommand*{\translateTicks}[1]{\ensuremath{translateTicks(#1)}\xspace}
\newcommand*{\translateSpec}{\ensuremath{translateSpec}\xspace}
\newcommand*{\translateVal}[1]{\ensuremath{translateVal(#1)}\xspace}

% System state
\newcommand*{\quiescT}{\ensuremath{\T \cup \{-\infty\}}\xspace}
\newcommand*{\globalSystem}{\ensuremath{S}\xspace}
\newcommand*{\graph}{\ensuremath{G}\xspace}
\newcommand*{\graphT}{\ensuremath{\tupleof{V,E}}\xspace}
\newcommand*{\eStates}{\ensuremath{\Sigma_E}\xspace}
\newcommand*{\vStates}{\ensuremath{\Sigma_V}\xspace}
\newcommand*{\inputTagT}{\ensuremath{InTag}\xspace}
\newcommand*{\nodeStateT}{\ensuremath{nodeState}\xspace}
\newcommand*{\edgeStateT}{\ensuremath{edgeState}\xspace}
\newcommand*{\vStateFunT}{\ensuremath{(V \to \nodeStateT)}\xspace}
\newcommand*{\eStateFunT}{\ensuremath{(V \times \inputTagT \to \edgeStateT)}\xspace}
\newcommand*{\dynSys}{\ensuremath{\Sigma}\xspace}
\newcommand*{\dynSysT}{\ensuremath{\tupleof{\vStateFunT, \eStateFunT}}\xspace}
\newcommand*{\portTag}[1]{\ensuremath{\mathtt{#1}}\xspace}

% Channel functions
\newcommand*{\writeChansFun}{\ensuremath{writeChannels}\xspace}
\newcommand*{\readChansFun}{\ensuremath{readChannels}\xspace}
\newcommand{\emptyChannel}{\specialbot{\D}{emptychannel}}

% Operational semantics
\newcommand{\ticksDef}[1]{\mathcal{E}^{ticks}_{#1}}
\newcommand{\valsDef}[1]{\mathcal{E}^{vals}_{#1}}



\newcommand{\colorHEval}[2]{\colorSigmaEval{#1}{H}{#2}}
\newcommand{\evalEH}[1]{\colorHEval{#1}{darkred}}
\newcommand{\evalTicksH}[1]{\colorHEval{#1}{brown}}

\newcommand{\opEval}[5]{\colorSigmaEval{#4}{#5}{#2\vert_{#1,#3}}(#3)}
\newcommand{\mayTick}[4]{\opEval{#2}{#3}{#4}{#1}{brown}}
\newcommand{\computeVal}[4]{\opEval{#2}{#3}{#4}{#1}{darkred}}
\newcommand{\computeTau}[4]{\opEval{#2}{#3}{#4}{#1}{darkblue}}

% Dependency graph
\newcommand{\ddep}{\rightarrow}
\newcommand{\ddepPres}{\xrightarrow{0}}
\newcommand{\ddepPresStar}{\xrightarrow{\smash{0}}^*}

\newcommand{\ddepPast}{\xrightarrow{-}}
\newcommand{\ddepFut}{\xrightarrow{+}}

\newcommand{\Present}{\textit{Present}}
\newcommand{\Past}{\textit{Past}}
\newcommand{\Future}{\textit{Future}}

%%%% For algorithmic

% New definitions
\algnewcommand\algorithmicswitch{\textbf{switch}}
\algnewcommand\algorithmiccase{\textbf{case}}
\algnewcommand\algorithmicassert{\texttt{assert}}
\algnewcommand\Assert[1]{\State \algorithmicassert(#1)}%
% New "environments"
\algdef{SE}[SWITCH]{Switch}{EndSwitch}[1]{\algorithmicswitch\ #1\ \algorithmicdo}{\algorithmicend\ \algorithmicswitch}%
\algdef{SE}[CASE]{Case}{EndCase}[1]{\algorithmiccase\ #1}{\algorithmicend\ \algorithmiccase}%
\algtext*{EndSwitch}%
\algtext*{EndCase}%

%%%
\newcommand{\SUP}{\ensuremath{\mathit{sup}}}

\newcommand{\tin}{\ensuremath{t_\textit{in}}}
%\newcommand{\tqin}{\ensuremath{t_q^{\textit{in}}}}
\newcommand{\tqin}{\tin}
\newcommand{\tout}{\ensuremath{t_\textit{out}}}
%\newcommand{\tqout}{\ensuremath{t_q^{\textit{out}}}}
\newcommand{\tqout}{\tout}
\newcommand{\append}{\textit{append}}
\newcommand{\tq}{\ensuremath{t_q}}

\newcommand{\App}{\ensuremath{\mathrel{\scalebox{0.8}{++}}}}

\newcommand{\PrevStt}[1]{\ensuremath{\textbf{\texttt{<<}}~#1}}
\newcommand{\PrevEqStt}[1]{\ensuremath{\textbf{\texttt{<\~}}~#1}}

\newcommand{\SY}[1]{\text{\texttt{#1}}}
\newcommand{\PrevEqV}[1]{\SY{(\~}#1\SY{)}}
\newcommand{\PrevV}[1]{\SY{(<}#1\SY{)}}
\newcommand{\SuccEqV}[1]{\SY{(}#1\SY{~)}}
\newcommand{\SuccV}[1]{\SY{(}#1\SY{>)}}

\newcommand{\VERT}{\phantom{|}}

\newcommand{\cemph}[1]{\textcolor{red}{\emph{#1}}}

%%% For LOLA
\newcommand{\IN}[1]{\textcolor{darkblue}{#1}}
\newcommand{\OUT}[1]{\textcolor{darkred}{#1}}
\newcommand{\DOM}[1]{\textcolor{darkgreen}{#1}}
\newcommand{\STRM}[1]{\textcolor{darkorange}{#1}}
\newcommand{\Input}{\IN{input}\xspace}
\newcommand{\Output}{\OUT{output}\xspace}
\newcommand{\Trigger}{\OUT{trigger}\xspace}
\newcommand{\Dbool}{\ensuremath{\text{\DOM{bool}}}\xspace}
\newcommand{\DboolThree}{\ensuremath{\text{\DOM{bool3}}}\xspace}
\newcommand{\DboolFour}{\ensuremath{\text{\DOM{bool4}}}\xspace}
\newcommand{\Dint}{\ensuremath{\text{\DOM{int}}}\xspace}
\newcommand{\Dset}{\ensuremath{\text{\DOM{set}}}\xspace}
\newcommand{\Delem}{\ensuremath{\text{\DOM{elem}}}\xspace}
\newcommand{\Keyword}[1]{\texttt{#1}\xspace}
\newcommand{\IFkw}{\Keyword{if}}
\newcommand{\THENkw}{\Keyword{then}}
\newcommand{\ELSEkw}{\Keyword{else}}
\newcommand{\IFk}{\ensuremath{\Keyword{if}\;\;}}
\newcommand{\THENk}{\ensuremath{\THENkw}\;\;}
\newcommand{\ELSEk}{\ensuremath{\ELSEkw\;\;}}

\newcommand{\sbool}{\textsf{bool}}
\newcommand{\sint}{\textsf{int}}
\newcommand{\sset}{\textsf{set}}
\newcommand{\ssetT}{\textsf{sestt}}
\newcommand{\selem}{\textsf{elem}}
\newcommand{\req}{\textit{req}}
\newcommand{\resp}{\textit{resp}}
\newcommand{\evresp}{\textit{evresp}}
\newcommand{\granted}{\textit{granted}}
\newcommand{\ok}{\textit{ok}}
\newcommand{\nreq}{\textit{nreq}}
\newcommand{\nresp}{\textit{nresp}}
\newcommand{\good}{\textit{good}}
\newcommand{\AlwaysGood}{\textit{G\_good}}
\newcommand{\END}{\textit{end}}
\newcommand{\pending}{\textit{pending}}
\newcommand{\reqid}{\textit{reqid}}
\newcommand{\respid}{\textit{respid}}
%\newcommand{\pending}{\textit{pending}}
\newcommand{\Int}{\text{Int}}

\newcommand{\hs}[1]{\hspace{#1cm}}
\newcommand{\mhs}{\hspace{0.5cm}}
\newcommand{\lsp}{\vspace{0.3cm}}

\newcommand{\StreamName}[1]{\textsf{#1}}
\newcommand{\Exp}[1]{\StreamName{E}_{\StreamName{#1}}}

\newcommand{\first}{\StreamName{first}}
\newcommand{\control}{\StreamName{control}}
\newcommand{\initial}{\StreamName{init}}
\newcommand{\last}{\StreamName{last}}
\newcommand{\test}[1]{\StreamName{test}_{\StreamName{#1}}}
\newcommand{\strFalse}{\StreamName{false}}
\newcommand{\strTrue}{\StreamName{true}}
\newcommand{\step}{\StreamName{step}}
\newcommand{\uniform}{\StreamName{uniform}}


\newcommand{\unique}{\StreamName{unique}}
\newcommand{\init}{\StreamName{initial}}
\newcommand{\transition}{\StreamName{transition}}
\newcommand{\accepting}{\StreamName{accepting}}

\newcommand{\prevEq}{\mathit{prev}_{\leq}}
\newcommand{\prev}{\mathit{prev}}

\newcommand{\cgreen}[1]{\textcolor{darkgreen}{#1}}
\newcommand{\cred}[1]{\textcolor{darkred}{#1}}
\newcommand{\cblue}[1]{\textcolor{darkblue}{#1}}
\newcommand{\corange}[1]{\textcolor{darkorange}{#1}}
\newcommand{\cmag}[1]{\textcolor{darkmagenta}{#1}}
\newcommand{\cmgreen}[1]{\textcolor{midgreen}{#1}}
\newcommand{\cmred}[1]{\textcolor{midred}{#1}}
\newcommand{\cmblue}[1]{\textcolor{midblue}{#1}}

% \newcommand{eventex}{
% \begin{tikzpicture}

% \matrix[column sep = 0.5em, draw] (m) {
%   \node[anchor = east] (m-1-2) {Event stream \texttt{e}};
%     \& \timing[name=m-2-3] at (0,-0.15) {ZZZ \n{0} Z \n{7} ZZZZ \n{2} ZZ \n{2} ZZZ \n{15} ZZ}; \\ 
  
%   \node[anchor = east] (m-1-3) {Signal \texttt{s}};  
%     \& \timing[name=m-2-2] at (0,-0.15) {4D{0}N(x1)4D{7}5D{2}2D{15}};\\
    
%   \node[anchor = east] (end-inputs) {Event stream \texttt{e2}}; 
%     \& \timing[name=m-2-3] at (0,-0.15) {ZZZZ \n{7} ZZZZ \n{2} ZZZZZ \n{15} ZZ}; \\  \\[0.5em]
% };
% \end{tikzpicture}
% }

\newcommand{\Time}{\ensuremath{\mathbb{T}}\xspace}
% \newcommand{\delay}{\ensuremath{\textit{delay}}\xspace}
\newcommand{\shift}{\ensuremath{\textit{shift}}\xspace}
\newcommand{\within}{\ensuremath{\textit{within}}\xspace}


\newcommand{\op}[1]{\operatorname{#1}}
\newcommand{\peek}{\op{peek}}
%\newcommand{\pop}{\op{pop}}
%\newcommand{\last}{\op{last}}
\newcommand{\tim}{\op{time}}
% \newcommand{\Val}{\op{value}}
\newcommand{\emit}{\op{emit}}
\newcommand{\dequeue}{\op{dequeue}}
\newcommand{\progress}{\op{progress}}

\newcommand{\plotwidth}{32mm}
\newcommand{\plotheight}{20mm}

\usepackage{mathtools}
\usepackage[ligature,reserved]{semantic}
\input{solidity-highlighting.tex}
\usepackage{paralist}
\usepackage{multicol}

\newcommand{\Hspeed}{H1}
\newcommand{\Hbetteralg}{H2}
\newcommand{\Hagr}{H3}
\newcommand{\Hbyzantine}{H4}
\newcommand{\Hdegrade}{H5}

\algblockdefx[Upon]{Upon}{EndUpon}%
[1]{{\bf upon} (#1) {\bf do}}%
{{\bf end upon}}
\algblockdefx[When]{When}{EndWhen}%
[1]{{\bf when} (#1) {\bf do}}%
{{\bf end when}}
\mathlig{<-}{\leftarrow}
\mathlig{==}{\equiv}
\mathlig{<<}{\langle}
\mathlig{>>}{\rangle}
%% Variables in implementation
\reservestyle{\variables}{\text}
\variables{epoch,current,history,theset[the\_set],pending, tobroadcast[to\_broadcast]}

\reservestyle{\setops}{\text}
\setops{add,get,Init,Add,Get,BAdd,EpochInc,Broadcast,Deliver,GetEpoch,Propose,SetDeliver}

\reservestyle{\structs}{\text}
\structs{DPO,BAB,BRB,SBC,DSO}

\reservestyle{\stmt}{\textbf}
\stmt{call,ack,drop,return,assert,wait}

\reservestyle{\messages}{\texttt}
\messages{madd[add],mepochinc[epinc]}

\reservestyle{\api}{\texttt}
\api{apiBAdd,apiAdd,apiGet,apiEpochInc,apiTheSet[{the\_set}],apiHistory}

\reservestyle{\schain}{\texttt}
\schain{history,epoch,theset[{the\_set}],add,get,epochinc[{epoch\_inc}],getepoch[{get\_epoch}],
Fast, Slow, Basic, Central}

\newcommand{\Servers}{N}

\usepackage{color, colortbl}

 \usepackage{tikz-timing}
 \usetikzlibrary{matrix, shapes.misc, decorations, decorations.pathreplacing, automata, fit}
 \tikzset{
 	ampersand replacement=\&, % alternative to using fragile frames
 	event/.style = {draw, cross out, inner sep = 1pt,xshift=0.75pt},
 	event label/.style = {font = \tiny, label distance = -2pt}  
 }

\usepackage{listings}
\lstset{
	%  extendedchars=true,
	%  inputencoding=utf8
	basicstyle=\small\ttfamily,
	%  stringstyle=
	%  keywordsyle={[2]\underbar},
	commentstyle=\rmfamily\it,
	extendedchars=true,
	keepspaces=true, 
	columns=flexible,
%	frame=tlrb,
	keywordstyle=\ttfamily\bfseries
}
\lstdefinestyle{long}{}
\lstdefinestyle{inline}{   
	basicstyle=\ttfamily\footnotesize,
	numbers=none,
	xleftmargin=2em,
	aboveskip=0.3ex,
	belowskip=0ex,
	%  belowcaptionskip=0em,
	%  lineskip=0em
}
\lstdefinelanguage{tessla}{
	keywords={on, if, fun, define, yield, out, trigger, while, window, and, in},
	keywordstyle=\ttfamily\color{darkblue}\bfseries,
	classoffset=1,
	morekeywords={EStream,VStream,Int,Bool},  
	%  keywordstyle=\ttfamily\bfseries,
	keywordstyle=\underline,
	classoffset=0,
	comment=[l]{--}
}
\definecolor{StriverBlue}{RGB}{42, 0, 255}
\definecolor{StriverGreen}{RGB}{0, 155, 0}
\definecolor{StriverRed}{RGB}{155, 0, 0}
\definecolor{StriverOrange}{RGB}{255, 183, 68}
\definecolor{StriverYellow}{RGB}{155, 155, 45}
\lstdefinelanguage{Striver}
{
    basicstyle=\ttfamily,
    keywordstyle=[1]\color{StriverYellow},
    keywordstyle=[2]\color{StriverBlue},
    keywordstyle=[3]\bfseries,
%    keywordstyle=[4]\itshape,
    keywordstyle=[4]\color{StriverBlue},
    keywordstyle=[5]\color{StriverOrange},
    keywordstyle=[6]\color{StriverGreen},
    keywordstyle=[7]\color{StriverRed},
%    otherkeywords = {
%    <~,<<,~>,>>,shift,macro,:=,const,input,output,true,false,ticks,define,outside,notick,U,delay,at,||,&&,+,-,==,!,!=,if,let,in then,else,char,int,bool,unit,Time,_eps,void},
    otherkeywords = {
    <~,<<,~>,>>,:=||,&&,+,-,==,!,!=},
    morekeywords = [1]{<~,<<,~>,>>},
    morekeywords = [2]{shift,macro, :=, input,const, output, ticks, define},
    morekeywords = [3]{outside, notick, true, false},
    morekeywords = [4]{if, then, else},
    morekeywords = [5]{char, int, bool, unit, Time, _eps, void},
    morekeywords = [6]{U, delay, ||, &&, +, -, !, !=, ==, at},
    morekeywords = [7]{let, in},
}
\lstdefinestyle{pseudo}{language={},%
 % basicstyle=\normalfont,%
 % keywordstyle=\normalfont\bfseries,%
  morecomment=[l]{//},%
  morekeywords={for,to,while,do,if,then,else,and,or,not},%
  mathescape=true,%
  columns=fullflexible
}

%%%% LTL Symbols %%%%%
\makeatletter
\DeclareFontFamily{OMX}{MnSymbolE}{}
\DeclareSymbolFont{MnLargeSymbols}{OMX}{MnSymbolE}{m}{n}
\SetSymbolFont{MnLargeSymbols}{bold}{OMX}{MnSymbolE}{b}{n}
\DeclareFontShape{OMX}{MnSymbolE}{m}{n}{
    <-6>  MnSymbolE5
   <6-7>  MnSymbolE6
   <7-8>  MnSymbolE7
   <8-9>  MnSymbolE8
   <9-10> MnSymbolE9
  <10-12> MnSymbolE10
  <12->   MnSymbolE12
}{}
\DeclareFontShape{OMX}{MnSymbolE}{b}{n}{
    <-6>  MnSymbolE-Bold5
   <6-7>  MnSymbolE-Bold6
   <7-8>  MnSymbolE-Bold7
   <8-9>  MnSymbolE-Bold8
   <9-10> MnSymbolE-Bold9
  <10-12> MnSymbolE-Bold10
  <12->   MnSymbolE-Bold12
}{}

\let\llangle\@undefined
\let\rrangle\@undefined
\DeclareMathDelimiter{\llangle}{\mathopen}%
                     {MnLargeSymbols}{'164}{MnLargeSymbols}{'164}
\DeclareMathDelimiter{\rrangle}{\mathclose}%
                     {MnLargeSymbols}{'171}{MnLargeSymbols}{'171}
\makeatother

% \newcommand{\mymid}{\mathbin{\mid}}
% \newcommand{\Power}[3]{{#2}\!\mymid\!{#1}\rangle\!\rangle{#3}}
% \newcommand{\WeakPower}[3]{{#2}\mymid\!{#1}\rangle{#3}} %warning: not dual power!!
% \newcommand{\PowerSub}[3]{{#2}\mymid{#1}\rangle\!\rangle{#3}}
% \newcommand{\WeakPowerSub}[3]{{#2}\mymid{#1}\rangle{#3}} %warning: not dual power!!

\newcommand{\Always}{\LTLsquare}
%\newcommand{\Always}{\mathbf{G}}
\newcommand{\Event}{\LTLdiamond}
\newcommand{\Eventually}{\Event}
\newcommand{\Next}{\LTLcircle}
\newcommand{\PrevLTL}{\LTLcircleminus}
\newcommand{\Previously}{\Prev}
\newcommand{\PrevNoFirst}{\LTLcircletilde}
\newcommand{\Before}{\PrevNoFirst}
\newcommand{\HasAlwaysBeen}{\LTLsquareminus}
\newcommand{\SoFar}{\HasAlwaysBeen}
\newcommand{\Once}{\LTLdiamondminus}
\newcommand{\Since}{\mathbin{\mathcal{S}}}
\newcommand{\BackTo}{\mathbin{\mathcal{B}}}
\newcommand{\WeakPrev}{\LTLcircletilde}
\newcommand{\UNTIL}{\mathbin{\mathcal{U}}}
\newcommand{\WaitFor}{\mathbin{\mathcal{W}}}


%\newenvironment{proposition}{\theoremlike{Proposition}}{\par\medskip}

\newcommand{\val}[1]{\ensuremath{[\![ #1 ]\!]}}

\newcommand{\der}[1]{\ensuremath{\;\;{\mathop{{ %
            \longrightarrow}}\limits^{{#1}}}\!}\;\;} %

\newcommand{\Der}[1]{\ensuremath{\rightarrow_{#1}}} %
\newcommand{\notDer}[1]{\ensuremath{\not\rightarrow_{#1}}} %
%\newcommand{\notDer}[1]{\ensuremath{\;\;\not\rightarrow_{#1}\;\;}} %


\newcommand \tpl[1]{\langle #1 \rangle}


\newcommand{\LTL}{\textsf{LTL}\xspace}
\newcommand{\ForSpec}{\textsf{ForSpec}\xspace}

\newcommand{\SVA}{\textsf{SVA}\xspace}
\newcommand{\VLTL}{\textsf{VLTL}\xspace}
\newcommand{\RLTL}{\textsf{RLTL}\xspace}
%\newcommand{\RE}{\textsf{RE}\xspace}
\newcommand{\NVPA}{\textsf{NVPA}\xspace}
\newcommand{\VPA}{\textsf{VPA}\xspace}
\newcommand{\NFA}{\textsf{NFA}\xspace}
\newcommand{\VPL}{\textsf{VPL}\xspace}
\newcommand{\CARET}{\textsf{CaRet}\xspace}
\newcommand{\VPLs}{\textsf{VPL}s\xspace}
\newcommand{\VRL}{\textsf{VRL}\xspace}
\newcommand{\VRE}{\textsf{VRE}\xspace}
\newcommand{\EMSO}{\textsf{MSO$_{\mu}$}\xspace}
\newcommand{\MSO}{\textsf{MSO}\xspace}
\newcommand{\FO}{\textsf{FO}\xspace}
\newcommand{\EFO}{\textsf{FO}$_{\mu}$\xspace}
\newcommand{\XML}{\textsf{XML}\xspace}
%\newcommand{\ETL}{\textsf{ETL}\xspace}
\newcommand{\PDL}{\textsf{PDL}\xspace}
\newcommand{\AJA}{\textsf{AJA}\xspace}
\newcommand{\AJT}{\textsf{AJT}\xspace}
\newcommand{\SAJA}{\textsf{SAJA}\xspace}
\newcommand{\MAJA}{\textsf{MAJA}\xspace}
\newcommand{\AFA}{\textsf{AFA}\xspace}
\newcommand{\AVPA}{\textsf{AVPA}\xspace}
\newcommand{\PSL}{\textsf{PSL}\xspace}

\newcommand{\NWTL}{\textsf{NWTL}\xspace}
\newcommand{\NWTLPLUS}{\textsf{NWTL}$^{+}$\xspace}


\newcommand{\TWOEXPTIME}{\textsf{2-EXPTIME}\xspace}
\newcommand{\EXPTIME}{\textsf{EXPTIME}\xspace}
\newcommand{\PTIME}{\textsf{PTIME}\xspace}

\newcommand{\WM}{\ensuremath{\mathit{WM}}}
\newcommand{\MR}{\ensuremath{\mathit{MR}}}
\newcommand{\MC}{\ensuremath{\mathit{MC}}\xspace}
\newcommand{\SWM}{\ensuremath{\mathit{MWM}}\xspace}
\newcommand{\Pu}{\ensuremath{\mathcal{P}}}
\newcommand{\Lang}{\ensuremath{\mathcal{L}}}
\newcommand{\Au}{\ensuremath{\mathcal{A}}}
\newcommand{\Gam}{\ensuremath{\mathcal{G}}}
\newcommand{\Tu}{\ensuremath{\mathcal{T}}}
\newcommand{\Stratum}{\ensuremath{\mathcal{S}}}
\newcommand{\RejMain}{\ensuremath{\bot_\MainStates}}
\newcommand{\RejSec}{\ensuremath{\bot_\SecStates}}

%%%%
%%%% The following three macros are confusing with MainStates and SecStates
%%%%
\newcommand{\Msubst}{$M$-substitution\xspace} % M-substitution CHANGE FONT
\newcommand{\Mclosure}{$M$-closure\xspace}    %
\newcommand{\Sclosure}{$S$-closure\xspace}    %

\newcommand{\Region}{\ensuremath{\mathcal{R}}}

%\newcommand{\MainStates}{\ensuremath{\textit{M}}}
%\newcommand{\MainStates}{\ensuremath{M}}
\newcommand{\MainStates}{\ensuremath{\mathsf{M}}}
%\newcommand{\SecStates}{\ensuremath{\textit{S}}}
%\newcommand{\SecStates}{\ensuremath{S}}
\newcommand{\SecStates}{\ensuremath{\mathsf{S}}}
%\newcommand{\FamStratum}{\ensuremath{\mathfrak{F}}}
\newcommand{\FamStratum}{\ensuremath{\mathsf{F}}}
\newcommand{\Acc}{\ensuremath{\textit{Acc}}}
\newcommand{\Inf}{\ensuremath{\textit{Inf}}}
\newcommand{\SF}{\ensuremath{\textit{SF}}}
\newcommand{\Model}{\ensuremath{\textit{X}}}



\newcommand{\Stilde}{\widetilde{\Sigma}\xspace}
\newcommand{\Scall}{\Sigma_{\mathit{call}}}
\newcommand{\call}{{\mathit{call}}}
\newcommand{\ret}{{\mathit{ret}}}
\newcommand{\intA}{{\mathit{int}}}
\newcommand{\Sret}{\Sigma_{\mathit{ret}}}
\newcommand{\Sint}{\Sigma_{\mathit{int}}}

\newcommand{\Nat}{\mathbb{N}}



\newcommand{\SubstOp}[1]{\mathrel{\curvearrowleft_{#1}}}
\newcommand{\SubstBox}{\SubstOp{\Box}}
\newcommand{\MClos}[2]{#1^{\SubstOp{#2}}}
\newcommand{\LangClos}{\MClos{\Lang}{\Box}}
\newcommand{\SClos}[2]{{#1}^{\circlearrowleft_{#2}}}
\newcommand{\SClosure}[1]{\SClos{#1}{\Box}}
\newcommand{\LangSClos}{\SClos{\Lang}{\Box}}

\newcommand{\caller}{\mathsf{c}}
\newcommand{\down}{\mathsf{+}}
\newcommand{\abs}{\mathsf{a}}
\newcommand{\absExt}{\mathsf{ea}}
\newcommand{\PosBool}{\ensuremath{\mathcal{B}^{+}}}

%\newcommand{\dirUp}{\ensuremath{\uparrow}}
\newcommand{\dirUp}{\ensuremath{\leftarrow}}
%\newcommand{\dirDown}{\ensuremath{\downarrow}}
\newcommand{\dirDown}{\ensuremath{\rightarrow}}
%\newcommand{\dirUpAbs}{\ensuremath{\uparrow^{\abs}}}
\newcommand{\dirUpAbs}{\ensuremath{\curvearrowleft}}
%\newcommand{\dirDownAbs}{\ensuremath{\downarrow_{\abs}}}
\newcommand{\dirDownAbs}{\ensuremath{\curvearrowright}}
%\newcommand{\dirDownAbsExt}{\ensuremath{\downarrow_{\absExt}}}
%\newcommand{\dirDownAbsExt}{\ensuremath{\curvearrowright_{\mkern-15.2mu\raisebox{.30em}{$\phantom{.}_{\bullet}$}}}}
\newcommand{\dirDownAbsExt}{\ensuremath{\curvearrowright_{\mkern-8.5mu{\bullet}}}}
%\newcommand{\dirUpAbsExt}{\ensuremath{\uparrow^{\absExt}}}
%\newcommand{\dirUpAbsExt}{\ensuremath{\curvearrowleft_{\mkern-30.2mu\raisebox{.30em}{$\phantom{.}_{\bullet}$}}}}
\newcommand{\dirUpAbsExt}{\ensuremath{\phantom{.}_\bullet{\mkern-8.2mu\curvearrowleft}}}

\newcommand{\DIR}{\textit{DIR}}
\newcommand{\Trans}{\texttt{t}}
\newcommand{\Moves}{\textit{Moves}}
%\newcommand{\dir}{\textit{dir}}
\newcommand{\SUCC}{\mathsf{succ}}
\newcommand{\NULL}{\mathsf{\bot}}
\newcommand{\acc}{\textit{acc}}
\newcommand{\rej}{\textit{rej}}
\newcommand{\rank}{\textit{rank}}
\newcommand{\widt}{\textit{width}}

\newcommand{\DefinedAs}{\,\stackrel{\text{def}}{=}\,}

%\newcommand{\DefOR}{\ensuremath{\;\;\;\big|\;\;\;}}
\newcommand{\DefORmini}{\ensuremath{\;\big|\;}}

\newcommand{\details}[1]{}

%%%%%% Temporal Modalities %%%%%%%%%%%%%%

\newcommand{\true}{\ensuremath{\textup{\texttt{true}}}}
\newcommand{\false}{\ensuremath{\textup{\texttt{false}}}}
\newcommand{\Power}[3]{\mathrel{{#1}|{#2}\rrangle{#3}}}
\newcommand{\Norm}[1]{\ensuremath{\|{#1}\|}}
\newcommand{\WeakPower}[3]{\mathrel{{#1}|{#2}\rangle{#3}}}
\newcommand{\PastPower}[3]{\mathrel{{#1}\llangle{#2}|{#3}}}
\newcommand{\PastWeakPower}[3]{\mathrel{{#1}\langle{#2}|{#3}}}
%\newcommand{\Next}{\ensuremath{\bigcirc}}
\newcommand{\Seq}{\ensuremath{;}}
%\newcommand{\Previous}{\ensuremath{{\mathrlap{\bigcirc}\,\minus\,}}}
\newcommand{\Previous}{\Prev}
%\newcommand{\until}{{\sf U}}
\newcommand{\until}{\Until}
%\newcommand{\since}{{\sf S}}
\newcommand{\since}{\Since}
%\newcommand{\sometimes}{\lozenge}
%\newcommand{\always}{\square}
\newcommand{\always}{\Always}

\newcommand{\comp}[1]{\ensuremath{\overline{#1}}} % complement automaton
\newcommand{\dual}[1]{\ensuremath{\widetilde{#1}}} % dualization

\newcommand{\Outcm}{c\,b\,r}
\newcommand{\UOutcm}{\underline{\Outcm}}

\newenvironment{changemargin}[2]{%
  \begin{list}{}{%
    \setlength{\topsep}{0pt}%
    \setlength{\leftmargin}{#1}%
    %\setlength{\footnotewidth}{#1}%
    \setlength{\rightmargin}{#2}%
    \setlength{\textheight}{21cm}%
   \setlength{\topmargin}{0cm}
   \setlength{\listparindent}{\parindent}%
    \setlength{\itemindent}{\parindent}%
    \setlength{\parsep}{\parskip}%
  }
  \item[]}{\end{list}}


\newcommand{\push}{\textit{push}}
\newcommand{\Regions}{\ensuremath{\textit{REG}}}
\newcommand{\pop}{\textit{pop}}
%\newcommand{\pending}{\ensuremath{\texttt{pending}}}

\newcommand{\Bu}{\ensuremath{\textsf{B}}}
\newcommand{\Co}{\ensuremath{\textsf{C}}}

\newcommand{\alphaANY}{\ensuremath{\alpha_{\mathit{ONE}}}}

%%%% ATTENTION HERE: DON'T USE OLD DEFINITION IN FSTTCS CLASS iS treated as a section %%%%
\newcommand{\myparagraph}[1]{\paragraph{\textup{\textbf{#1}}}}
%\newcommand{\myparagraph}[1]{\noindent\textbf{#1}}

\newcommand{\myparagraphApp}[1]{\vspace{0.5em}\noindent\textup{\textbf{#1}}}

\newcommand{\qin}{\ensuremath{q_{\textit{in}}}}

% MINE
\newcommand{\striver}{\textsf{Striver}\xspace}

\newcommand{\tessla}{TeSSLa\xspace}

\newcommand{\reserved}[1]{\ensuremath{\mathit{#1}}\xspace}
\newcommand{\Syntax}[1]{\keyword{#1}}


\newcommand{\wakeUpTicks}{\reserved{ticks^*}}
\newcommand{\ticks}{\reserved{ticks}}

\newcommand{\caseof}[1]{\mathtt{case}~#1~\mathtt{of}}
%\newcommand{\IF}{\mathtt{if}}  %% Defined in algorithmic
%\newcommand{\THEN}{\mathtt{then}}
%\newcommand{\ELSE}{\mathtt{else}}
%\newcommand{\ELSIF}{\mathtt{elsif}}

\newcommand{\IfThenElse}[3]{\mathtt{if}~#1~\mathtt{then}~#2~\mathtt{else}~#3}

\newcommand{\IntoP}{\mathrel{\rightharpoondown}}


\newcommand{\Vc}[1]{\ensuremath{\langle #1 \rangle}}

\newcommand{\default}{\reserved{default}}
\newcommand{\lift}{\reserved{lift}}
\newcommand{\delay}{\keyword{delay}}
\newcommand{\SHIFT}{\keyword{shift}}

%\newcommand{\at}{\keyword{delayLast}}
\newcommand{\at}{\keyword{at}}

\newcommand{\PrevVal}{\reserved{valprev}}
%\newcommand{\Prev}{\reserved{tprev^{\leftarrow}}}
%\newcommand{\Prev}{\reserved{prev}}
%\DeclareMathOperator*{\prevOp}{\reserved{prev}}
\DeclareMathOperator{\prevOp}{\reserved{prev}}
\DeclareMathOperator{\succOp}{\reserved{succ}}
%\newcommand{\Prev}{\ensuremath{\ensuremath{\reserved{prev}}\limits{\leq}}}
\newcommand{\Prev}{\ensuremath{\prevOp\limits_{<}}}
\newcommand{\PrevEqVal}{\reserved{valprev_\circ}}
%\newcommand{\PrevEq}{\reserved{tprev}}
%\newcommand{\PrevEq}{\reserved{prev_\circ}}
\newcommand{\PrevEq}{\ensuremath{\prevOp\limits_{\leq}}}
\newcommand{\Succ}{\ensuremath{\succOp\limits_{>}}}
\newcommand{\SuccEqVal}{\reserved{valsucc_\circ}}
%\newcommand{\SuccEq}{\reserved{tsucc}}
%\newcommand{\SuccEq}{\reserved{succ_\circ}}
\newcommand{\SuccEq}{\ensuremath{\succOp\limits_{\geq}}}

\newcommand{\PrevS}{\Syntax{\ll}}
%\newcommand{\PrevEqS}{\Syntax{<\sim}}
%\newcommand{\PrevEqS}{\Syntax{<\text{\raisebox{10ex}{$\sim$}}}}
%\newcommand{\PrevEqS}{\ensuremath{\mathrel{\ooalign{$<$\cr\hidewidth\raise{20ex}\hbox{$\sim$}\cr}}}}
\newcommand{\PrevEqS}{\ensuremath{\mathrel{\ooalign{$<$\cr\hidewidth\raisebox{-1.2ex}{$\sim$}\cr}}}}
\newcommand{\SuccS}{\Syntax{\gg}}
%\newcommand{\SuccEqS}{\Syntax{\sim>}}
\newcommand{\SuccEqS}{\ensuremath{\mathrel{\ooalign{$>$\cr\hidewidth\raisebox{-1.2ex}{$\sim$}\cr}}}}
\newcommand{\setS}[1]{\Syntax{\{} #1 \Syntax{\}}}
\newcommand{\valS}[1]{\textsf{(}#1\textsf{)}}
\newcommand{\Ticks}{\keyword{ticks}}
\newcommand{\ticksS}{\keyword{ticks}}
\newcommand{\Union}{\mathrel{\textsf{U}}}
%\newcommand{\delaySym}{\Syntax{delay}}
%\newcommand{\delayS}[1]{\delaySym{}(#1)}
\newcommand{\delayS}[1]{\delay{}(#1)}
%\newcommand{\atSym}{\Syntax{at}}
%\newcommand{\atS}[1]{\atSym{}(#1)}
\newcommand{\atS}[1]{\at{}(#1)}


\newcounter{Problem} \setcounter{Problem}{0}

\newcommand{\Prob}{\addtocounter{Problem}{1}\noindent\paragraph{\textbf{Problem
      \#\theProblem}}}

\newcommand{\Abs}[1]{\ensuremath{#1^\#\xspace}}

\newcommand{\LET}{\mathtt{let}}
\newcommand{\MAX}{\ensuremath{\mathit{max}}}
%\newcommand{\DefOr}{\ensuremath{\hspace{1em}\big|\hspace{1em}}}


\newcommand{\tprev}{\ensuremath{t_{\textit{prev}}}}

\newcommand{\Tstar}{\ensuremath{\mathbb{T}^*}\xspace}

\newcommand{\dom}{\ensuremath{\mathit{dom}}}

%%%%%%%%%%% FROM ANOTHER PAPER ON Nested SRV

%%%%%%%%%%%%%%%%%%%%%%%%%%%%%%%%%%%%%%%%%%%%%%%%%%
%%%% Runtime Verification Macros
%%%%%%%%%%%%%%%%%%%%%%%%%%%%%%%%%%%%%%%%%%%%%%%%%%

\newcommand{\Clock}{\textit{clock}}
\newcommand{\Xclock}{\ensuremath{X_\Clock}}
\newcommand{\SRV}{SRV\xspace}
\newcommand{\NRV}{NestedSRV\xspace}

%%%% types
\newcommand{\BoolT}{\textit{Bool}}
\newcommand{\NatT}{\textit{Nat}}
\newcommand{\IntT}{\textit{Int}}
\newcommand{\StackT}{\textit{Stack}}
\newcommand{\SetT}{\textit{Set}}
\newcommand{\ElemT}{\textit{Elem}}
\newcommand{\andsym}{\texttt{and}}
\newcommand{\orsym}{\texttt{or}}
\newcommand{\notsym}{\texttt{not}}
\newcommand{\truesym}{\texttt{true}}
\newcommand{\falsesym}{\texttt{false}}
\newcommand{\SP}{\hspace{0.03em}}

\newcommand{\ifthenelsesym}{
  $\begin{bmatrix}
    \sym{if}\SP\cdot\SP\sym{then}\SP\cdot\SP\\
        \hspace{0.12em}\sym{else}\SP\cdot\SP
  \end{bmatrix}$}
\newcommand{\ifthenelsetype}[1]{
  $\begin{array}{l}\BoolT \times #1 \times #1\\
    \hspace{3em}\Into #1
  \end{array}$
}
\newcommand{\BoolName}{\textit{Booleans}}
\newcommand{\BoolSorts}{\BoolT}
\newcommand{\NatName}{\textit{Naturals}}
\newcommand{\NatSorts}{\NatT, \BoolT}
\newcommand{\SetName}{\textit{Sets}}
\newcommand{\SetSorts}{\SetT, \BoolT, \ElemT}
\newcommand{\StackName}{\textit{Stacks}}
\newcommand{\StackSorts}{\StackT, \BoolT, \ElemT}

%%%%%

\newcommand{\Sres}{S_{\text{res}}}
\newcommand{\calS}{\mathcal{S}}
\newcommand{\calF}{\mathcal{F}}
\newcommand{\NExpr}{\ensuremath{\textit{NestedExpr}}}
\newcommand{\NestedAtomic}[1]{\ensuremath{\textit{NestedOffset}(#1)}}
\newcommand{\Atomic}[1]{\ensuremath{\textit{Offset}(#1)}}
%\newcommand{\In}{\mathcal{I}}
%\newcommand{\Out}{\mathcal{O}}
\newcommand{\Dom}{\mathcal{D}}
%\newcommand{\sym}[1]{#1}
%\newcommand{\sym}[1]{\ensuremath{\mathtt{#1}}}
\newcommand{\sym}[1]{\ensuremath{\mathtt{#1}}}
\newcommand{\symIF}{\sym{if}}
\newcommand{\symTHEN}{\sym{then}\xspace}
\newcommand{\symELSE}{\sym{else}\xspace}
\newcommand{\symTRUE}{\sym{true}\xspace}
\newcommand{\symFALSE}{\sym{false}\xspace}
\newcommand{\symf}{\sym{f}}
%\newcommand{\DOM}[1]{{#1}^{\Dom}}
\newcommand{\domf}{\DOM{f}}
\newcommand{\clock}[1]{\ensuremath{\text{$#1$}}\xspace}
\newcommand{\clockn}{\clock{n}}
\newcommand{\symDIV}{\sym{div}}
\newcommand{\otau}{\overline{\tau}}
\newcommand{\osigma}{\overline{\sigma}}
\newcommand{\anext}[3]{\ensuremath{#1 \leadsto_{#3} #2}}


\newcommand{\Pred}{\ensuremath{\textit{pred}}}
% Language name
\newcommand{\NSRV}{NestedSRV\xspace}
\newcommand{\LOLA}{LOLA\xspace}


% Translation stuff
\newcommand{\tigger}{trigger\xspace}
\newcommand{\tiggers}{triggers\xspace}
\newcommand{\Tiggers}{Triggers\xspace}

% To change!!!
\newcommand{\SIC}{\ensuremath{\parallel_{\mathcal{P}}}}

\newcommand{\ctrue}{\tbf{true}}
\newcommand{\cfalse}{\tbf{false}}

%\newcommand{\default}{c}

% Specification
\newcommand{\Spec}{\varphi}

% Input variables
\newcommand{\iv}{t}
\newcommand{\ivcall}{\iv_{call}}
\newcommand{\ivbeg}{\iv_{beg}}
\newcommand{\ivend}{\iv_{end}}
\newcommand{\ivret}{\iv_{ret}}
% Output variable
\newcommand{\ov}{s}
% Any stream variable 
\newcommand{\asv}{v}


% Output stream
\newcommand{\os}{\sigma}
% Input streams
\newcommand{\is}{\tau}
\newcommand{\iscall}{\is_{call}}
\newcommand{\isbeg}{\is_{beg}}
\newcommand{\isend}{\is_{end}}
\newcommand{\isret}{\is_{ret}}
% Any stream
\newcommand{\as}{\sigma}

% Abstract sum over execution model
\newcommand{\asum}[1]{\oplus_{#1}}
\newcommand{\aminus}[1]{\ominus_{#1}}

% Guess
\newcommand{\G}{\mathcal{G}}

\newcommand{\gaprev}[1]{\ensuremath{#1 . \ttt{prev}_{abs}}}
\newcommand{\gcprev}[1]{\ensuremath{#1 . \ttt{prev}_{con}}}
\newcommand{\gnow}[1]{\ensuremath{#1 . \ttt{now}}}
\newcommand{\gcnext}[1]{\ensuremath{#1 . \ttt{next}_{con}}}
\newcommand{\ganext}[1]{\ensuremath{#1 . \ttt{next}_{abs}}}

% Input
\newcommand{\inp}{\sigma}
\newcommand{\Inp}{\Sigma}

% Execution model
\newcommand{\exm}{\eta}
\newcommand{\Exm}{H}

% Automata
\newcommand{\LA}[1]{\mathcal{L}(#1)}

% Path
\newcommand{\path}{\alpha}

% Common types
\newcommand{\FNat}{\Nat_\bot}

% Brackets, tuples
\newcommand{\Bracket}[1]{\llbracket #1 \rrbracket}
\newcommand{\Tup}[1]{\ensuremath{\langle #1 \rangle}}

% Felipe's notes
\newcommand{\felitodo}[1]{\todo[linecolor=red,backgroundcolor=red!25,bordercolor=red]{Felipe: #1}}
\newcommand{\felitext}[1]{\todo[inline,linecolor=red,backgroundcolor=red!25,bordercolor=red]{Felipe: #1}}


\definecolor{darkblue}{rgb} {0.200, 0.200, 0.800}
\definecolor{darkred} {rgb} {0.800, 0.200, 0.200}
\definecolor{darkgreen}{rgb}{0.200, 0.800, 0.200}

\newcommand{\colorEv}[2]{\ensuremath{\textcolor{#2}{\llbracket} #1
    \textcolor{#2}{\rrbracket}}}
\newcommand{\colorSigmaEval}[3]{\ensuremath{\colorEv{#1}{#3}_{#2}}}
\newcommand{\colorEval}[2]{\colorEv{#1}{#2}}
\newcommand{\colorEvalSigma}[2]{\colorSigmaEval{#1}{\sigma}{#2}}
%\newcommand{\evalSpec}[1]{\colorSigmaEval{#1}{darkblue}}
\newcommand{\evalSpec}[1]{\colorEval{#1}{black}}
\newcommand{\evalSpecsigma}[2]{\colorEval{#1}{#2}{black}}
\newcommand{\evalE}[1]{\colorEval{#1}{darkred}}
\newcommand{\evalEsigma}[2]{\colorSigmaEval{#1}{#2}{darkred}}
%\newcommand{\evalTau}[1]{\colorSigmaEval{#1}{darkgreen}}
\newcommand{\evalTau}[1]{\colorEvalSigma{#1}{darkblue}}
\newcommand{\evalTauN}[1]{\colorEval{#1}{darkblue}}
\newcommand{\evalTausigma}[2]{\colorSigmaEval{#1}{#2}{darkblue}}
\newcommand{\evalTicks}[1]{\colorEvalSigma{#1}{brown}}
\newcommand{\evalTicksN}[1]{\colorEval{#1}{brown}}
\newcommand{\evalTickssigma}[2]{\colorSigmaEval{#1}{#2}{brown}}
\newcommand{\presentLabel}{\ensuremath{\mathcal{O}}}
\newcommand{\pastLabel}{\ensuremath{\prec}}
\newcommand{\edgesSpec}{\ensuremath{Edges_\spec}}
\newcommand{\interp}{\mathcal{I}}
\newcommand{\Val}{\mathit{V}}

%\newcommand{\undef}{\reserved{undef}}

\newcommand{\specialbot}[2]{\ensuremath{\bot^{#1}_{\text{#2}}}\xspace}
\newcommand{\untypedspecialbot}[1]{\ensuremath{\bot_{\text{#1}}}\xspace}
\newcommand{\D}{D}

% \newcommand{\maybe}[2]{\ensuremath{\mathtt{Maybe}}^#1_{\text{#2}}}
\newcommand{\T}{\ensuremath{\mathbb{T}}\xspace}
\newcommand{\outT}{\specialbot{\T}{outside}}
\newcommand{\outD}{\specialbot{\D}{outside}}
\newcommand{\notickD}{\specialbot{\D}{notick}}
\newcommand{\outDS}{\Syntax{outside}_\D}
\newcommand{\outTS}{\Syntax{outside}_{\T}}
\newcommand{\notickDS}{\Syntax{notick}_\D}
\newcommand{\maybe}[2]{\ensuremath{#1 \cup \{\specialbot{#1}{#2}\}}}
% \newcommand{\outT}{\untypedspecialbot{outside}}
% \newcommand{\outD}{\untypedspecialbot{outside}}
% \newcommand{\notickD}{\untypedspecialbot{notick}}
% \newcommand{\outDS}{\Syntax{outside}}
% \newcommand{\outTS}{\Syntax{outside}}
% \newcommand{\notickDS}{\Syntax{notick}}

\newcommand{\maybeOutT}{\maybe{\T}{outside}}
\newcommand{\maybeOutD}{\maybe{D}{outside}}
\newcommand{\maybeNotickD}{\maybe{D}{notick}}
\newcommand{\undef}{\reserved{undef}} % Might be confusing with bottom

\newcommand{\entails}{\vdash}
\newcommand{\nonZenoT}{2^{\T}_{nz}}
\newcommand{\offsetEps}{\T_\epsilon}

\newcommand{\lambdatS}[1]{\Syntax{(\text{\textlambda} t . }#1\Syntax{)}}
\newcommand{\evalAlphaT}[1]{\colorSigmaEval{#1}{black}\alpha_t}
\newcommand{\evalAlpha}[1]{\colorSigmaEval{#1}{black}\alpha}

\newcommand{\Eval}{\mathit{Eval}\xspace}
\newcommand{\EvalPhi}{\ensuremath{\Eval_\varphi}\xspace}

% Processors
\newcommand{\colorboxnode}[2]{%
    \colorlet{currentcolor}{.}%
    {\color{#1}%
    \fbox{\color{currentcolor}\strut \ensuremath{#2}}}%
    \xspace
}
\newcommand*{\tickNode}[1]{\colorboxnode{green}{#1}}
\newcommand*{\valueNode}[1]{\colorboxnode{blue}{#1}}
\newcommand*{\gateNode}[1]{\colorboxnode{black}{#1}}

\newcommand*{\delayTickNode}{\tickNode{delay}}
\newcommand*{\srcTickNode}{\tickNode{srcTick}}
\newcommand*{\constTickNode}{\tickNode{constTick~c}}
\newcommand*{\UTickNode}{\tickNode{UTick}}

\newcommand*{\prevNode}{\valueNode{prev}}
\newcommand*{\prevEqNode}{\valueNode{prevEq}}
\newcommand*{\prevValNode}{\valueNode{prevVal}}
\newcommand*{\prevEqValNode}{\valueNode{prevEqVal}}
\newcommand*{\funcNode}{\valueNode{func~f}}
\newcommand*{\tNode}{\valueNode{\mathbf{t}}}

\newcommand*{\coreNode}{\gateNode{core}}
\newcommand*{\inputNode}{\gateNode{in}}

\newcommand*{\StreamT}[1]{\ensuremath{\mathcal{S}_{#1}}}
\newcommand*{\StreamProgressT}[1]{\ensuremath{\mathcal{S}^{\progressEv}_{#1}}}
\newcommand*{\StreamUnit}{\StreamT{Unit}}
\newcommand*{\StreamProgressUnit}{\StreamProgressT{Unit}}

% Processor functions
\newcommand*{\voteFun}{\ensuremath{vote}\xspace}
\newcommand*{\execFun}{\ensuremath{exec}\xspace}
\newcommand*{\rinseFun}{\ensuremath{rinse}\xspace}
\newcommand*{\mayTickFun}{\ensuremath{mayTick}\xspace}
\newcommand*{\computeValFun}{\ensuremath{computeVal}\xspace}

% Progress events data
\newcommand*{\progressEv}{\untypedspecialbot{progress}}

% Translation phase
\newcommand*{\signalFun}{\ensuremath{signalNode}\xspace}
\newcommand*{\translateTicks}[1]{\ensuremath{translateTicks(#1)}\xspace}
\newcommand*{\translateSpec}{\ensuremath{translateSpec}\xspace}
\newcommand*{\translateVal}[1]{\ensuremath{translateVal(#1)}\xspace}

% System state
\newcommand*{\quiescT}{\ensuremath{\T \cup \{-\infty\}}\xspace}
\newcommand*{\globalSystem}{\ensuremath{S}\xspace}
\newcommand*{\graph}{\ensuremath{G}\xspace}
\newcommand*{\graphT}{\ensuremath{\tupleof{V,E}}\xspace}
\newcommand*{\eStates}{\ensuremath{\Sigma_E}\xspace}
\newcommand*{\vStates}{\ensuremath{\Sigma_V}\xspace}
\newcommand*{\inputTagT}{\ensuremath{InTag}\xspace}
\newcommand*{\nodeStateT}{\ensuremath{nodeState}\xspace}
\newcommand*{\edgeStateT}{\ensuremath{edgeState}\xspace}
\newcommand*{\vStateFunT}{\ensuremath{(V \to \nodeStateT)}\xspace}
\newcommand*{\eStateFunT}{\ensuremath{(V \times \inputTagT \to \edgeStateT)}\xspace}
\newcommand*{\dynSys}{\ensuremath{\Sigma}\xspace}
\newcommand*{\dynSysT}{\ensuremath{\tupleof{\vStateFunT, \eStateFunT}}\xspace}
\newcommand*{\portTag}[1]{\ensuremath{\mathtt{#1}}\xspace}

% Channel functions
\newcommand*{\writeChansFun}{\ensuremath{writeChannels}\xspace}
\newcommand*{\readChansFun}{\ensuremath{readChannels}\xspace}
\newcommand{\emptyChannel}{\specialbot{\D}{emptychannel}}

% Operational semantics
\newcommand{\ticksDef}[1]{\mathcal{E}^{ticks}_{#1}}
\newcommand{\valsDef}[1]{\mathcal{E}^{vals}_{#1}}



\newcommand{\colorHEval}[2]{\colorSigmaEval{#1}{H}{#2}}
\newcommand{\evalEH}[1]{\colorHEval{#1}{darkred}}
\newcommand{\evalTicksH}[1]{\colorHEval{#1}{brown}}

\newcommand{\opEval}[5]{\colorSigmaEval{#4}{#5}{#2\vert_{#1,#3}}(#3)}
\newcommand{\mayTick}[4]{\opEval{#2}{#3}{#4}{#1}{brown}}
\newcommand{\computeVal}[4]{\opEval{#2}{#3}{#4}{#1}{darkred}}
\newcommand{\computeTau}[4]{\opEval{#2}{#3}{#4}{#1}{darkblue}}

% Dependency graph
\newcommand{\ddep}{\rightarrow}
\newcommand{\ddepPres}{\xrightarrow{0}}
\newcommand{\ddepPresStar}{\xrightarrow{\smash{0}}^*}

\newcommand{\ddepPast}{\xrightarrow{-}}
\newcommand{\ddepFut}{\xrightarrow{+}}

\newcommand{\Present}{\textit{Present}}
\newcommand{\Past}{\textit{Past}}
\newcommand{\Future}{\textit{Future}}

%%%% For algorithmic

% New definitions
\algnewcommand\algorithmicswitch{\textbf{switch}}
\algnewcommand\algorithmiccase{\textbf{case}}
\algnewcommand\algorithmicassert{\texttt{assert}}
\algnewcommand\Assert[1]{\State \algorithmicassert(#1)}%
% New "environments"
\algdef{SE}[SWITCH]{Switch}{EndSwitch}[1]{\algorithmicswitch\ #1\ \algorithmicdo}{\algorithmicend\ \algorithmicswitch}%
\algdef{SE}[CASE]{Case}{EndCase}[1]{\algorithmiccase\ #1}{\algorithmicend\ \algorithmiccase}%
\algtext*{EndSwitch}%
\algtext*{EndCase}%

%%%
\newcommand{\SUP}{\ensuremath{\mathit{sup}}}

\newcommand{\tin}{\ensuremath{t_\textit{in}}}
%\newcommand{\tqin}{\ensuremath{t_q^{\textit{in}}}}
\newcommand{\tqin}{\tin}
\newcommand{\tout}{\ensuremath{t_\textit{out}}}
%\newcommand{\tqout}{\ensuremath{t_q^{\textit{out}}}}
\newcommand{\tqout}{\tout}
\newcommand{\append}{\textit{append}}
\newcommand{\tq}{\ensuremath{t_q}}

\newcommand{\App}{\ensuremath{\mathrel{\scalebox{0.8}{++}}}}

\newcommand{\PrevStt}[1]{\ensuremath{\textbf{\texttt{<<}}~#1}}
\newcommand{\PrevEqStt}[1]{\ensuremath{\textbf{\texttt{<\~}}~#1}}

\newcommand{\SY}[1]{\text{\texttt{#1}}}
\newcommand{\PrevEqV}[1]{\SY{(\~}#1\SY{)}}
\newcommand{\PrevV}[1]{\SY{(<}#1\SY{)}}
\newcommand{\SuccEqV}[1]{\SY{(}#1\SY{~)}}
\newcommand{\SuccV}[1]{\SY{(}#1\SY{>)}}


\title{Hashchain: una implementación eficiente de Setchain construída sobre Tenderminmt}

\author{Gabina Luz Bianchi}
% \institute{Facultad de Ciencias Exactas, Ingeniería y Agrimensura}


%\date{\today}

\begin{document}

\maketitle

\tableofcontents{}

\input{Abstract}

% \begin{keywords}
% 	Blockchain, Setchain, Scalability, Tendermint.
% \end{keywords}

% %% Introduction:
% %% Introducing the problem to the reader and motivating the work
% \section{Objetivo del capítulo}
En este capítulo se presenta el problema a abordar a lo largo del trabajo, junto con un breve
marco teórico, analizando el estado del arte y comentando las principales contribuciones de esta
tesina.

\section{Historia}
% \section{Disclaimer}

% Es por acá, amiga...
% Usuarias de instagram alrededor de Argentina
% creyendo saber por dónde va

A lo largo de la historia, el término \textit{bitácora} ha servido para referir a
distintos objetos relacionados con la orientación, el orden y el registro. Por
primera vez aparecida en un texto escrito en 1538,
%% Fuente: El diccionario etimológico de la lengua castellana de Joan Coromines
la palabra \textit{bitácora}
oficialmente refiere a un concepto de la navegación: \textit{especie de armario
inmediato al timón, en el cual se coloca la brújula}~\cite{dic.Etimologico}.
Desde ahí surge también
el conocido \textit{cuaderno de bitácora}, un libro en el que los marinos, durante
sus guardias, registraban los datos de lo acontecido, y que se guardaba en el
interior de la bitácora para preservarlo de los malos tiempos. El concepto
evolucionó y hoy en día la palabra bitácora usualmente se utiliza para hablar
de registros metodológicos de un suceso particular, ya sea un viaje, una
construcción, etc. Naturalmente, también se transformó la forma en que estos
registros se escriben y dónde se guardan, y existe hoy la noción de \textit{bitácora
virtual}. En esos casos, son las computadoras, las redes y los protocolos,
los que permiten el acceso a esos registros mediante internet; la potencia de la
metáfora garantizando que continúe siendo un concepto de la navegación.
%

En Ciencias de la Computación, las \textit{bitácoras distribuidas} son un tipo de
base de datos que se comparte
alrededor de múltiples lugares, países o instituciones, y que típicamente es de
acceso público. Los registros pueden ser guardados usando distintas estructuras,
y solo pueden agregarse cuando los participantes logran un \textit{quórum}. En
contraste con un sistema centralizado, las bitácoras distribuidas no requieren
un administrador central y, en consecuencia, no tienen un punto de falla central.
Un caso particular de bitácora distribuida, en donde los registros se empaquetan
en forma de bloques, es la \textit{cadena de bloques} o \textit{blockchain}.

Las blockchains tomaron
popularidad con la implementación de Bitcoin~\cite{nakamoto06bitcoin},
una tecnología propuesta por Nakamoto en 2008.
Allí se presentó como un método para eliminar terceras partes confiables en sistemas
de pago electrónico.
%

Las versiones más modernas de blockchains incorporan contratos inteligentes o
\textit{smart contracts}~\cite{szabo96smart,wood2014ethereum},
los cuales son programas inmutables alojados en la blockchain. Dichos
programas describen la funcionalidad de las transacciones, incluyendo el intercambio
de criptomonedas.
%
Los contratos inteligentes permiten describir funcionalidades sofisticadas, habilitando
diversas aplicaciones en finanzas descentralizadas\footnote{En diciembre de 2021,
el valor monetario alojado en finanzas descentralizadas (o \emph{DeFi}, abreviación de \emph{decentralized finance})
estaba estimado en alrededor de \$100B, de acuerdo a Statista
\url{https://www.statista.com/statistics/1237821/defi-market-size-value-crypto-locked-usd/}.},
gobierno descentralizado, Web3, etc.
%

Conceptualmente, la blockchain es un \textit{objeto distribuido}
que contiene las transacciones realizadas
en nombre de los usuarios, empaquetadas en bloques, y totalmente
ordenadas~\cite{anta2018formalizing,anta2021principles}.
%
En entornos reales, el objeto blockchain es mantenido por múltiples servidores
sin una autoridad central, usando un \emph{algoritmo de consenso}, uno de los componentes
principales de las blockchains.
%

\section{Evolución de las blockchains}
Si bien las blockchains son un campo de estudio relativamente joven, desde 2014 se publicaron en promedio 250 artículos por año
en esta temática.
%
A su vez, la mayor parte del trabajo de blockchain se publica para plataformas industriales (sin necesidad de revisión
por pares) o directamente como \textit{white papers}~\cite{consensus.blockchain}.
%
Por este motivo, la recopilación del universo blockchain es una tarea ardua
y escapa al alcance de este trabajo.
%
Sin embargo, es interesante destacar que, a lo largo de la evolución de las blockchains, se han desarrollado distintos
criterios de clasificación basados en sus características principales.
%
A continuación se presentan algunos de estos criterios, como un panorama general de diferentes aspectos
relevantes en el análisis de las blockchains.
Los conceptos pertinentes para este trabajo serán explicados en detalles en las siguientes secciones y
capítulos de este informe.

En cuanto a su algoritmo de consenso, las blockchains pueden usar protocolos \textit{probabilísticos}
como el esquema \emph{proof-of-work}\footnote{La noción de
\emph{proof-of-work} (PoW) refiere a situaciones en las que un \emph{probador} demuestra a un \emph{verificador}
que ha realizado una cierta cantidad de trabajo computacional en un intervalo de tiempo especificado.
Los PoWs han servido como base de diversos protocolos de seguridad en la literatura~\cite{Jakobsson1999ProofsOW}.
Su traducción al español se conoce como \emph{prueba de trabajo}.} de Bitcoin,
o protocolos de consenso de la familia BFT (\textit{Byzantine Fault Tolerant}).
Estos últimos suelen ser \textit{deterministas} y emplear esquemas \emph{proof-of-stake}\footnote{El concepto de \emph{proof-of-stake}
(traducido al español
como \emph{prueba de participación}) surge como oposición a los esquemas de prueba de trabajo.
En estos tipos de protocolos, en lugar de probarse que se realizó un esfuerzo computacional lo suficientemente
costoso, el \emph{probador} demuestra la propiedad de algún bien valioso, en general, alguna criptomoneda~\cite{King2012PPCoinPC}.}
~\cite{consensus.blockchain}.
%
Mientras que los protocolos probabilísticos ofrecen solo \textit{consistencia débil}, ya que los nodos pueden terminar teniendo
diferentes vistas de las blockchains, los procolos BFT clásicos ofrecen \textit{consistencia fuerte}~\cite{Vukolic2016EventuallyRT}.

Las blockchains también pueden clasificarse como \textit{permissioned} o \textit{permissionless}.
En los sistemas \textit{permissioned} las identidades de todos los nodos son conocidas,
es decir, solo pueden acceder nodos que tengan permiso para hacerlo.
Por otra parte, los sistemas \textit{permissionless}
son completamente descentralizados: cualquiera puede correr un nodo y unirse a la red cuando lo desee.

Otra posible clasificación distingue si la red sobre la que corre la blockchain es síncrona, asíncrona o
parcialmente síncrona~\cite{sync.network.consensus}.
%
En una red síncrona los mensajes pueden sufrir retrasos acotados superiormente por un tiempo $\Delta$.
Por otro lado, en una red asíncrona los mensajes pueden ser retrasados arbitrariamente, y no existe
una cota confiable $\Delta$ para este retraso.
Por último, las redes parcialmente síncronas asumen que la red tarde o temprano será
síncrona, pese a potenciales períodos largos de asincronía.

Independientemente de las características particulares que puede presentar una tecnología blockchain,
hay conceptos fundamentales que permanecen centrales en todas sus variantes, como lo son
la escalabilidad y el rendimiento.
%
En la siguiente sección se discuten estos aspectos.

\section{El problema}
% What is the problem. Justifiy that the problem is a problem.

%Independientemente del tipo de blockchain, 
% The major hurdle in the widespread adoption of blockchains
% is their performance and scalability—while improvements
% have been made, they are nowhere near as ubiquitous as
% their traditional counterparts. These properties are deeply
% related to the consensus protocol

Actualmente, uno de los principales obstáculos para la adopción rápida y generalizada
de las tecnologías blockchain es su límite en la \emph{escalabilidad}, debido principalmente a los límites
de \emph{rendimiento} inherentes a los algoritmos de consenso
~\cite{Tyagi@BlockchainScalabilitySol,Croman2016ScalingDecentralizedBlockchain}.

En términos generales, la escalabilidad de un sistema es su habilidad para gestionar una
cantidad creciente de trabajo.
%
Usualmente se habla de escalabilidad \emph{horizontal} cuando se añaden nodos de trabajo a un sistema,
en contraposición a la escalabilidad \emph{vertical} que refiere a aumentar los recursos de un nodo particular.

En el caso específico de la tecnología blockchain, la escalabilidad horizontal consiste en añadir nuevos
nodos a la red que participen del protocolo de consenso.
%
Por otro lado, la escalabilidad vertical en la blockchain implica aumentar el poder de cómputo de los
nodos existentes en la red, mediante la actualización de sus capacidades de hardware, como la mejora de
GPUs o la ampliación de almacenamiento.
% Por otro lado, un ejemplo de problema de escalabilidad vertical en este contexto es la creciente cantidad de datos
% que requiere almacenar un nodo en la blockchain para guardar transacciones desde el bloque más reciente
% hasta el bloque \emph{génesis} (inicial).

En este trabajo nos enfocaremos en la escalabilidad de blockchains desde el punto de vista de su \emph{rendimiento}, medido
en términos del número de transacciones añadidas por segundo.
%
Algunos números representativos de hoy en día ilustran el estado actual de situación.
Al sistema Bitcoin~\cite{nakamoto06bitcoin} le toma 10 minutos o más
confirmar transacciones, logrando un rendimiento máximo de 7 transacciones por segundo.
%
Ethereum~\cite{wood2014ethereum}, una de las blockchains más populares, admite en promedio 20 transacciones por segundo.
%
Estos números son muy bajos comparados con los que observamos en otros campos de pago convencionales,
como PayPal que soporta hasta 200 transacciones por segundo, o Visa, que confirma y procesa 2000 transacciones
por segundo en promedio~\cite{tps.numbers}.
%
Por lo tanto, el rendimiento de las blockchains es un aspecto clave en la adopción de esta tecnología.
%

%  from seven transactions per second to thousands of transactions per second to handle real-life problems in the field of visa, healthcare, flights, etc
% In bitcoin Blockchain 7 transactions are carried out in one
% second ,which is very low compared to VISA and PayPal.
% The time required to confirm the transaction is around 10
% minutes and the size of each block is around 1 MB . 7
% transactions per second is obtained by dividing the
% maximum size of block by an average size of each bitcoin
% transaction as 250 bytes.

% Today’s representative blockchain such as Bitcoin takes 10 min or longer
% to confirm transactions, achieves 7 transactions/sec maximum throughput. In
% comparison, a mainstream payment processor such as Visa credit card confirms a
% transaction within seconds, and processes 2000 transactions/sec on average, with
% a peak rate of 56,000 transactions/sec

% Bitcoin processes 7 transactions per second and Etherium
% processes 20 transactions per second ,whereas payment by
% VISA is 2000 and PayPal is 200 .
% Scaling can be achieved by including more transactions in a
% block .

% Un aspecto clave en la adopción de las tecnologías blockchain es \emph{su rendmiento}, medido
% en términos del número de transacciones añadidas por segundo.
%

\section{Estado del arte}
Diversas técnicas están siendo desarrolladas para incrementar el rendimiento de las blockchains y
aumentar la cantidad de transacciones añadidas por segundo.
%
A continuación se exploran distintas propuestas, intentando abarcar enfoques novedosos
aplicados a diferentes tecnologías blockchains.

\subsection{Desarrollo de algoritmos más rápidos}

En los sistemas blockchains existen múltiples parámetros que se tienen en cuenta
a la hora de querer calcular o mejorar su rendimiento.
%
FastChain~\cite{fastchain} es un enfoque para mejorar la escalabilidad de los sistemas blockchains
(originalmente diseñado para aquellos que utilizan \emph{proof-of-work})
que se basa en reducir el \emph{tiempo de propagación de los bloques}.
%
El algoritmo propuesto por FastChain trabaja sobre la red de pares de la blockchain.

% En contextos de blockchains que utilizan \emph{proof-of-work}\footnote{La noción de \emph{proof-of-work} (PoW)
% refiere a situaciones en las que un \emph{probador} demuestra a un \emph{verificador}
% que ha realizado una cierta cantidad de trabajo computacional en un intervalo de tiempo especificado.
% Los PoWs han servido como base de diversos protocolos de seguridad en la literatura~\cite{Jakobsson1999ProofsOW}.
% Su traducción al español se conoce como \emph{prueba de trabajo}.},
% como Bitcoin, el \emph{ratio de bloque}
% se define como el número de bloques minados por segundo.
% %
% El \emph{tamaño de bloque} se define como el número de bytes en un bloque.
% %
% Toma tiempo transmitir un bloque sobre la red de pares.
% %
% El \emph{tiempo de propagación de bloque} se define como el tiempo necesario para que un minero
% reciba un nuevo bloque.
% %
% Dado que un bloque es recibido por diferentes mineros en diferentes momentos, usualmente se utiliza
% la \emph{distribución del tiempo de propagación de bloques}.
% %
% Si se mina un bloque conflictivo mientras otro bloque se propaga en la red, ocurre una bifurcación
% (o \emph{fork}) en la red.
% %
% Los mineros resuelven las bifurcaciones en la blockchain eligiendo la cadena que contiene la mayor
% cantidad de bloques.
% %
% Los bloques no incluidos en la cadena más larga se descartan.
% %
% El \emph{ratio de bifurcación} se define como el número de bloques descartados por segundo.

% Para mejorar el rendimiento de la blockchain no se puede simplemente incrementar el tamaño del bloque
% o minar bloques con mayor frecuencia.
% %
% Hay un triple compromiso entre el tamaño del bloque, el ratio de bloques y el ratio de bifurcación.


%
En la Figura \ref{fig:fastchain} se muestra un ejemplo motivador para la lógica detrás de FastChain.
%
El nodo $A$ tiene 10Mbps de ancho de banda.
%
El nodo $B$ tiene 100Mbps.
%
Todos los enlaces tienen 200ms de latencia.
%
La Figura \ref{subfig:fastchain-a} corresponde a una política de selección de vecinos aleatoria.
%
Los nodos $A$ y $B$ tienen 5 vecinos.
%
La Figura \ref{subfig:fastchain-b} muestra una política de selección de vecinos
con información de ancho de banda.
%
El nodo $A$ se conecta únicamente con el nodo $B$, mientras que el nodo $B$ se conecta
a todos los otros.
%
Supongamos que el nodo $A$ mina un bloque de tamaño $10^6$ bytes.
%
En la Figura \ref{subfig:fastchain-a}, el ancho de banda del nodo $A$ se reparte entre las 5 conexiones.
%
Cada conexión tiene un ancho de banda de 2Mbps.
%
Por lo tanto, a sus vecinos les toma 4.2s recibir el nuevo bloque.
%
En la Figura \ref{subfig:fastchain-b}, el nodo $A$ primero le transmite el bloque a $B$ en 1s.
%
El nodo $B$ luego transmite el bloque al resto de los nodos en 0.52s.
%
El tiempo de propagación de bloque promedio en la topología de FastChain es 1.416s,
casi 3 veces menor que su contraparte para la topología de selección de vecinos aleatoria.
%
El nodo $A$ es el cuello de botella para la propagación de bloques.
%
Mientras el nodo $A$ está transmitiendo el bloque, otros nodos con mayor ancho de banda
están ociosos.
%

\begin{figure}
  \centering
  \subfloat[Política aleatoria.]{\includegraphics[scale=0.1]{figures/fastchain0.png}\label{subfig:fastchain-a}}
  \hspace{1.5cm}
  \subfloat[Política de FastChain.]{\includegraphics[scale=0.1]{figures/fastchain1.png}\label{subfig:fastchain-b}}
  \caption{Selección de vecinos. Ejemplo motivador para FastChain}
  \label{fig:fastchain}
\end{figure}

%
FastChain propone un algoritmo de selección del mejor vecino para reducir el tiempo de propagación de los bloques.
%
De esta forma, los nodos se desconectan de los vecinos con ancho de banda limitado y
favorecen a los nodos con mayor ancho de banda.
%
%Esta política de selección de vecinos escala el sistema blockchain,
%reduciendo el tiempo de propagación de bloque sin incrementar el ratio de bifurcación.
%


% system. We observe that
% reducing the block propagation time improves the scalability of
% the blockchain without increasing the forking rate. We propose
% FastChain to scale the effective block rate of blockchain systems.
% FastChain reduces the block propagation time through its
% informed neighbor selection policy.
% Miners periodically refresh
% their connections and disconnect from bandwidth-limited
% neighbors.
% We evaluate the performance of FastChain through
% extensive simulations.
% FastChain increases effective block rate
% by up to 40%. Network topologies with shorter average link
% latency can benefit significantly from FastChain. FastChain is
% effective when nodes have different mining powers. When the
% mining power of a node is inversely proportional to the node
% bandwidth, the advantage of FastChain is the largest.

\subsection{\emph{Inter-Blockchain Communication}}
La tecnología blockchain está creciendo masivamente; el número de plataformas y
aplicaciones descentralizadas se incrementó rápidamente en los últimos años.
%
Sin embargo, la mayoría de las redes de blockchain operan en entornos autónomos aislados
de los demás.
%
La \textit{inter-operabilidad} de blockchains es la habilidad de conectar múltiples redes de
blockchain entre sí, lo cual puede constituir un enfoque para mejorar la escalabilidad en
las plataformas.
%
De esta forma múltiples blockchains paralelas pueden operar entre ellas, manteniendo las
propiedades de seguridad de las mismas.

Una arquitectura novedosa de redes de blockchains es Cosmos~\cite{cosmos}, que conecta diversas
blockchains independientes, llamadas \emph{zonas}.
%
Las zonas funcionan con Tendermint Core~\cite{Buchman.2018.Tendermint}, un motor de consenso de alto rendimiento.
%
El algoritmo de consenso del Tendermint Core es adecuado para escalar blockchains públicas que
trabajan con modelos \emph{proof-of-stake}.
%
Sin embargo, blockchains con modelos de consenso \emph{proof-of-work}, como Bitcoin,
se pueden conectar a la red de Cosmos utilizando adaptadores de zonas.

% Blockchain technology is growing massively where the number of
% blockchain platforms and decentralized applications are increasing
% rapidly in the last years. However, most of the existing blockchain
% networks are operating in a standalone environment isolated from
% each other, which increases scalability and connectivity issues in
% the current blockchain platforms as well as limiting the
% blockchain adoption in industry ecosystems. In the current phase,
% different blockchain networks don’t have mutual trust where they
% cannotot interact with each other and their capacity level has only
% reached a level similar to LAN. Due to the high barriers between
% the independent isolated blockchain platforms, researchers have
% started to focus on the concept of Blockchain interoperability.
% Blockchain interoperability is the ability of connecting multiple
% blockchain networks together, which significantly increases and
% solves scalability and connectivity issues in the blockchain
% platforms.
% An ideal solution is one that allows multiple parallel blockchains to interoperate while retaining their security properties.

% Here we present Cosmos, a novel blockchain network architecture that addresses all of these problems.
% Cosmos is a network connecting many independent blockchains, called zones. The zones are powered by Tendermint Core [8],
% which provides a high-performance, consistent, secure PBFT-like consensus engine, where strict fork-accountability guarantees
% hold over the behaviour of malicious actors. Tendermint Core's BFT consensus algorithm is well suited for scaling public
% proof-of-stake blockchains. Blockchains with other consensus models, including proof-of-work blockchains like Ethereum and
% Bitcoin can be connected to the Cosmos network using adapter zones.

La primera zona de Cosmos se llama \emph{Cosmos Hub}.
%
Conecta varias blockchains (o zonas) mediante un protocolo de comunicación entre blockchains, conocido en inglés
como \emph{inter-blockchain communication} (IBC).
Funciona como un protocolo TCP para blockchains\footnote{El protocolo de control de transmisión conocido como
TCP (proveniente de \emph{Transmission Control Protocol}) es uno de los protocolos fundamentales de internet,
perteneciente a la capa de transporte.}.
%
Gestiona numerosos tipos de \emph{tokens} y mantiene un registro del número total de los mismos en cada zona conectada.
%
Los tokens pueden transferirse de una zona a otra de forma segura y rápida sin necesidad de un intercambio \emph{líquido} entre zonas.
%
En su lugar, todas las transferencias de tokens inter-blockchains pasan a través del Cosmos Hub.
%
En la Figura \ref{fig:cosmos} se ilustra un ejemplo simplificado de esta arquitectura.

\begin{figure}[!h]
  \centering
  \includegraphics[scale=0.12]{figures/cosmos.png}
  \caption{Arquitectura simplificada de Cosmos.}
  \label{fig:cosmos}
\end{figure}

La interoperabilidad de blockchains puede contribuir a la escalabilidad del sistema, debido a que 
la carga de procesamiento se distribuye, evitando cuellos de botella en una sola red.
%
A su vez, al operar varias blockchains en paralelo, el número total de transacciones que
pueden ser procesadas simultáneamente aumenta significativamente.

% Esta arquitectura resuelve muchos problemas a los que se enfrenta actualmente el espacio blockchain, como la interoperabilidad de las aplicaciones,
% escalabilidad y capacidad de actualización sin fisuras. Por ejemplo, las zonas derivadas de Bitcoind, Go-Ethereum, CryptoNote, ZCash, o
% cualquier sistema de cadena de bloques. Estas zonas permiten a Cosmos escalar infinitamente para satisfacer la demanda global de transacciones.
% demanda global de transacciones. Las zonas son también un gran ajuste para un intercambio distribuido, que será apoyado también.

% Cosmos no es sólo un libro mayor distribuido, y el Hub Cosmos no es un jardín amurallado o el centro de su universo.
% Estamos diseñando un protocolo para una red abierta de libros de contabilidad distribuidos que puede servir como una nueva base para
% futuros sistemas financieros,
% basado en principios de criptografía, economía sólida, teoría del consenso, transparencia y responsabilidad.

% A constant stream of recent block commits from zones posted on the Hub allows the Hub to keep up with the state of each zone.
% Likewise, each zone keeps up with the state of the Hub (but zones do not keep up with each other except indirectly through the Hub).
% Packets of information are then communicated from one zone to another by posting Merkle-proofs as evidence that the information was
% sent and received. This mechanism is called inter-blockchain communication, or IBC for short.


\subsection{Paralelismo}
Los sistemas de base de datos tradicionales logran escalabilidad dividiendo
los estados de la base de datos en fragmentos independientes (o particiones).
%
Al distribuir la carga sobre múltiples particiones, la capacidad global del sistema
se incrementa.
%
Esta técnica, conocida como \emph{sharding}~\cite{Dang2019Sharding}, requiere cierta coordinación para lograr propiedades
básicas de atomicidad, consistencia, aislamiento y durabilidad para aquellas transacciones
que acceden a múltiples fragmentos.
%
%En la Figura ~\ref{fig:sharding} se representa la idea de \emph{sharding} como la combinación entre
%réplica y partición.
%
La idea de \emph{sharding} se puede pensar como la combinación entre réplica y partición.
%
Cada partición se replica sobre múltiples réplicas, y su contenido se mantiene consistente
mediante protocolos de consenso.

En el contexto de blockchains, el enfoque de \emph{sharding} se basa en aplicar este concepto
originario de la escalabilidad de bases de datos, dividiendo la red de blockchain en comités más
pequeños de modo de reducir la sobrecarga de los protocolos de consenso.
%
Sin embargo, este concepto no puede ser directamente extendido a los sistemas de blockchain, debido
a diferencias fundamentales en los modelos de fallas que se consideran en las bases de datos
y en las blockchains.
%
Tradicionalmente, las bases de datos asumen modelos de fallo por caída, en los cuales un nodo que falla
simplemente deja de enviar y responder peticiones.
%
Por el otro lado, los sistemas de blockchain operan en ambientes más hostiles, muchas veces asumiendo modelos
de fallas más fuertes.
%
%La Figura ~\ref{fig:sharding} remarca las diferencias entre las bases de datos distribuidas y las blockchains fragmentadas,
%mostrando la necesidad de un protocolo de formación de particiones.

% \begin{figure}
%   \centering
%   \subfloat[Bases de datos distribuidas.]{\includegraphics[scale=0.13]{figures/sharding1.png}\label{subfig:sharding-a}}
  
%   \hspace{1.5cm}
%   \vspace{0.5cm}

%   \subfloat[Blockchains fragmentadas.]{\includegraphics[scale=0.13]{figures/sharding2.png}\label{subfig:sharding-b}}
%   \caption{Protocolos de fragmentación.}
%   \label{fig:sharding}
% \end{figure}

% Traditional databases assume the crash-failure model, in which a
% faulty node simply stops sending and responding to requests. On
% the other hand, blockchain systems operate in a more hostile environment, therefore they assume a stronger failure model, namely
% Byzantine failure, to account for malicious attackers. Figure 1 highlights the differences between distributed databases and sharded
% blockchains


% approach is to use sharding, a well-studied and proven
% technique to scale out databases, to divide the blockchain network
% into smaller committees so as to reduce the overhead of consensus
% protocols. Examples of sharded blockchains include Elastico [33],
% OmniLedger [27] and RapidChain [49].

% Two important coordination protocols are distributed commit such as twophase commit (2PC) which ensures atomicity, and concurrency
% control such as two-phase locking (2PL) which achieves isolation.
% In this paper, we use the term sharding to refer to the combination
% of replication and partitioning as shown in Figure 1a.
% This architecture is adopted by recent distributed database systems to achieve
% fault tolerance and scalability [22]. Each partition is replicated over
% multiple replicas, and its content is kept consistent by consensus
% protocols [29, 42]. Transaction management and consensus protocols can be combined more judiciously, as opposed to layering one
% on top of another, to achieve better performance [40, 48, 50].

% Sharding in database systems assumes crash-failure model, in
% which a faulty node stops sending and responding to requests.
% There are three important implications of this assumption. First,
% efficient consensus protocols catered for crash-failure model can
% be used to achieve high performance. Second, creating a shard is
% simple. For example, a node can be assigned to a shard based on its
% location. Third, the coordinators that drive coordination protocols
% are fully trusted

% sharding is a well-studied and proven technique to scale out databases.
% In this work, we take a principled approach to apply
% to blockchain systems in order to improve their transaction throughput at scale.

\subsection{Capa 2}
Los enfoques para incrementar el número de transacciones por segundo en las blockchains pueden
ser categorizados según la \emph{capa} que involucran.
%
Las técnicas de capa 0 (\textit{layer 0}) son aquellas que trabajan sobre la infraestructura de la red.
%
Por ejemplo, apostando a decrementar la latencia de la misma.
% 
La capa 1 (\textit{layer 1}) es la que cuenta con más propuestas en la literatura, abarcando todas aquellas que
mejoran los algoritmos de consenso.
%
Por último, la capa 2 (\textit{layer 2}), también conocida como \emph{off-chain} (fuera de la cadena),
incluye aquellos protocolos que funcionan con una interacción mínima con la
blockchain\cite{Jourenko2019SoKAT}.

Existen diferentes enfoques dentro de la capa 2.
%
Uno de los principales consiste en la computación \emph{off-chain} de pruebas de \emph{conocimiento cero}
(\emph{Zero-Knowledge proofs})~\cite{Sasson2014ZKvonNeumann}, que solo necesitan ser validadas dentro de la cadena.
%
La adopción de funcionalidades limitadas (pero útiles) de canales como \emph{Lightning}~\cite{Poon2016lightning}
es otra técnica perteneciente a la capa 2.
%
Por último, tecnologías conocidas como \emph{optimistic rollups} (por ejemplo, \emph{Arbitrum}~\cite{Kalodner2018Arbitrum})
se basan en evitar ejecutar contratos inteligentes
en los servidores, manteniendo la mínima sincronización requerida con la cadena para preservar las garantías
de la blockchain (por ejemplo, cuando se necesitan anotar reclamos o resolver disputas).

\subsection{\setchain}
Las blockchains actuales requieren de algoritmos de consenso que garanticen que las
transacciones, empaquetadas en bloques, estén totalmente ordenadas.
%
Esta imposición de un orden total puede ser innecesaria para algunas aplicaciones.

%
Un enfoque prometedor para escalar la cantidad de transacciones agregadas por segundo
es \textit{Setchain}~\cite{Capretto.2022.Setchain},
un tipo de datos concurrente que implementa conjuntos de solo crecimiento distribuidos,
proveyendo barreras o puntos de sincronización (llamados épocas).
% A su vez, \textit{Setchain} es tolerante a comportamientos bizantinos.
%
\setchain relaja el requerimiento de orden total y, por lo tanto, logra mayor
rendimiento y escalabilidad.
%
Las \setchains se pueden usar para aquellas aplicaciones, como los registros digitales,
en donde los elementos en la blockchain no necesitan estar ordenados, excepto a través
de barreras ocasionales.

Una descripción más detallada del funcionamiento de \setchain y sus propiedades se da
en el capítulo \ref{chapter:setchain}.

\section{Contribuciones}\label{sec:contributions}
En este trabajo se realiza una contribución al estudio de la escalabilidad en blockchain mediante 
una familia de implementaciones robustas de \setchain.
%
Si bien distintos algoritmos bizantinos distribuidos que implementaban \setchain fueron
propuestos, no existía al momento ninguna implementación eficiente compatible con
los requerimientos de una aplicación del mundo real.
Es decir, una aplicación capaz de soportar un \textit{deployment} con miles de nodos\footnote{Si bien
es difícil saber con exactitud cuántos nodos existen en la red de Bitcoin o Ethereum, se estima
que este número varía entre aproximadamente 10000 y 50000 nodos.}
distribuidos geográficamente en todos los continentes, con clientes enviando peticiones para
añadir nuevas transacciones a los servidores constantemente.

Las implementaciones prototípicas previas de \setchain\footnote{Presentadas en ~\cite{Capretto.2022.Setchain}.} fueron construidas
a partir de distintos
componentes básicos: difusión confiable bizantina o \emph{Bizantine Reliable Broadcast}~\cite{DBLP:journals/iandc/Bracha87, raynal.dist.systems},
difusión atómica bizantina o \emph{Byzantine Atomic Broadcast}~\cite{Defago2004BAB},
conjuntos distribuidos que solo crecen bizantinos o \emph{Byzantine Distributed Grow-only Sets}~\cite{Cholvi2021BDSO-arxiv} y
consenso bizantino de conjunto o \emph{Set Byzantine Consensus}~\cite{redbelly}.
%
En este trabajo se toma un enfoque diferente, haciendo hincapié en la construcción de implementaciones
que cumplan con la especificación de \setchain pero que al mismo tiempo sean compatibles con los requerimientos
del mundo real.
Es por eso que cada una de las soluciones originales propuestas en este trabajo está enteramente construida sobre 
la plataforma de aplicación de blockchain \textit{Tendermint}~\cite{Buchman.2018.Tendermint}.

%% Motivate the use of tendermint. Why tendermint is relevant

% Fuente: TENDERMINT: BYZANTINE FAULT TOLERANCE IN THE AGE OF BLOCKCHAINS
% https://docs.tendermint.com/v0.34/introduction/what-is-tendermint.html

Tendermint es una plataforma novedosa y popular para la
replicación segura y consistente de aplicaciones en distintas máquinas.
%
\textit{Segura} en este contexto significa que Tendermint funciona incluso si a lo sumo
un tercio de las máquinas falla de formas arbitrarias, brindando información conflictiva
a las diferentes partes del sistema.
%
\textit{Consistente} refiere a que toda máquina correcta verá el mismo lote de
transacciones y computará el mismo estado.
%
Tendermint es un abordaje robusto de las blockchains implementado
en \emph{Go}~\cite{donovan15go}, que presenta una separación clara entre las
capas de bajo nivel de la blockchain,
tales como un protocolo \textit{gossip} y un algoritmo de consenso, y los conceptos de alto
nivel relacionados a la estructura de datos que la blockchain mantiene\footnote{Al momento
de la realización de las implementaciones y la evaluación empírica de las mismas,
Tendermint 0.34 era la versión de la plataforma vigente y más utilizada. Sin embargo,
actualmente, la versión más utilizada es un \textit{fork} de Tendermint llamado \textit{CometBFT}~\cite{cometbft.repo}.
El desarrollo planteado a lo largo de este trabajo podría en principio portarse a la nueva versión sin demasiado esfuerzo.}.

%
% What is the solution.
%

La familia de implementaciones que se presenta en este trabajo
sigue un enfoque incremental, proveyendo diversas aproximaciones
a la solución final y más compleja; consta de tres variantes:
\begin{itemize}
  \item \textit{\vanilla}, una primera solución básica de \setchain, en donde cada elemento añadido
  a la \setchain se traduce como una transacción en la blockchain.
  \item \textit{\compresschain}, una variante intermedia que usa un algoritmo de compresión,
  en la cual los elementos enviados por los clientes son agrupados en un lote, que se comprime antes
  de transmitirse como una transacción. Por lo tanto, las transacciones en la blockchain son
  lotes comprimidos de elementos.
  \item \textit{\hashchain}, la contribución principal de este trabajo, una solución 
  a \setchain usando funciones hash. Sigue una lógica similar a \compresschain,
  en donde los elementos se agrupan en lotes y se aplica una función hash antes de transmitirlos
  como una única transacción de tamaño fijo.
\end{itemize}
%
Hashchain explota el poder de compresión de las funciones hash para reducir la
comunicación necesaria durante difusión y consenso, comunicando un hash de tamaño
fijo en lugar de cientos o miles de elementos.
%
El precio a pagar es un algoritmo distribuido adicional para obtener el conjunto de
elementos desde un hash, garantizando tolerancia a servidores bizantinos.
%

La hipótesis fundamental en la que se basa \hashchain es que intercambiar latencia por un aumento en el
ancho de banda resulta en una mejora significativa en el rendimiento.
%
Es decir, hacer consenso sobre hashes de tamaño fijo conlleva un incremento en la cantidad de elementos
añadidos a la \setchain por segundo, incluso si esto significara un aumento en la latencia,
debido al algoritmo distribuido necesario para obtener el conjunto de elementos desde un hash.


% The article is structured as follows: in Section~\ref{sec:prelim}, we present a
% brief description of Setchain and Tendermint; in Section~\ref{sec:impl}, we
% describe three implementations of Setchain on top of Tendermint; and in
% Section~\ref{sec:conclusion}, we conclude our work.
% %
% This article is presented as a work-in-progress and shows partial results, and
% thus, the experimental section is lacking.

% We prove how Hashchain can implement Setchain correctly.

%%% Local Variables:
%%% TeX-master: "article.tex"
%%% TeX-PDF-mode: t
%%% End:


% %% Preliminaries:
% % + Blockchain: what blockchain are we working with? A general abstract thing or Tezos?
% % + Monitoring?
% \section{Preliminares}\label{sec:prelim}

En esta sección se presenta una breve descripción de los conceptos fundamentales
más importantes para hacer posible la comprensión y discusión sobre las
contribuciones de este trabajo.

\subsection{Sistemas distribuídos}
\subsubsection{El problema de los generales bizantinos}

\subsection{Los orígenes de la blockchain}
Las blockchains tomaron popularidad con la invención de Bitcoin en 2008, luego de la publicación
de un artículo llamado \textit{Bitcoin: A Peer-to-Peer
Electronic Cash System}\footnote{Bitcoin: un sistema de dinero electrónico de par a par.}, escrito
bajo el pseudónimo de Satoshi Nakamoto.
Nakamoto combinó diversas invenciones previas como
\textit{b-money} y \textit{HashCash} para crear un sistema de dinero electrónico completamente
descentralizado que, justamente, no necesita confiar en ninguna autoridad central para la emisión de monedas,
o la liquidación o validación de las transacciones. 
La innovación clave en Bitcoin fue el uso de un sistema de cómputo distribuído que lleva
a cabo una elección global de las nuevas transacciones cada 10 minutos, permitiendo
a la red descentralizada llegar a un consenso sobre el estado de las mismas. Estas transacciones
se guardan empaquetadas como bloques en una bitácora distribuída pública conocida como blockchain.
Se dice que estas transacciones se encuentran totalmente ordenadas porque es posible determinar para
dos transacciones cualesquieras en la blockchain cuál de ellas ocurrió antes.

%Esto resolvió de forma elegante el problema del doble gasto,
%en el cual una misma unidad de moneda se utiliza dos veces. Previamente, el problema del doble gasto fue
%una debilidad de los sistemas de moneda digital que fue abordado mediante servidores de información centrales.

Bitcoin es una colección de conceptos y tecnologías que forman las bases de un ecosistema de dinero
digital. Si bien la noción de bitcoin como criptomoneda no es imprescindible para la comprensión de las
contribuciones de este trabajo, es el primer caso de uso que da particular importancia a la blockchain
como tal y a su rendimiento. Por este motivo se considera pertinente presentar un breve resumen sobre
su funcionamiento.

\subsubsection{Funcionamiento de bitcoin}

Las unidades de moneda llamadas bitcoins son usadas para almacenar y transmitir valor a través
de los participantes de la red de bitcoin. Los usuarios se comunican entre ellos usando un protocolo 
específico principalmente mediante internet. La pila del
protocolo bitcoin, disponible como software de código abierto, se puede correr en una amplia gama de
dispositivos, incluyendo computadoras portátiles y teléfonos inteligentes, haciéndolo una tecnología
fácilmente accesible.

Los usuarios pueden transferir bitcoins en la red y hacer básicamente cualquier operación que puede ser
realizada con monedas tradicionales, incluyendo comprar y vender bienes, enviar dinero a personas u
organizaciones, o extender créditos. Los bitcoins también pueden ser vendidos, comprados, o intercambiados por
otras monedas en casas de cambio especializadas.

A diferencia de las monedas tradicionales, los bitcoins son completamente virtuales. No hay monedas
físicas e incluso no existen monedas en sí. Las monedas se representan en transacciones que transfieren
valor desde el remitente hacia el destinatario. Los usuarios de bitcoin poseen claves criptográficas
que permiten probar la propiedad de los bitcoins en la red. Con estas claves se pueden firmar las
transacciones para desbloquear el valor y utilizarlo transfiriéndolo a un nuevo dueño. 
%Las claves
%son a menudo guardadas en billeteras digitales en las computadoras o celulares de los usuarios. La
%posesión de una clave que puede firmar una transacción es el único prerequisito para utilizar bitcoin,
%poniendo el control enteramente en manos de cada usuario.

Bitcoin es un sistema distribuído de par a par. Como tal no tiene un servidor central o un punto de control.
Los bitcoins son creados a través de un proceso llamado \textit{minería}, el cual involucra una competencia
para encontrar solución a un problema matemático mientras se procesan las transacciones de bitcoin.
Cualquier participante de la red (es decir, cualquiera quien use un dispositivo que corre la pila
completa de protocolo de bitcoin) puede operar como minero, usando el poder de procesamiento de su
computadora para verificar y guardar transacciones. Cada 10 minutos, en promedio, alguien logra validar
las transacciones de los últimos 10 minutos y es compensando con bitcoin nuevo. Esencialmente, la minería
de bitcoins descentraliza la emisión de moneda y las funciones de compensación de un banco central, evitando
la necesidad del mismo.

El protocolo de bitcoin incluye algoritmos nativos que regulan las funciones de minería a lo largo de la
red. La dificultad de la tarea de procesamiento que los mineros deben realizar se ajusta dinámicamente
de modo que, en promedio, alguien tiene éxito cada 10 minutos sin importar la cantidad de mineros (y
de poder de cómputo) existente en un momento dado. El protocolo además disminuye a la mitad el ratio de creación
de bitcoins cada 4 años, y limita el número total de bitcoins que serán creados a un número fijo de
aproximadamente 21 millones de bitcoins.

%El resultado es que el número de bitcoins en circulación
%sigue una curva predecible fácilmente que llega a 21 millones para el 2140. Debido a la disminución de
%emisión de bitcoins, a largo plazo, el bitcoin es deflacionario. Incluso más, bitcoin no puede sufrir
%inflación por "imprimir" más dinero más allá de ratio esperado. 

%Detrás de escena, bitcoin es el nombre de un protocolo, una red de pares, y una innovación de
%computación distribuída.
Bitcdoin representa la culminación de décadas de investigación en criptografía
y sistemas distribuídos, e incluye 4 innovaciones claves funcionando juntas:
\begin{itemize}
  \item Una red de par a par descentralizada (el protocolo de bitcoin)
  \item Una bitácora distribuída pública de transacciones organizadas en bloques (la blockchain)
  \item Un conjunto de reglas para la validación independiente de transacciones y la emisión de monedas (reglas de consenso)
  \item Un mecanismo para lograr un consenso global descentralizado en la blockchain (algoritmo de prueba de trabajo)
\end{itemize}

\subsubsection{La forma de la blockchain}

Cada nodo completo en la red de bitcoin guarda independientemente una blockchain que contiene solo
bloques validados por ese nodo. Cuando cierta cantidad de nodos tienen los mismos bloques en su propia blockchain
se dice que llegaron a un consenso. 
%The validation rules these nodes follow to maintain consensus are
%called consensus rules. This section describes many of the consensus rules used by Bitcoin Core.

La Figura muestra una versión simplificada de una blockchain. En la sección de datos de un bloque
se agrupan y alojan una o más transacciones nuevas. Copias de cada transacción se hashean, los hashes luego
se emparejan, se hashean, se emparejan de nuevo, y se hashean hasta que queda un único hash: la raíz de un
\textit{merkle tree}. 

La raíz del merkle tree se aloja en el encabezado del bloque. Cada bloque además contiene el hash del encabezado
del bloque anterior, encadenándolos. Esto asegura que una transacción no pueda modificarse sin modificar el bloque
que la contiene y todos los bloques siguientes.

% Agregar imágenes de blockchain y de merkle tree.

%Las transacciones también están encadenadas. Es decir, cada transacción gasta los \textit{satoshis}\footnote{
%Satoshi es la unidad mínima de Bitcoin, llamada así por el pseudónimo de su creador, Satoshi Nakamoto. Un bitcoin
%es equivalente a 100 millones de Satoshis.} previamente recibidos en una o más transacciones previas, por lo
%que la entrada de una transacción es la salida de una transacción previa.

%\subsection{Blockchain actuales}


\subsection{Setchain}
\setchain \cite{Capretto.2022.Setchain} is a distributed concurrent data-type
that implements Byzantine tolerant distributed grown-only sets with barriers
(called epochs).
%
Barriers impose an order between elements in different epochs but not with
elements in the same epoch.
%
Therefore, \setchain relaxes the total order requirement imposed by blockchains,
and thus, can achieve higher throughput and scalability.
%
\setchain can be used for those applications, like digital
registries, where different elements in the blockchain need not be
ordered except across infrequent barriers.

\paragraph*{Setchain API.}
%
Let \(U\) be a set of elements that client processes can inject into the
\setchain.
%
Moreover, let \isValidElement\ be a function that nodes can use to locally
validate elements in \(U\).
%
A \setchain is a distributed data structure where a collection of server nodes
maintain:
% \begin{compactitem}
\begin{itemize}
\item a set $\<theset> \subseteq U$ of elements added;
\item a natural number $\<epoch> \in \mathbb{N}$;
\item a map $\<history> : [1..\<epoch>] \rightarrow \mathcal{P}(U)$\footnote{$\mathcal{P}(U)$ denotes the
power set of $U$} describing sets of elements that have been stamped with an
epoch number.
\end{itemize}
%
Server nodes support two operations: \(\<add>\) and \(\<get>\).
%
Operation \(\<add>\) requests to add an element, while operation \(\<get>\)
returns the values maintained by the node~(\(\<theset>,\<history>,\<epoch>\)).
%
We use dot-notation to invoke these operations, let \(v\) be a server node and
\(e\) an element: \(v.\<add>(e),v.\<get>\).

% Each server node $v$ supports two operations,
% available to any client process:
% % \begin{compactitem}
% \begin{itemize}
% \item $v.\<add>(e)$: requests to add $e$ to $\<theset>$.
% \item $v.\<get>()$: returns the values of $\<theset>$, $\<history>$,
%   and $\<epoch>$, as perceived by $v$.
% \end{itemize}

When barriers are triggered, nodes maintaining the \setchain collaboratively
decide which added elements are stamped with the current epoch and increase the
epoch number.
%
We call these events \emph{epoch increments}.
%
We assume that periodic synchronization barriers are triggered.

A typical workflow from the point of the view of a client is as follows: a
client invokes $\<add>(e)$ in one (or more) servers to insert a new valid
element $e$ in the \setchain.
%
The element $e$ will be propagated among the servers, and when an
epoch increment occurs, the servers will attempt to include it in the
new epoch.
%
After waiting for some time, the client invokes $\<get>$ in one (or
more) servers to check that the element has been effectively added and
stamped.

\paragraph*{Properties.}
To ensure correctness, \setchain implementations must satisfy certain
properties that provide eventual guarantees for elements added and
guarantee consistency between correct servers. 
%
These properties reason about correct servers since Byzantine servers
do not provide any guarantee.

\begin{compactitem}
  \item Every valid element added by a correct
server is eventually returned in all future gets issued in all correct
servers.
  \item All valid elements added in a correct server must be eventually
be stamped in all correct servers.
  \item Once an element is stamped with an epoch, it cannot be unstamped, nor
can it be stamped with another epoch.
\item Any two correct servers agree on the content of
all epochs that both have computed~\footnote{Not all correct servers process epoch increments
simultaneously, as some may be more delayed than others.}.
%
% Nevertheless,
\item Every element that is stamped comes from the result of a client adding the element.
  \end{compactitem}
\gabina{Epoch proofs are part of the epoch but are not coming directly from the result of a client adding element.}

\subsection{Tendermint}\label{sec:tendermint}
%
Tendermint is a state machine replication engine that tolerates Byzantine faults.
%
It was among the first systems to adapt classical Byzantine Fault Tolerant consensus protocols
to the blockchain paradigm, whereby consensus is performed on cryptographic hash-linked batches of
transactions (i.e., blocks) in a public, open-membership network.
%
Tendermint functions as a blockchain middleware that supports the replication of arbitrary
applications, written in any programming language~\cite{tendermint.design}.

In Figure~\ref{fig:replication}, we show the overview of replicated state machine architecture.
%
A replicated state machine replicates a transaction log and resulting state across multiple machines.
%
Transactions are received from the client, run through the consensus protocol, ordered in the
transaction log, and executed against the state.
%
% In the figure, each , with .
%

\begin{figure}
  \centering
  \includegraphics[scale=0.25]{figures/replication_engine.pdf}
  \caption{Overview of replicated state machine architecture~\cite{Buchman.2018.Tendermint}.
    %
    Diamonds represent machines.
    %
    Dotted lines represent communication between machines to carry out the
consensus protocol for ordering transactions
  }
  \label{fig:replication}
\end{figure}

%

Tendermint consists of two chief technical components: a blockchain consensus engine and a
generic application interface.
%
The consensus engine, called Tendermint Core, ensures that the same transactions are recorded
on every machine in the same order.
%
The application interface, called the Application BlockChain Interface (ABCI), enables the
transactions to be processed in any programming language.
%

\subsection{Application BlockChain Interface}
ABCI is the interface between Tendermint Core
% (the \textquotedblleft
% consensus engine\textquotedblright)
and the replicated application.
%
A Tendermint node maintains three main ABCI connections with the replicated application.
%

The \textit{consensus connection} is used only when a new block is committed,
and communicates all information from the block in a series of 
requests: \<BeginBlock>, [\<DeliverTx>, ...], \<EndBlock>, \<Commit>.
%
That is, when a block is committed in the consensus, Tendermint sends a 
list of \<DeliverTx> requests (one for each transaction) sandwiched by 
\<BeginBlock> and \<EndBlock> requests, and followed by a \<Commit>.

%

The \textit{mempool connection} is used by the transaction
pool protocol to validate transactions submitted
by clients against the application state. 
%
It is used only for \<CheckTx> requests. Transactions
are run using \<CheckTx> in the same order they were received
by the validator\footnote{Validator nodes are responsible for committing new blocks in the blockchain. These validators participate in the consensus protocol by broadcasting votes which contain cryptographic signatures signed by each validator's private key.}. If the \<CheckTx> returns OK, the transaction
is kept in memory and relayed to other peers in the same order
it was received. Otherwise, it is discarded.
%
It is up to the application to define whether a transaction is valid or not, and
the validation is optional. 

%

The \textit{query connection} allows retrieving information
from the local instance of the application, used by several
Tendermint modules (e.g., peer filtering).
%
It is used to query the application without engaging consensus. 
%It is exposed over the tendermint core rpc, so clients can query
%the app without exposing a server on the app itself.

Figure~\ref{fig:abci_flow} shows the flow of messages via consensus and mempool connections.
%

\begin{figure}
  \centering
  \includegraphics[scale=0.35]{figures/abci_msg_flow.pdf}
  \caption{Flow of messages via ABCI~\cite{tendermint.site}.}
  \label{fig:abci_flow}
\end{figure}

%%% Local Variables:
%%% TeX-master: "article.tex"
%%% TeX-PDF-mode: t
%%% End:


% %% Model of computation
% % + Define the model of computation
% % + Introduce leading example
% %\input{ModelOfComputation}

% %% Setchain implementations
% \section{Implementaciones}\label{sec:impl}

En esta sección se presenta una familia de implementaciones de \setchain de mundo
real construídas sobre Tendermint. 
%
En particular, se exponen tres alternativas diferentes, comenzando con una
solución inocente pero
trivialmente correcta, y finalizando con un algoritmo
complejo que implementa \setchain utilizando funciones hash.
%

Para evitar repeticiones, las definciones de funciones que permanezcan sin cambios
de una versión a la siguiente, no serán re-escritas. Solo se volverán a presentar
aquellas funciones para las cuales la definición se vea modificada.
%
A su vez, con la intención de mantener consistencia en la nomenclatura, el término \textit{transacción}
se utiliza siempre para referirse a las \textit{transacciones de Tendermint}, mientras
que \textit{elemento} queda reservado para elementos a agregarse a la \setchain.
%
Dependiendo de la alternativa sobre la que se esté trabajando, una transacción de
Tendermint puede contener uno
o más elementos a ser agregados.


% \subsection{General Considerations about Implementation}\label{subsec:general}
Las implementaciones correctas de \setchain implementan dos métodos~(ver sección
~\ref{sec:prelim}): \Add y \Get y, por lo tanto, cada solución provee definiciones
para ambas.
%

Tendermint provee dos \textit{endpoints} RPC principales: \texttt{Tendermint.Broadcast}
inyecta una transacción con el objetivo de que se haga \textit{commit} sobre ella, y
\texttt{Tendermint.Query} consulta el estado de la aplicación.
%
\begin{comment}
Therefore, from the point-of-view of the \setchain client, we only have two
methods (\Add and \Get).
However, we also have two additional ones to communicate with the underlying
Tendermint network (\<broadcastTx> and \<abciquery>).

%
% Moreover, let \(U\) be the universe of elements, Setchain implementations
% provide the following variables:
% \begin{itemize}
%   \item the elements added so far \(\<theset> \subseteq U\)
%   \item the number of current epoch \(\<epoch> \in \mathbb{N}\)
%   \item a map \(\<history> : [1..\texttt{epoch}] \rightarrow \mathcal{P}(U)\) defining which elements
% form an epoch.
% \end{itemize}
% %
% Some elements in \(\<theset>\) may not have been assigned an epoch yet.

% along with providing a set \(\<theset> \subseteq U\) of elements added,
% a natural number \texttt{epoch} $\in \mathbb{N}$; and a map \texttt{history}
% :$$, that describes the sets of elements that have
% been stamped with an epoch number ($\mathcal{P}(U)$ denotes the power set of $U$, being
% $U$ the universe of elements that client processes can inject into the set).
% Since we are using the Tendermint framework, some names may vary, which will be
% duly clarified.

Finally, Setchain assumes there is a user-defined predicate defining when an
element is valid to be in the set.
%
In this section, we assume that there is a definition of \(\<isValidElement>\)
function.

% To inject elements into the \setchain or to query the application state, clients
% need to communicate with the Tendermint node which runs the \setchain through
% RPC.
%
%
% \begin{itemize}
% 	\item \<broadcastTx>, which allows to broadcast transactions.
%         %
%         % This is the equivalent to the \texttt{add} operation.
%         %
% %         From now on, this endpoint will be abstracted as
% % \texttt{Tendermint.Broadcast()}.
% 	\item \<abciquery>, which allows to query the application (in our scenario, the \setchain). This is the equivalent to \texttt{get} operation. We will use it as \\ \texttt{Tendermint.Query()}
% \end{itemize}
% In addition, \texttt{epochinc} will not be provided, as the epoch increase will be periodically triggered as part of the normal behavior.
%

Below we give different function definitions for the application running on
Tendermint to implement \setchain\ specification.
%
As mentioned in the previous section~(see Section~\ref{sec:prelim}), we
implement each algorithm by defining different instances of the Tendermint ABCI
and the methods presented as the Setchain API.

\subsection{First Implementation: Vanilla}\label{subsec:vanilla}

% Vanilla API - alg0
\begin{figure}[t!]
  \begin{adjustbox}{minipage=[t]{\columnwidth}}
    \begin{algorithm}[H]
      \renewcommand{\thealgorithm}{API Vanilla}         
      \caption{}%
      \label{alg:api-vanilla}%
      \small
      \begin{algorithmic}[1]
            \Function{\<add>}{$transaction$}\label{alg:van_add}
                \State \textbf{return} \<broadcastTx>($transaction$)
                % \Comment {Hit broadcast\_tx\_async RPC endpoint}.
            \EndFunction
      
            \Function{\<get>}{\null}\label{alg:van_get}
                	\State \textbf{return} \<abciquery>()
            \EndFunction
            
        \end{algorithmic}
      \end{algorithm}
	\end{adjustbox}
  \end{figure}


In Algorithm~\ref{alg:api-vanilla}, we show the most basic solution to \setchain API
using Tendermint.
%
Clients add elements by
% indirectly
invoking function \<broadcastTx>,
% Adding an element is done by just hitting the Tendermint RPC endpoint to
% broadcast elements,
%
while they can also get the \setchain state by invoking function
\<abciquery>.
% involves hitting the query
% Tendermint RPC endpoint.
%
% As mentioned earlier, \texttt{Epochinc} is not part of the \setchain API, since
% the epoch increase is triggered periodically.
% An explicit demonstration of how
% the epoch increase works will be shown soon.

%

To give a complete implementation of Setchain using Tendermint, we also need to
define our application ABCI~(see Section~\ref{sec:tendermint}).
%
% The \setchain API is not enough to define a new \setchain implementation, as
% Tendermint needs the ABCI definition too.
%
Algorithm~\ref{alg:abci-vanilla} shows our ABCI implementation for the first
naive implementation.
%
We only define functions \<CheckTx>, \<DeliverTx>, and
\<EndBlock> from the ABCI, the rest have trivial implementations.
%
% All the other non-mentioned ABCI functions have trivial implementations.

% Vanilla ABCI - alg1
\begin{figure}[t!]
  \begin{adjustbox}{minipage=[t]{\columnwidth}}
    \begin{algorithm}[H]
      \renewcommand{\thealgorithm}{ABCI Vanilla}         
      \caption{}%
      \label{alg:abci-vanilla}%
      \small
      \begin{algorithmic}[1]
            \State \textbf{Init:} \texttt{epoch} $\leftarrow$ \textbf{0}, \texttt{history} $\leftarrow$ \{\}

            \Function{\<CheckTx>}{$transaction$}\label{alg:van_check_tx}
                \State \textbf{return} \Call{\<isValidTransaction>}{$transaction$}
            \EndFunction
      
            \Function{\<DeliverTx>}{$transaction$}\label{alg:van_deliver_tx}
                \If {\Call{\<isValidTransaction>}{$transaction$}}
                		\State \<element> $\leftarrow$ \Call{\<getElementFromTransaction>}{$transaction$}
                		\State  \texttt{history[epoch]} \(<- \, \texttt{history}[\texttt{epoch}] \cup \<element>\)
                	\EndIf
                	\State \textbf{return}
            \EndFunction
            
            \Function{\<EndBlock>}{\null}\label{alg:van_end_block}
            		\State \texttt{hash} $\leftarrow$ \<Hash>(\texttt{history[epoch]}, \texttt{epoch})
            		\Comment{Calcular el hash de la época.}
                \State \texttt{epochProof} $\leftarrow$  \texttt{Sign(\texttt{hash}, privateKey)}
                \State \Call{\<add>}{\texttt{epochProof}}
                \State \(\<epoch> \, <- \<epoch> + 1\)
                \Comment{Cada bloque de Tendermint define una época.}
                \State \textbf{return}
            \EndFunction
            
             \Function{\<Query>}{\null}\label{alg:van_query}
                \State \textbf{return} (\texttt{history}, \texttt{history} up to \<epoch>\ - 1, \<epoch>\ - 1)            
             \EndFunction
            
            \Function{\<isValidTransaction>}{$transaction$}\label{alg:van_is_valid_tx}
                \State \<element> $\leftarrow$ \Call{\<getElementFromTransaction>}{$transaction$}
                \State \textbf{return} \<isValidElement>(\<element>)
            \EndFunction
            
            \Function{\<getElementFromTransaction>}{$transaction$}\label{alg:van_get_element}
                \State \textbf{return} $transaction$
            \EndFunction
        \end{algorithmic}
      \end{algorithm}
	\end{adjustbox}
  \end{figure}


The usual flow of elements added by \<add>\((e)\) in the Vanilla implementation
is: first Tendermint checks that the element \(e\) is valid with \<CheckTx>, and if the
element is valid, then \(e\) is added to the mempool.
% The call to \texttt{Add(e)} will generate a request to
% \texttt{CheckTx()} to run validations over the element $e$.
% %
% If the element passes the validations, it is inserted to the mempool.
%
After a while, $e$ is expected to be added to a Tendermint block, and when the
block gets committed in the consensus, Tendermint sends first a \<BeginBlock>
request followed by a list of \<DeliverTx>, an \<EndBlock> request, and finally a
\<Commit>.
% requests (one for each transaction in the consented order) sandwiched by
% \<BeginBlock> and \<EndBlock> requests, and followed by a Commit.
%

Tendermint invokes function \texttt{CheckTx} to decide
whether a new transaction should be added to the mempool or discarded.
%
In this version, checking that a transaction is valid consists on simply
checking the validity of the element, as transactions contain only one
element.
%
%Commented the following line as I'm not sure about talking about future implementations
%in this subsection:
%However, in the following implementation transaction are not the same as
%elements.
%
% It may seem silly to define the \texttt{getElementFromTransaction()} function in
% this case (given that it has no effect, it is the identity function).
%
% However, the decision to explicitly define such a function is to reinforce the
% conceptual difference between a Tendermint transaction, and the elements to be
% added to the \setchain.

%

%The Tendermint consensus doesnt invoke the deliverTx
Tendermint core sends \texttt{DeliverTx} requests
asynchronously (but in order) once per each transaction in the block.
%
Transactions have already been ordered in the global consensus by
the Tendermint protocol when \texttt{DeliverTx} is executed.
%
For this algorithm, the only thing to do when transactions are delivered is to add
the underlying element to the \setchain only if it is a valid one.
% Should we mention Byzantine proposers here?
We cannot avoid checking if a transaction is valid because it may have been
valid by the time it was inserted in the mempool but rendered invalid by the
time is consolidated.
%
Additionally, a Byzantine node may add transactions without checking their
validity.
% If the element is not valid, the corresponding transaction should have been
% rejected by the mempool, but may have been included in a block by a Byzantine
% proposer.
%
Tendermint invokes function \texttt{EndBlock} once per block, after delivering
all transactions inside the block.
%
In this algorithn, the end of a block triggers an epoch increment, and thus,
each Tendermint block defines a different \setchain epoch.
%
Finally, when clients request a \texttt{Get}, they call the \<abciquery>
function.

\subsubsection{Elements Membership Proof}\label{subsubsec:membership}
%
Client processes do not know if they are contacting a Byzantine or correct
process.
%
To ensure that an element is going to be added to the \setchain, the client
needs to interact with enough servers to guarantee that at least one server is
correct~\cite{Capretto.2022.Setchain}.
%
% An extension to Setchain presented an optimistic client with the following
% approach.
%
Correct servers sign cryptographically a hash of the set of elements in an
epoch, and insert this hash in the \setchain as an element.
%
We call those hashes epoch proof elements. 
%
Epoch proofs may be implemented using Ed25519~\cite{ed25519} as signature system.
%
Epoch proofs may contain the epoch number being proved, the validator's public key
signing, and the signature of the elements in the given epoch (ordered in a well-known
specific way). 
%
Clients only perform a single $\<add>(e)$ request to one server, hoping it is a
correct server.
%
After waiting for some time, clients can invoke a \<get> from a single server
and checks whether $e$ is in some epoch and there exists enough epoch proofs\footnote{We
  assume that an upper bound, $f$, in the number of Byzantine servers is
  known and therefore $f+1$ signatures are required.} to guarantee that at
least one correct server has signed it.
% (at least) $f$ + 1 servers.
Clients are able to verify if epoch proofs are valid by generating the hash of elements
in the given epoch, and verifying if the signature in the epoch proof element is valid
for the hash and the public key.
%
Validators' public keys need to be publicly known to users.
%
If $e$ is in an epoch with enough epoch proofs, clients can conclude
that the epoch is correct and $e$ has been successfully inserted and
stamped.
%
Note that this requires only one message per \<add> and one message per \<get>.
%
We implemented this epoch proof mechanism in the definition of \<EndBlock>.

%
Another option to implement epoch proof elements is using ethereum crypto library.
%
This library provides a \texttt{Sign} function that calculates an ECDSA signature.
%
The produced signature is in the $[R || S || V]$ format, where $V$ is 0 or 1.
%
The $V$ value is usually called recovery bit, and it allows to use the \texttt{Ecrecover}
function, which takes in a hash and a signature, and returns the uncompressed public key
that created the given signature.
%
This way, epoch proof elements may contain the epoch number and the signature in the format
mentioned above, and clients may use the \texttt{Ecrecover} function to verify that the given
signature was indeed generated by a known validator.
%

%
Notice that epochs now contain two kind of elements: client regular elements, and epoch
proof elements; meaning that every epoch proof element belongs to an epoch. However,
epoch proof elements do not need to be included as elements in the signed hash that is
part of the epoch proof for the given epoch.
%
Consequently, we need to differentiate between regular elements, and epoch proof elements.
%
Regular elements are checked for validity using the already mentioned \texttt{isValidElement}
function.
%
However, an epoch proof element needs to be validated in other way: verifying the signature.
%

While the implementation described in this section implements \setchain, it is
not exploiting its main idea: to lift the total order between elements.
%
% and take advantage of the
% performance improvements that implies.
%
Even though several elements may belong to the same epoch, meaning that no total
order between them could be established, Tendermint consensus protocol is
running underneath deciding a total order for the elements.
% that then is broken
% when adding the elements from the same block to a specific epoch.


\subsection{Second Implementation: Compresschain}\label{subsec:compresschain}

To get closer to the specification of Setchain, we propose to pack elements
together into a batch, reassembling a set, and subsequently compressing it
prior to transmission.
%
We propose a new implementation, called \emph{Compresschain}, that attempts to
explore the relaxed order proposed by \setchain, even though we still run
Tendermint consensus underneath.
%
Instead of immediately broadcasting each element added by clients as a
Tendermint transaction, a new middleware piece of software called
\textit{collector} is responsible for collecting elements until reaching a batch
long enough and broadcast that batch as a transaction.
%
% This means that a new
% \setchain API version is defined, which uses the new collector component.
%
% We define a collector in Algorithm.

Following the current practice of Ethereum, we use Ethereum's Recursive Length
Prefix~\cite{ethereum} encoding and Brotli
compression~\cite{brotli.compressor}~\footnote{ Although other compressions
algorithms can be used if required.}.
%
% To achieve a better throughput in
% terms of size per element broadcasted,  is used.
%
Instead of adding elements sent by users into the network, we first collect them
and encode them using RLP, and once the batch is ready, we compress the batch
with the Brotli compression algorithm.
%
We do not provide any specific criteria to determine when batches are ready.
%
However, we have it depends on a maximum size as well as a reasonable amount of
time since the first element came in.
%
In Algorithm~\ref{alg:collector-brotli}, we show an implementation of a
collector algorithm, and in Algorithm~\ref{alg:api-brotli}, we present the new
\setchain API.
%
In this implementation, a client requesting an \<add> invokes the
collector \texttt{AddElement} method.
%, instead of \<broadcastTx>.
%
% Some simplifications were made, as for instance a race condition could occur if
% several routines add an element to the same collector instance at the same time.
% %
% However, that can be easily solved using locks.

% Compresschain API - 
\begin{figure}[t!]
  \begin{adjustbox}{minipage=[t]{\columnwidth}}
    \begin{algorithm}[H]
      \renewcommand{\thealgorithm}{API Compresschain}         
      \caption{}%
      \label{alg:api-brotli}%
      \small
      \begin{algorithmic}[1]
      
            \Function{\<add>}{$element$}\label{alg:brotli_add}
                \State \textbf{return} \texttt{CompressCollector.AddElement($element$)}
                \Comment {Use the middleware collector component}.
            \EndFunction
      
            \Function{\<get>}{\null}\label{alg:brotli_get}
                	\State \textbf{return} \<abciquery>()
            \EndFunction
            
        \end{algorithmic}
      \end{algorithm}
	\end{adjustbox}
  \end{figure}


% Brotli Collector - alg2
\begin{figure}[t!]
  \begin{adjustbox}{minipage=[t]{\columnwidth}}
    \begin{algorithm}[H]
      \renewcommand{\thealgorithm}{Compress Collector}         
      \caption{}%
      \label{alg:collector-brotli}%
      \small
      \begin{algorithmic}[1]
            \State \textbf{Init:} \texttt{batch} $\leftarrow$ \{\}
      
            \Function{\<AddElement>}{$element$}\label{alg:brotli_add_tx}
            		\If {\<isValidElement>($element$)}
            			\State \texttt{encoded\_element} $\leftarrow$ \texttt{RLP.Encode}($element$)
					\State \texttt{batch} $\leftarrow$ \texttt{batch} $\cup$ \{\texttt{encoded\_element}\}
            
             		\If {\<isReady>(\<batch>)}
             			\State \texttt{compressed\_batch} $\leftarrow$  \texttt{Brotli.Compress}(\texttt{batch})
                			\State \texttt{Tendermint.Broadcast}(\texttt{compressed\_batch})
                			\State \Call{\<reset>}{\null}
             	   \EndIf
             	\EndIf
                	
                \State \textbf{return}
            \EndFunction
            
            \Function{\<reset>}{\null}\label{alg:brotli_reset}
            		\State \texttt{batch} $\leftarrow$ \{\}
                \State \textbf{return}
            \EndFunction
        \end{algorithmic}
      \end{algorithm}
	\end{adjustbox}
  \end{figure}



%

% After the meeting we decided to call isValidElement() locally in the collector, so the following is not true:
% It is worth mentioning that, as the algorithm shows, the collector adds every element that receives to the batch without running any validation on it. While this clearly allows the batch to contain garbage, making a \textit{CheckTx()} request to the ABCI app for each new element would mean a decrease in performance, as the number of requests would keep constant in the number of elements sent by clients, instead of being constant in the number of batches, as it is desired.

%
To complete our definition, we provide the expected ABCI of Compresschain~(see
Algorithm~\ref{alg:abci-brotli}).
%
% There are some main differences as regards the previous \setchain implementation.

%
% Brotli ABCI - alg1
\begin{figure}[t!]
  \begin{adjustbox}{minipage=[t]{\columnwidth}}
    \begin{algorithm}[H]
      \renewcommand{\thealgorithm}{Compresschain ABCI}         
      \caption{\small }%
      \label{alg:abci-brotli}%
      \small
      \begin{algorithmic}[1]
            \State \textbf{Init:} \texttt{epoch} $\leftarrow$ \textbf{0}, \texttt{history} $\leftarrow$ \{\}

            \Function{\<CheckTx>}{$batch$}\label{alg:brotli_check_tx}
                \State \textbf{return} \Call{\<isValidBatch>}{$batch$}
            \EndFunction
      
            \Function{\<DeliverTx>}{$batch$}\label{alg:brotli_deliver_tx}
				\State \texttt{elements} $\leftarrow$ \Call{\<getElementsFromBatch>}{$batch$}
				\State \Call{\<newEpoch>}{\texttt{elements}}
            		
            		\State \textbf{return}
            \EndFunction
            
            \Function{\<isValidBatch>}{$batch$}\label{alg:brotli_is_valid}
            		\State elements $\leftarrow$ \Call{\<getElementsFromBatch>}{$batch$}
            		
            		\Comment{If at least one element in the batch is valid, then the batch is considered valid.}
            		\For{\texttt{e in} $elements$}
                    \If {\<isValidElement>(\texttt{e}) and not \texttt{e} in \texttt{history}}
                    		\State \textbf{return} \texttt{True}
                    \EndIf
                \EndFor
                \State \textbf{return} \texttt{False}
            \EndFunction
            
            \Function{\<getElementsFromBatch>}{$batch$}\label{alg:brotli_get_element}
                \State \texttt{decompressedBatch} $\leftarrow$ \texttt{Brotli.Decompress}($batch$)
                \State \texttt{elements} $\leftarrow$ \texttt{RLP.Decode}(\texttt{decompressedBatch})
                \State \textbf{return} \texttt{elements}
            \EndFunction
            
            \Function{\<newEpoch>}{$elements$}\label{alg:brotli_new_epoch}
            		\For{\texttt{e in} $elements$}
             		\If {\<isValidElement>(\texttt{e}) and not \texttt{e} in \texttt{history}}
                				\State \texttt{history[epoch]}.AddElement(e)
                				\Comment{Only add new valid elements.}
                    	 \EndIf
                	\EndFor
                	
                	\State \texttt{hash} $\leftarrow$ \<Hash>(\texttt{history[epoch]}, \texttt{epoch})
            		\Comment{Hash epoch (elements and number).}
                \State \texttt{epochProof} $\leftarrow$  \<Sign>(\texttt{hash}, privateKey)
                \State \Call{\<add>}{\texttt{epochProof}}
                	
                	\State \texttt{epoch} $\leftarrow$ \texttt{epoch} + 1
                \State \textbf{return}
            \EndFunction
        \end{algorithmic}
      \end{algorithm}
	\end{adjustbox}
  \end{figure}



The main difference with the Vainilla implementation is that transactions
contain potentially more than one element, because transactions are 
compressed batches of elements.
%
Moreover, to retrieve the elements inside transactions, we need to decompress
the batch first, and once the original batch is recovered, it has to be
RLP-decoded to get the original elements sent by the clients.
%
% Those original transactions corresponds to the elements to be added to
% the \setchain. \textit{getElementsFromTransaction()} function shows that
% behavior.

Using Compresschain, Tendermint transactions may contain both valid
and invalid elements.
%
We need to define a new criteria to define when a transaction is
considered valid.
%
The function \<isValidBatch>~(in Algorithm~\ref{alg:abci-brotli})
implements this new criteria, allowing transactions to be considered valid if at
least one element on it is valid.
%
The choice of this criteria is related to the fact that someone sending invalid
elements~\footnote{Elements may be valid at the moment clients send them, but
invalid when blocks are commited.} to a node should not prevent valid
elements sent to the same node from being added to the \setchain.
%
However, as it is illustrated in function \<DeliverTx>, only valid
elements are added to the \setchain, while invalid ones are simply discarded.

Finally, in Compresschain, blocks do not delimit epochs, and thus, we do not
give a definition to function \<EndBlock>.
%
In this case, we define epochs as transactions: each batch of elements defines an
epoch.
% %
% definition was given in this case, as the block does not
% delimit the epoch anymore, but the epoch is defined by all the elements
% belonging to the same Tendermint transaction.
% \textit{DeliverTx()} function
% shows the epoch number increase.

% To dicuss: if epochs are defined by batches, then epochs are generated by the same node.
% This is a new setchain property and may be undesirable. 
\subsection{Third Implementation: HashChain}\label{subsec:hashchain}
%
Hashchain aligns with the underlying concept introduced in Compresschain;
however, it uses hash functions rather than Brotli compression.
%Hashchain follows a similar idea introduced in Compresschain but employing hash
%functions instead of Brotli compression.
%
While the compression power of hash functions may be enormous, as those
functions map data of arbitrary size to fixed-size values, hashes are
irreversible, meaning that a non-trivial method to recover the original batch of elements
has to be provided.
%
The implementation of Hashchain involves two aspects: a Tendermint blockchain of
hashes and a distributed inverse function retrieving batches from hashes.
%
% We shouldn't mention "consolidated hashes" before explaining it:
%Therefore, we now have to maintain an inverse function for the consolidated
%hashes.

Often in this section we will take the liberty of referring to \textit{elements
in a hash H} for those elements belonging to a batch $B$, such that
$Hash(B) = H$.

Instead of using the compression collector as in Compresschain (see
Algorithm~\ref{alg:collector-brotli}), we now use a \emph{hash
collector}, where elements are batched and then hashed. 
%
Hashed batches are then added to the Tendermint network as transactions and
shared across the whole network, in a manner analogous to the approach employed
for compressed batches in Compresschain.

\subsubsection{Distributed Algorithm for Hash Reversal}

While the hash collector adopts a concept similar to the compression collector,
the ABCI side of Hashchain is considerably more complex.
%
In this scenario, \<CheckTx> as well as \<DeliverTx> receive hashes as a transaction.
%
The computer running the \setchain is not immediately able to translate
hashes into their original batch of elements, owing to the irreversible
nature of hashes.
%
The absence of the original batch of elements renders both \<CheckTx> incapable
of verifying the elements in the transaction and \<DeliverTx> unable to
add elements to the \setchain.
%

At this juncture, our distributed algorithm for hash reversal comes into play.
%
We need to communicate with a Tendermint node that knows the hash
(initially, only its creator).
%
To distribute the information about who knows the data of a hash, transactions
contain not only the hash of the batch but also a signature.
%
Signatures accompanying hashes indicate that a specific Tendermint
validator (the one signing the hash) claims to know the hash, hence,
if it is a correct node, it has the original batch of elements.
%
Then, transactions are represented by a tuple $(h, s)$, where $h$ denotes
the hash value and $s$ represents the signature obtained by signing $h$ with
the private key of the node that knows the hash.
%
Thanks to the signature, we can communicate with the node claiming to know the
data, and get the original batch, if the node behaves properly.
%
We can run the process of requesting hashes asynchronously to avoid potential
delays.

%

After leaving the hash collector, hashes are expected to be checked against
\<CheckTx> to be either added to the mempool or discarded.
%
To determine the validity of a hash, we need to check the elements in it.
%
Similar to Compresschain, hashes are considered valid if at least one element
in the hash is valid.
%
At the point \<CheckTx> runs, we cannot ensure that the original batch of
elements is known for the given instance.
%
Hashchain is optimistic in the sense that in the event of impossibility
to run the transaction check due to the absence of the original batch,
\<CheckTx> considers hashes as valid transactions with the hope that
their batches will be sent later on.

%
The optimistic nature of Hashchain entails notable consequences.
%
As the default behaviour for \<CheckTx> is to return \texttt{True},
we can have transactions in the blockchain without having the elements
but just the hash of the elements.
%
Moreover, a transaction may be committed without anyone having
previously checked it, but passing the check because of the optimistic
behavior.
%
For example, this is likely the case of the first time a new hash shows
up in the blockchain.

%

%Therefore, we need a way to extract the elements inside the transactions and
%provide the user a uniform view, that of a Setchain.
Therefore, we need to define the criteria by which a hash residing within the
blockchain is safe to be part of the \setchain.
% the HashChain living in
% Tendermint, and the \setchain we want to build.
%
To achieve that, we define a natural number \texttt{SIGNATURES\_PER\_HASH}
defining the number of signatures a hash has to possess to be considered a batch
of elements to be added to the \setchain.
%
We are only interested in hashes signed by at least
\texttt{SIGNATURES\_PER\_HASH} nodes, and we call them
%
% A hash for which the ABCI has seen \texttt{SIGNATURES\_PER\_HASH} or more
% signatures, it is said to be a
\textit{consolidated hashes}.
%
\texttt{SIGNATURES\_PER\_HASH} is defined in such a way that a consolidated hash
is guaranteed to be known by at least one correct node.
%
The elements of consolidated hashes are the candidates to belong to the
Setchain.

%Following the same criteria as in Compresschain, we consider a transaction to
%be valid if we can get the data and validate at least one of its elements.
%
% With this in mind, whenever the ABCI gets the revert of a hash, it runs the
% validation against it, and
%Moreover, if at least one element in the batch is valid, the node verifying the
Whenever a node asks for the revert of hash, if at least one element in the batch
is valid, the node signs the hash, and broadcast the hash along with its own signature.
%
By broadcasting the hash and its signature, the node announces to the network
that they know the data behind that hash.
%
This way, valid batches will eventually consolidate, meaning their elements will
be added to the \setchain.

For all this to work, we need to keep track of how many signatures a hash has,
as well as a map from hashes to batches.
%
Each time \<DeliverTx> gets a transaction, we check the
signature for validity.
%
If the signature is valid and new for the given hash, we increase the
\textit{signatures per hash} counter.
%
In addition, each time we revert a hash, we check the correctness of the
original batch (i.e $hash = Hash(originalBatch)$).
In case of success, the map from hashes to
batches is updated with the new discovery.

%The first time a hash is passed around, other nodes would see the new hash and
%try to get the data from other servers.
%
%The first one claiming to have the data is the creator of the transaction.


%When a \<get> arrives, we need to get the elements out of the consolidated
%hashes.
% it is necessary to ask for the batch
% corresponding to the hash in the transaction.
%



% Something we discussed with Marga: when the collector creates a hash h and signs it, and broadcast (h, s) every call to CheckTx(h, s) will be true, as no one is expected to have the reverse of h the first time they see it. Because of that, (h, s) in the hashchain doesn't mean that there is at least one good valid element in the h. h could be full of garbage
% A client could contact every collector node with the same elements, and every collector would broadcast (h, sn), with n being the n-collector. This could cause that hash h consolidates, without anyone really having checking elements in h. Is that bad? not too bad, only an empty epoch would occur. Possible fixes: collector hashes elements + pubkey, chain of signatures

\subsubsection{Consolidation Epoch}\label{subsubsec:consolidation}

In this implementation, epochs are defined by all the elements in a hash, similar to
Compresschain implementation.
%
In other words, elements from the same consolidated hash belong to the same
epoch.
%
%Now we have to be more careful when adding elements to the setchain, elements
%may appear several times depending on how nodes form batches and
%hashes. \marga{esto no puede pasar tambien en compresschain? un
%  elemento no puede aparecer en varios batches?}
%
% However, what epoch number is that? We need to be more specific
% to define this.

%

In this paper, we present two alternatives to assign an epoch to a consolidated
hash.
%
The first one is called \textit{Current Epoch Consolidation} (CEC), meaning that
once a hash consolidates, all their underlying elements are added to the current
epoch (i.e the epoch in which the \texttt{SIGNATURES\_PER\_HASH}-th signature
has been seen).
%
The second one, \textit{First Seen Epoch Consolidation} (FSEC) assigns the epoch
according to when the hash has been seen for the very first time, once
consolidated.
%
Transactions (hashes in this case) in Tendermint blocks are totally
ordered, so both strategies can determine which hash occurs or
consolidates first, and all correct nodes will agree on it.

%

In Figure~\ref{fig:consolidation_epoch}, we show an example of these two different
strategies to assign an epoch to a consolidated hash when
\texttt{SIGNATURES\_PER\_HASH} is 3.
%
Hash $j$ is the first hash in block $m$, and hash $k$ is the hash that first
consolidates (getting its third signature).
%
On one hand, current Epoch Consolidation strategy assigns the epoch as soon as the hash
consolidates, stamping elements from hash $k$ with epoch $a$ first, and then
elements from hash $j$ with epoch $a+1$.
%
On the other hand, First Seen Epoch Consolidation implementation does not assign the
epoch to hash $k$ just when it consolidates, because hash $j$ has been seen for first
time before hash $k$.
%
As hash $j$ consolidates right after hash $k$, elements from hash $j$ are
stamped with epoch $a$, and then elements from hash $k$ are stamped with epoch
$a+1$.

\begin{figure}
  \centering
  \includegraphics[scale=0.5]{figures/consolidation_epoch.pdf}
  \caption{Consolidation epoch strategies}
  \label{fig:consolidation_epoch}
\end{figure}

First Seen Epoch Consolidation variant grants hashes a grace period in which
they can be either consolidated or rejected.
%
% The grace period is necessary, let
% us briefly examine why.
Assuming no grace period at all, if a hash $j$ occurs for
first time before hash $k$, once consolidated, elements from hash $j$ would be
stamped with epoch $E$, while elements from hash $k$ would be stamped with epoch
$F$ \textgreater \ $E$.
%
Hence, let $j$ be a hash first seen at a given moment such
that, once consolidated, it would be assigned the epoch $E$.
%
If hash $j$ never achieves the \texttt{SIGNATURES\_PER\_HASH} signatures to
consolidate, a newer hash $k$ (a hash first seen after $j$) would not be able to
consolidate, as its underlying elements could not be stamped with an epoch $F$
\textgreater \ $E$, if epoch $E$ was not defined yet.
%
In Figure~\ref{fig:grace_period}, we show an example of this.
%
We conclude that a grace period is necessary to guarantee the successful evolution of
the hashchain, as hashes $k$ and $l$ (and even $o$ and $q$) are not able to
consolidate until $j$ consolidates or gets rejected.


It is worth noting that the crucial aspect to consolidation in FSEC lies in the
occurrence of all \texttt{SIGNATURES\_PER\_HASH} signatures within a window
equivalent to the duration of the grace period.
%
Hashes that are initially rejected can consolidate in the future as long as
they reach the \texttt{SIGNATURES\_PER\_HASH} signatures within the grace period.
%
Hashes achieving the \texttt{SIGNATURES\_PER\_HASH}-th signature after
the grace period will not undergo consolidation in the current round;
however, they can do it in the future.

\begin{figure}
  \centering
  \includegraphics[scale=0.5]{figures/grace_period.pdf}
  \caption{Scenario requiring grace period under FSEC.}
  \label{fig:grace_period}
\end{figure}

%\subsubsection{Elements Membership Proof}
% Do not use this because we want a proof that specific elements belong to a specific epoch.
%Client processes do not know if they are contacting a Byzantine or correct process. The general idea of a client protocol that wants to ensure that an element is going to be added to the \setchain is to interact with enough servers to guarantee that some are correct. To guarantee contacting at least one correct server, the client needs to send f + 1 messages. However, each element added to the \setchain comes from a consolidated hash. That means that we already have \textbf{SIGNATURES\_PER\_HASH} validator's signatures for that hash. Those signatures may act as a proof for the batch of elements that really belongs to the \setchain. If that proof is added to the epoch as an extra element, a client sending an element, waiting for a while, and then requesting a Get to only one server (hoping it is correct server) may be sure that its element belongs to the \setchain because the epoch comes along a proof of their elements. 

% Setchain API - 
\begin{figure}[t!]
  \begin{adjustbox}{minipage=[t]{\columnwidth}}
    \begin{algorithm}[H]
      \renewcommand{\thealgorithm}{API Hashchain}         
      \caption{}%
      \label{alg:api-hashchain}%
      \small
      \begin{algorithmic}[1]
      
            \Function{\<add>}{$element$}\label{alg:hash_add}
                \State \textbf{return} \texttt{HashCollector.AddElement($element$)}
                \Comment {Usar el componente intermedio.}
            \EndFunction
      
            \Function{\<get>}{\null}\label{alg:hash_get}
                	\State \textbf{return} \<abciquery>()
            \EndFunction
            
        \end{algorithmic}
      \end{algorithm}
	\end{adjustbox}
  \end{figure}


\subsubsection{Implementation details}\label{subsubsec:details}

We do not define a new collector for Hashcahin since it is similar to the one
define for Compresschain where the only difference is the use of hash functions.
%
% No collector piece for Hashchain appears as it is similar to the already
% presented collector for Brotli, employing hash functions instead of Brotli
% compression.
We define the Setchain API in Algorithm ~\ref{alg:api-hashchain}.
% shows how a new collector is
% used now in order to add elements to the \setchain.

Finally, as regards the ABCI, we show a possible implementation for Hashchain
with Current Epoch Consolidation strategy in Algorithm~\ref{alg:abci-hash1} and ~\ref{alg:abci-hash2}. It is
noteworthy that within the algorithm, the function \<Query> initiates the construction
of \<history>, in addition to broadcasting epoch proofs. Although this approach is employed
for clarity, a node that receives infrequent query requests may experiment delays in
generating the epoch proofs. As a result, the process of constructing \<history> and
epoch proofs can be scheduled periodically, independent of query requests.

% Hash ABCI - alg4
\begin{figure}[t!]
  \begin{adjustbox}{minipage=[t]{\columnwidth}}
    \begin{algorithm}[H]
      \renewcommand{\thealgorithm}{Current Epoch Consolidation - Part 1}         
      \caption{\small ABCI implementation of HashChain - Part 1}%
      \label{alg:abci-hash1}%
      \small
      \begin{algorithmic}[1]
            \State \textbf{Init:} \texttt{epoch} $\leftarrow$ \textbf{0}, \texttt{history} $\leftarrow$ \{\}, \texttt{hashToSignatures} $\leftarrow$ \{\}, \texttt{epochToHash} $\leftarrow$ \{\}, \texttt{hashToBatch} $\leftarrow$ \{\}

            \Function{\<CheckTx>}{$hash, signature$}\label{alg:hash_check_tx}
            		\Comment{Discuss how the validation is done}
				\If {\<isValidSignature>($hash, signature$)}
            			\If {\texttt{hashToBatch}[$hash$] \textbf{exists}}
            				\State \textbf{return} \Call{\<isValidBatch>}{\texttt{hashToBatch}[$hash$]}
            			\EndIf
            		
            			\Comment{If the node does not have the original batch, it spawns an asynchronous routine that requests the batch.}
            		
                		\State \textbf{spawns} \Call{\<revertHash>}{$hash, signature$}
                		\State \textbf{return True}
                		\Comment {In case of absence of information, it returns True.}
                	\Else
                		\State \textbf{return False}
                	\EndIf
            		\EndFunction
      
            \Function{\<DeliverTx>}{$hash, signature$}\label{alg:hash_deliver_tx}
            		\If {\<isValidSignature>($hash, signature$)}
            			\State \texttt{hashToSignatures}[$hash$] $\leftarrow$ \texttt{hashToSignatures}[$hash$] $\cup$  $signature$
            			\If {\Call{\<shouldConsolidateHash>}{$hash$}}
            				\State \texttt{epochToHash[epoch]} $\leftarrow$ $hash$
            				\State \texttt{epoch} $\leftarrow$ \texttt{epoch} + 1
            				\If {not \texttt{hashToBatch}[$hash$] \textbf{exists}}
                				\State \textbf{spawns} \Call{\<revertHash>}{$hash, signature$}
                			\EndIf
               	 	\EndIf
               	 \EndIf
                \State \textbf{return}
            \EndFunction
            
            \Function{\<Query>}{\null}
            		\State \texttt{lastEpochInHistory} $\leftarrow$ max(\texttt{history})
            		\For{\texttt{i in (\texttt{lastEpochInHistory, epoch})}}
            			\State \texttt{hash} $\leftarrow$  \texttt{epochToHash[i]}
             		\If {\texttt{hashToBatch}[\texttt{hash}] \textbf{exists}}
                			\State elements $\leftarrow$ \Call{\<getElementsFromBatch>}{\texttt{hashToBatch}[\texttt{hash}]}
            				\For{\texttt{e in elements}}
             				\If {\<isValidElement>(\texttt{e}) and not \texttt{e} in \texttt{history}}
                					\State \texttt{history[i]}.AddElement(e)
                					\Comment{Only add new valid elements.}
                    	 		\EndIf
                    	 	\EndFor
                    	    \State \texttt{epochHash} $\leftarrow$ \<Hash>(\texttt{history[i]}, \texttt{i})
            				%\Comment{Hash epoch (elements and number).}
                			\State \texttt{epochProof} $\leftarrow$  \texttt{Sign(\texttt{epochHash}, privateKey)}
               			\State \Call{\<add>}{\texttt{epochProof}}
               		 \Else
               		 	\State \textbf{break}
                    	\EndIf
                	\EndFor
            		\State \textbf{return} \texttt{history}
            	\EndFunction

        \end{algorithmic}
      \end{algorithm}
	\end{adjustbox}
  \end{figure}
  
  \begin{figure}[t!]
  \begin{adjustbox}{minipage=[t]{\columnwidth}}
    \begin{algorithm}[H]
      \renewcommand{\thealgorithm}{Current Epoch Consolidation - Part 2}         
      \caption{\small ABCI implementation of HashChain - Part 2}%
      \label{alg:abci-hash2}%
      \small
      \begin{algorithmic}[1]
            	\Function{\<revertHash>}{$hash, signature$}\label{alg:hash_revert}
                \State \texttt{originalBatch} $\leftarrow$ requestHash($hash, signature$)
                \If {Hash(\texttt{originalBatch}) = $hash$}
					\State \texttt{hashToBatch}[$hash$]  $\leftarrow$ \texttt{originalBatch}     
                		\If {\Call{\<isValidBatch>}{\texttt{originalBatch}}}
                			\State \texttt{mySignature} $\leftarrow$ \<Sign>($hash$, privateKey)
                			\State \texttt{Tendermint.Broadcast}($hash$, mySignature)
					\EndIf   
				\EndIf             	
                	\State \textbf{return}
            \EndFunction
            
             \Function{\<shouldConsolidateHash>}{$hash$}\label{alg:hash_consolidated}
            		\State \textbf{return} \textbf{\#}\texttt{hashToSignatures}[$hash$] = \texttt{SIGNATURES\_PER\_HASH}
            \EndFunction

        \end{algorithmic}
      \end{algorithm}
	\end{adjustbox}
  \end{figure}
  
%As mentioned above, Hashchain First Seen Consolidation version uses an idea of grace period in which the hashes are either consolidated or discarded. In this case, we will define the grace period in terms of number of blocks, meaning that if a hash has been seen for the first time in block $B$, and the grace period is $P$ blocks, then on block $B$ + $P$ we will determine if the hash is consolidated (hence, their elements are added to the \setchain), or is discarded. To implement this, we will keep a fixed capacity queue (with capacity $P + 1$), in which the \textit{i}-th element has a list of hashes seen for first time in block $current block - (P - i)$. Notice that the first element of the queue has the list of hashes that were seen for first time in block $current block - P$, meaning that those hashes will be decided in the current block. The last element of the queue has the list of hashes first seen at the current block. Algorithm ~\ref{alg:abci-hash-first-seen} illustrates this behavior.

%% Hash ABCI - alg4
\begin{figure}[t!]
  \begin{adjustbox}{minipage=[t]{\columnwidth}}
    \begin{algorithm}[H]
      \renewcommand{\thealgorithm}{ABCI Hashchain}         
      \caption{\small First Seen Consolidation}%
      \label{alg:abci-hash-first-seen}%
      \small
      \begin{algorithmic}[1]
            \State \textbf{Init:} \texttt{epoch} $\leftarrow$ \textbf{0}, \texttt{history} $\leftarrow$ \{\}, \texttt{hashToOriginal} $\leftarrow$ \{\}, \texttt{hashToSignatures} $\leftarrow$ \{\}, \texttt{mBlocksHashes} $\leftarrow$ \{\}
      
            \Function{DeliverTx}{$transaction$}\label{alg4:deliver_tx}
            		\State \texttt{hash, signature} $\leftarrow$ $transaction$
            		\State \texttt{hashToSignatures}[\texttt{hash}] $\leftarrow$ \texttt{hashToSignatures}[\texttt{hash}] $\cup$  \texttt{signature}
            		\If {not (\texttt{hash} in \texttt{mBlocksHashes})}
					\State \texttt{mBlocksHashes}.Push(\texttt{hash}) 
            		\EndIf
            		\If {not \texttt{hashToOriginal}[ \texttt{hash}] \textbf{exists}}
            			\State \textbf{spawns} asyncRevertHash($transaction$)
                	\EndIf
                \State \textbf{return}
            \EndFunction
            
            \Function{EndBlock}{}
            		\State \texttt{hashesToDefine} $\leftarrow$ \texttt{mBlocksHashes}.Pop()
            		\For{\texttt{hash in hashesToDefine}}
            			\If {shouldConsolidateHash(\texttt{hash})}
            				\If {\texttt{hashToOriginal}[\texttt{hash}] \textbf{exists}} 				
            					\If {isValidBatch(\texttt{hashToOriginal}[\texttt{hash}])}
            				
            						\State elements $\leftarrow$ getElementsFromBatch(\texttt{hashToOriginal}[ \texttt{hash}])
            		
            						%\\Comment{A transaction may contain invalid elements. Only add the valid ones.}
            						\State addElements(\texttt{hash})
            					\EndIf
            				\Else
            					\Comment{Unusual but still a problem}
            				\EndIf
					\Else
						  \Comment{Discard hash}          			
            			\EndIf
            		\EndFor
            \EndFunction
            
        \end{algorithmic}
      \end{algorithm}
	\end{adjustbox}
  \end{figure}


\end{comment}

%%% Local Variables:
%%% TeX-master: "article.tex"
%%% TeX-PDF-mode: t
%%% End:


% %% Executoin Mechanisms
% %\input{ExecutionMechanisms}


% %% Impossibility results
% %\input{results}


% %% Leading Example?
% %\input{Studycase}

% %% Conclusions
% \section{Conclusion}\label{sec:conclusion}
%
Blockchain technologies are facing the problem of scalability.
%
Compared to other transaction-based systems, blockchains are just too slow.
%
We have (at least) two possible ways to solve these problems, either we design
faster consensus algorithms or place clever solutions on top of blockchains.
%
One example of the latter is what are called \emph{Layer-2} solutions where we
have a computing machine on top of the blockchain consuming less resources.
%
An example of the former is Setchain.

Setchain presents a faster consensus algorithm based on set-consensus lifting
the total order of transactions of classic blockchains.
% and uses another order
% instead.
%
The partial order Setchain uses is no-order, i.e. all elements in an epoch do
not follow an order.
%
However, we can say that elements from one epoch come before or later to another
epoch.
%
In other words, we have a relative order, we can relate elements from different
epochs, but we cannot relate elements within the same epoch.

Although Setchain is a clever idea, it lacks a real-world implementation.
%
The authors presenting Setchain gave a prove-of-concept implementation to
observe that Setchain is faster than other algorithms in laboratory conditions.
%
Implementing and verifying such claims are the goals of this work in progress.

Implementing distributed algorithms is hard, and thus, we took an off-the-shelf
solution: Tendermint.
%
Tendermint offers a real-world implementation of a blockchain-based structure
presenting a clear separation of the blockchain level primitives and the user
defined interface.

We presented three implementations of Setchain on top of Tendermint.
%
Since the algorithms of Setchain presented in~\cite{Capretto.2022.Setchain}
employ a set-consensus algorithm, we can not replicate the same algorithms.
%
However, we took a slightly different approach trying to get as much as we can
from what Tendermint can offer.
%
Each implementation tries to get closer to an ideal implementation of Setchain
by exploiting the idea that each epoch is a set of elements.
%
The first implementation (see Section~\ref{subsec:vanilla}) presents a naive and
correct but inefficient implementation of Setchain where Tendermint only
maintains a grow-only-set.
%
The second implementation (see Section~\ref{subsec:compresschain}) improves the
naive implementation using a compression algorithm: transactions are now a
(compressed) set of elements.
%
Finally, the third algorithm~\footnote{Implementation lacking at the moment of
writing the document.} (see Section~\ref{subsec:hashchain}) goes beyond the
second implementation and uses hash functions to reduce network overload.

While reducing network traffic, the use of hash functions requires a way to get
the elements back.
%
To solve this, we designed a \textbf{new} distributed Byzantine-tolerant
algorithm working as a hash-resolution distributed object.
%
Putting both solutions together, we would have a real-world Setchain
implementation.

Hashchain is the closest solution to Setchain without using a set-consensus
algorithm.
%
Since nodes do not know \emph{a priory} the elements inside a hashed set,
consensus then consists in choosing what epoch is going to go next.
%
However, since in Tendermint we have a mempool and a gossip network, elements
are gossiped around, and hashes have the biggest set of elements known to the
node proposing the next epoch.

Our results are exciting but inconclusive.
%
On the one hand, we presented three algorithms ready to be deployed implementing
Setchain specification.
%
While on the other hand, we have not yet tested and experimented on the most
interesting one: Hashchain.
%
Leaving the implementation of Hashchain as future work.
%
Finally, we also want to run intense tests on each of the implementations
employing Tendermint deployment systems.

%%% Local Variables:
%%% TeX-master: "article.tex"
%%% TeX-PDF-mode: t
%%% End:


% %\vfill\pagebreak

% \bibliographystyle{abbrv}% the mandatory bibstyle
% \bibliography{bibfile}

% \vfill\pagebreak

% \newcounter{backup}
%  \appendix
% \input{Appendix/Mechanisms.Equiv.tex}

% \vfill\pagebreak

% \input{Appendix/BFS.ImpProof}

% \vfill\pagebreak

% \input{Appendix/DFS.Proofs}

% \vfill\pagebreak


% \input{Appendix/FlashLoans.tex}
\end{document}

%%% Local Variables:
%%% mode: latex
%%% TeX-master: t
%%% TeX-PDF-mode: t
%%% End:
